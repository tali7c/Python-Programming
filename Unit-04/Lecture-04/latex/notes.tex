\documentclass[11pt]{article}
\usepackage[utf8]{inputenc}
\usepackage[T1]{fontenc}
\usepackage{geometry}
\usepackage{amsmath}
\usepackage{hyperref}
\usepackage{xcolor}
\usepackage{listings}
\geometry{margin=1in}

\lstset{
  basicstyle=\ttfamily\small,
  keywordstyle=\color{blue},
  commentstyle=\color{gray},
  stringstyle=\color{teal},
  showstringspaces=false,
  columns=fullflexible,
  frame=single,
  framerule=0.2pt
}

\title{Python Programming\\Unit 04 -- Lecture 04 Notes\\Tkinter + Databases (Concepts and Integration)}
\author{Tofik Ali}
\date{\today}

\begin{document}
\maketitle
\tableofcontents

\section{Lecture Overview}
Many GUI apps need to store data persistently: users, tasks, marks, inventory, etc.
Files can work for simple cases, but databases provide:
\begin{itemize}
  \item structured storage,
  \item fast searching,
  \item constraints (unique IDs),
  \item and transactions for safe updates.
\end{itemize}

\section{Core Concepts}

\subsection{Relational Databases (SQL)}
Relational databases store data in \textbf{tables}.
Each table has:
\begin{itemize}
  \item columns (fields), e.g., \texttt{name}, \texttt{sapid}
  \item rows (records), one row per student
\end{itemize}

\textbf{Primary key:} a column (or set of columns) that uniquely identifies a row.
Examples: \texttt{id}, \texttt{sapid}.

\subsection{NoSQL Databases}
NoSQL databases store data in different formats depending on the type:
\begin{itemize}
  \item document (MongoDB),
  \item key-value (Redis),
  \item graph (Neo4j),
  \item column-family (Cassandra).
\end{itemize}
They often offer flexible schemas and scale for certain use-cases.

\subsection{SQLite for Learning and Small Apps}
SQLite is an embedded relational database:
\begin{itemize}
  \item no server is required,
  \item database is stored as a single file,
  \item Python provides \texttt{sqlite3} in the standard library.
\end{itemize}

\subsection{GUI-to-Database Workflow}
Typical workflow:
\begin{enumerate}
  \item user enters values in Entry widgets,
  \item program validates input,
  \item program executes SQL (INSERT/SELECT/UPDATE/DELETE),
  \item program refreshes UI (Listbox/Table) to show latest data.
\end{enumerate}

\subsection{Parameterised Queries (Important)}
Never build SQL by string concatenation with user input (SQL injection risk).
Use parameters:
\begin{lstlisting}[language=Python]
cur.execute("INSERT INTO students(name, sapid) VALUES (?, ?)", (name, sapid))
\end{lstlisting}

\section{Demo Walkthrough}
\textbf{File:} \texttt{demo/tkinter\_sqlite\_student\_app.py}

Key things to observe:
\begin{itemize}
  \item Create the database file in \texttt{data/} automatically.
  \item Create table using \texttt{CREATE TABLE IF NOT EXISTS}.
  \item Insert rows using parameterised queries.
  \item Refresh a Listbox with the latest records.
  \item Commit changes after INSERT.
\end{itemize}

\section{Interactive Checkpoints (with Solutions)}

\subsection*{Checkpoint 1 Solution}
\textbf{Question:} One advantage of databases over text files?

\textbf{Answer (examples):} fast queries/search, data integrity via constraints,
structured schema, transactions.

\subsection*{Checkpoint 2 Solution}
\textbf{Question:} What is a primary key?

\textbf{Answer:} A primary key uniquely identifies each row and prevents duplicates
for that key value.

\section{Practice Exercises (with Solutions)}

\subsection*{Exercise 1: Create a Table}
\textbf{Task:} Write SQL to create a table \texttt{students} with columns \texttt{id} (primary key) and \texttt{name}.

\textbf{Solution:}
\begin{lstlisting}
CREATE TABLE IF NOT EXISTS students (
  id   INTEGER PRIMARY KEY,
  name TEXT NOT NULL
);
\end{lstlisting}

\subsection*{Exercise 2: Insert a Student Safely}
\textbf{Task:} Insert (name, sapid) using parameters.

\textbf{Solution:}
\begin{lstlisting}[language=Python]
cur.execute(
    "INSERT INTO students(name, sapid) VALUES (?, ?)",
    (name, sapid)
)
con.commit()
\end{lstlisting}

\section{Exit Question (with Solution)}
\textbf{Question:} SQL to create \texttt{students(id, name)} with \texttt{id} as primary key?

\textbf{Answer:}
\begin{lstlisting}
CREATE TABLE students (
  id   INTEGER PRIMARY KEY,
  name TEXT
);
\end{lstlisting}

\end{document}

