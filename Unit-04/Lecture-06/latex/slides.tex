\documentclass{beamer}

\usetheme{Berlin}
\usecolortheme{Orchid}
\useoutertheme{miniframes}
\setbeamertemplate{navigation symbols}{}

\usepackage{amsmath}
\usepackage{booktabs}
\usepackage{graphicx}
\usepackage{xcolor}
\usepackage{listings}
\usepackage{hyperref}

\lstset{
  basicstyle=\ttfamily\small,
  keywordstyle=\color{blue},
  commentstyle=\color{gray},
  stringstyle=\color{teal},
  showstringspaces=false
}

\title[Python Programming]{Python Programming}
\subtitle{Unit 04 -- Lecture 06: Transactions and Database Error Handling}
\author{Tofik Ali}
\institute{School of Computer Science, UPES Dehradun}
\date{\today}

\begin{document}

\begin{frame}[fragile]
  \titlepage
  \vspace{0.5em}
  \begin{center}
  \footnotesize Repository: \texttt{https://github.com/tali7c/Python-Programming}
  \end{center}
\end{frame}

\begin{frame}[fragile]{Quick Links}
  \centering
  \hyperlink{sec:core}{\beamerbutton{Core Concepts}}\hspace{1em}
  \hyperlink{sec:demo}{\beamerbutton{Demo}}\hspace{1em}
  \hyperlink{sec:interactive}{\beamerbutton{Interactive}}\hspace{1em}
  \hyperlink{sec:summary}{\beamerbutton{Summary}}
\end{frame}

\begin{frame}[fragile]{Agenda}
  \tableofcontents
\end{frame}

\section{Core Concepts}
\label{sec:core}

\begin{frame}[fragile]{Learning Outcomes}
  \begin{itemize}[<+->]
    \item Explain what a transaction is and why it matters
    \item Use \texttt{commit} and \texttt{rollback} correctly
    \item Handle common database errors (constraints, invalid input)
    \item Write safer DB code using try/except and context managers
  \end{itemize}
\end{frame}

\begin{frame}[fragile]{What is a Transaction?}
  \begin{itemize}[<+->]
    \item A transaction is a group of operations treated as one unit
    \item Either \textbf{all succeed} (commit) OR \textbf{none apply} (rollback)
    \item Prevents partial updates (data corruption)
  \end{itemize}
\end{frame}

\begin{frame}[fragile]{Commit vs Rollback}
  \begin{itemize}[<+->]
    \item \texttt{commit()} saves changes permanently
    \item \texttt{rollback()} cancels changes since last commit
    \item Always rollback on failure in multi-step updates
  \end{itemize}
\end{frame}

\begin{frame}[fragile]{Typical Safe Pattern}
  \begin{lstlisting}[language=Python]
import sqlite3

con = sqlite3.connect("app.db")
try:
    con.execute("INSERT ...")
    con.execute("UPDATE ...")
    con.commit()
except sqlite3.Error as e:
    con.rollback()
    print("DB Error:", e)
finally:
    con.close()
  \end{lstlisting}
\end{frame}

\begin{frame}[fragile]{Common DB Errors}
  \begin{itemize}[<+->]
    \item \texttt{IntegrityError}: constraint failed (duplicate unique key)
    \item \texttt{OperationalError}: SQL error, missing table, locked DB
    \item Input issues: wrong types, empty fields
  \end{itemize}
\end{frame}

\section{Demo}
\label{sec:demo}

\begin{frame}[fragile]{Demo: Rollback on Failure}
  \begin{itemize}[<+->]
    \item File: \texttt{demo/transactions\_rollback\_demo.py}
    \item Inserts two rows where the second violates a UNIQUE constraint
    \item Demonstrates that rollback prevents partial insertion
  \end{itemize}
\end{frame}

\section{Interactive}
\label{sec:interactive}

\begin{frame}[fragile]{Checkpoint 1}
  \textbf{Question:} What problem can happen if you do not use transactions for multi-step updates?
\end{frame}

\begin{frame}[fragile]{Checkpoint 2}
  \textbf{Question:} Which exception might occur when you insert a duplicate unique value?
\end{frame}

\begin{frame}[fragile]{Think-Pair-Share}
  Discuss:
  \begin{itemize}
    \item In a banking app, why is rollback critical?
  \end{itemize}
\end{frame}

\section{Summary}
\label{sec:summary}

\begin{frame}[fragile]{Key Takeaways}
  \begin{itemize}[<+->]
    \item Transactions ensure all-or-nothing updates
    \item Use commit on success and rollback on failure
    \item Handle constraint errors and operational errors cleanly
    \item Always close DB connections (use try/finally or context managers)
  \end{itemize}
\end{frame}

\begin{frame}[fragile]{Exit Question}
  In one sentence: what does \texttt{rollback()} do?
\end{frame}

\end{document}

