\documentclass{beamer}

\usetheme{Berlin}
\usecolortheme{Orchid}
\useoutertheme{miniframes}
\setbeamertemplate{navigation symbols}{}

\usepackage{amsmath}
\usepackage{booktabs}
\usepackage{graphicx}
\usepackage{xcolor}
\usepackage{listings}
\usepackage{hyperref}

\lstset{
  basicstyle=\ttfamily\small,
  keywordstyle=\color{blue},
  commentstyle=\color{gray},
  stringstyle=\color{teal},
  showstringspaces=false
}

\title[Python Programming]{Python Programming}
\subtitle{Unit 04 -- Lecture 01: Tkinter Introduction and Widgets}
\author{Tofik Ali}
\institute{School of Computer Science, UPES Dehradun}
\date{\today}

\begin{document}

\begin{frame}[fragile]
  \titlepage
  \vspace{0.5em}
  \begin{center}
  \footnotesize Repository: \texttt{https://github.com/tali7c/Python-Programming}
  \end{center}
\end{frame}

\begin{frame}[fragile]{Quick Links}
  \centering
  \hyperlink{sec:core}{\beamerbutton{Core Concepts}}\hspace{1em}
  \hyperlink{sec:demo}{\beamerbutton{Demo}}\hspace{1em}
  \hyperlink{sec:interactive}{\beamerbutton{Interactive}}\hspace{1em}
  \hyperlink{sec:summary}{\beamerbutton{Summary}}
\end{frame}

\begin{frame}[fragile]{Agenda}
  \tableofcontents
\end{frame}

\section{Core Concepts}
\label{sec:core}

\begin{frame}[fragile]{Learning Outcomes}
  \begin{itemize}[<+->]
    \item Create a basic Tkinter window and run the GUI loop
    \item Understand widgets and how they are placed in a window
    \item Use common widgets: Label, Button, Entry, Text, Canvas, Listbox
    \item Use selection widgets: Checkbutton, Radiobutton, Combobox, Spinbox
  \end{itemize}
\end{frame}

\begin{frame}[fragile]{What is Tkinter?}
  \begin{itemize}[<+->]
    \item Tkinter is Python's standard GUI library
    \item Allows building desktop apps (forms, tools, mini-projects)
    \item Works with widgets (buttons, text boxes, etc.)
  \end{itemize}
\end{frame}

\begin{frame}[fragile]{Smallest Tkinter App}
  \begin{lstlisting}[language=Python]
import tkinter as tk

root = tk.Tk()
root.title("My App")
root.geometry("400x200")
root.mainloop()
  \end{lstlisting}
  \begin{itemize}[<+->]
    \item \texttt{Tk()} creates the main window
    \item \texttt{mainloop()} keeps the window running
  \end{itemize}
\end{frame}

\begin{frame}[fragile]{Widgets (GUI Building Blocks)}
  \begin{itemize}[<+->]
    \item Widgets are UI components: Label, Button, Entry, etc.
    \item Most widgets need:
      \begin{itemize}
        \item a parent (usually \texttt{root} or a Frame)
        \item configuration options (text, width, command, ...)
      \end{itemize}
  \end{itemize}
\end{frame}

\begin{frame}[fragile]{Label, Button, Entry (Example)}
  \begin{lstlisting}[language=Python]
import tkinter as tk

root = tk.Tk()
name_var = tk.StringVar()

tk.Label(root, text="Name:").pack()
tk.Entry(root, textvariable=name_var).pack()

def greet():
    print("Hello,", name_var.get())

tk.Button(root, text="Greet", command=greet).pack()
root.mainloop()
  \end{lstlisting}
\end{frame}

\begin{frame}[fragile]{More Widgets}
  \begin{itemize}[<+->]
    \item \texttt{Text}: multi-line text input/output
    \item \texttt{Canvas}: drawing shapes
    \item \texttt{Scrollbar}: scrolling support (Text/Listbox)
    \item \texttt{Listbox}: list selection
  \end{itemize}
\end{frame}

\begin{frame}[fragile]{Selection Widgets}
  \begin{itemize}[<+->]
    \item \texttt{Checkbutton}: multiple independent selections
    \item \texttt{Radiobutton}: choose exactly one option
    \item ttk \texttt{Combobox}: dropdown list
    \item \texttt{Spinbox}: numeric selection
  \end{itemize}
\end{frame}

\section{Demo}
\label{sec:demo}

\begin{frame}[fragile]{Demo: Widget Showcase}
  \begin{itemize}[<+->]
    \item File: \texttt{demo/widgets\_showcase.py}
    \item Shows:
      \begin{itemize}
        \item Label, Button, Entry
        \item Checkbutton and Radiobutton
        \item Listbox and Text
        \item ttk Combobox
      \end{itemize}
  \end{itemize}
\end{frame}

\section{Interactive}
\label{sec:interactive}

\begin{frame}[fragile]{Checkpoint 1}
  \textbf{Question:} Why do we need \texttt{root.mainloop()} in Tkinter?
\end{frame}

\begin{frame}[fragile]{Checkpoint 2}
  \textbf{Question:} What is the benefit of using \texttt{StringVar} with an \texttt{Entry} widget?
\end{frame}

\begin{frame}[fragile]{Think-Pair-Share}
  Discuss:
  \begin{itemize}
    \item What makes a GUI \emph{easy to use}?
    \item Give 3 design rules (labels, spacing, validation, feedback, etc.).
  \end{itemize}
\end{frame}

\section{Summary}
\label{sec:summary}

\begin{frame}[fragile]{Key Takeaways}
  \begin{itemize}[<+->]
    \item Tkinter builds GUI apps using widgets
    \item \texttt{Tk()} creates a window; \texttt{mainloop()} runs the event loop
    \item Widgets like Label/Button/Entry handle input and actions
    \item Selection widgets (check/radio/combobox) capture choices
  \end{itemize}
\end{frame}

\begin{frame}[fragile]{Exit Question}
  Write the 3 key lines needed to create a Tkinter window titled \texttt{"Hello GUI"}
  and keep it running.
\end{frame}

\end{document}

