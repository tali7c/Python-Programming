\documentclass[11pt]{article}
\usepackage[utf8]{inputenc}
\usepackage[T1]{fontenc}
\usepackage{geometry}
\usepackage{amsmath}
\usepackage{hyperref}
\usepackage{xcolor}
\usepackage{listings}
\geometry{margin=1in}

\lstset{
  basicstyle=\ttfamily\small,
  keywordstyle=\color{blue},
  commentstyle=\color{gray},
  stringstyle=\color{teal},
  showstringspaces=false,
  columns=fullflexible,
  frame=single,
  framerule=0.2pt
}

\title{Python Programming\\Unit 04 -- Lecture 01 Notes\\Tkinter Introduction and Widgets}
\author{Tofik Ali}
\date{\today}

\begin{document}
\maketitle
\tableofcontents

\section{Lecture Overview}
Tkinter is Python's built-in GUI toolkit. In this unit we learn to build
desktop applications such as:
\begin{itemize}
  \item calculators,
  \item forms (registration, login),
  \item and small utilities.
\end{itemize}
This lecture introduces the Tkinter window, the event loop, and commonly used widgets.

\section{Core Concepts}

\subsection{Basic Tkinter Program Structure}
Every Tkinter app has:
\begin{itemize}
  \item a root window (\texttt{tk.Tk()}),
  \item widgets attached to the window,
  \item and an event loop (\texttt{mainloop()}).
\end{itemize}

\begin{lstlisting}[language=Python]
import tkinter as tk

root = tk.Tk()
root.title("My App")
root.geometry("400x200")
root.mainloop()
\end{lstlisting}

\subsection{Widgets}
Widgets are UI components. Examples:
\begin{itemize}
  \item \texttt{Label}: show text
  \item \texttt{Button}: clickable action
  \item \texttt{Entry}: single-line input
  \item \texttt{Text}: multi-line input/output
  \item \texttt{Listbox}: list selection
  \item \texttt{Canvas}: drawing
  \item \texttt{Checkbutton}, \texttt{Radiobutton}: choices
\end{itemize}

\subsection{Control Variables: \texttt{StringVar}, \texttt{IntVar}}
Tkinter provides variables that connect widget values to your code:
\begin{itemize}
  \item \texttt{StringVar} for text
  \item \texttt{IntVar} for integers (often used for check/radio)
\end{itemize}

\begin{lstlisting}[language=Python]
name_var = tk.StringVar()
entry = tk.Entry(root, textvariable=name_var)
\end{lstlisting}
Now you can read \texttt{name\_var.get()} and set \texttt{name\_var.set("...")}.

\subsection{The Event Loop}
\texttt{mainloop()} listens for events:
\begin{itemize}
  \item button clicks,
  \item key presses,
  \item window close,
  \item etc.
\end{itemize}
Without \texttt{mainloop()}, your window appears briefly (or not at all) and the program ends.

\subsection{ttk Widgets}
\texttt{tkinter.ttk} provides themed widgets like \texttt{Combobox} (dropdown).
They often look nicer and more modern.

\section{Demo Walkthrough}
\textbf{File:} \texttt{demo/widgets\_showcase.py}

This demo shows multiple widgets in one window. Focus on:
\begin{itemize}
  \item how widget values are read (\texttt{StringVar.get()}),
  \item how button callbacks work,
  \item and how selection widgets (check/radio/combobox) store values.
\end{itemize}

\section{Interactive Checkpoints (with Solutions)}

\subsection*{Checkpoint 1 Solution}
\textbf{Question:} Why do we need \texttt{root.mainloop()}?

\textbf{Answer:} It starts the Tkinter event loop, keeps the window alive, and processes user events.

\subsection*{Checkpoint 2 Solution}
\textbf{Question:} Benefit of using \texttt{StringVar} with \texttt{Entry}?

\textbf{Answer:} It provides an easy way to get/set widget values and keeps the UI and code connected.

\section{Practice Exercises (with Solutions)}

\subsection*{Exercise 1: Simple Greeting GUI}
\textbf{Task:} Create an Entry for name and a Button that prints \texttt{"Hello <name>"}.

\textbf{Solution (idea):}
\begin{lstlisting}[language=Python]
import tkinter as tk

root = tk.Tk()
name = tk.StringVar()

tk.Label(root, text="Name:").pack()
tk.Entry(root, textvariable=name).pack()

def greet():
    print("Hello", name.get())

tk.Button(root, text="Greet", command=greet).pack()
root.mainloop()
\end{lstlisting}

\subsection*{Exercise 2: Checkbox + Radio}
\textbf{Task:} Add one checkbox \texttt{"Agree"} and two radio options \texttt{"Male/Female"}.
Print selected values when a button is clicked.

\textbf{Solution (outline):}
\begin{lstlisting}[language=Python]
agree = tk.IntVar()
gender = tk.StringVar(value="Male")

tk.Checkbutton(root, text="Agree", variable=agree).pack()
tk.Radiobutton(root, text="Male", variable=gender, value="Male").pack()
tk.Radiobutton(root, text="Female", variable=gender, value="Female").pack()
\end{lstlisting}

\section{Exit Question (with Solution)}
\textbf{Question:} 3 lines to create a window titled \texttt{"Hello GUI"} and keep it running.

\textbf{Solution:}
\begin{lstlisting}[language=Python]
root = tk.Tk()
root.title("Hello GUI")
root.mainloop()
\end{lstlisting}

\end{document}

