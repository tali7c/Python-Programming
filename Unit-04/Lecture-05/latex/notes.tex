\documentclass[11pt]{article}
\usepackage[utf8]{inputenc}
\usepackage[T1]{fontenc}
\usepackage{geometry}
\usepackage{amsmath}
\usepackage{hyperref}
\usepackage{xcolor}
\usepackage{listings}
\geometry{margin=1in}

\lstset{
  basicstyle=\ttfamily\small,
  keywordstyle=\color{blue},
  commentstyle=\color{gray},
  stringstyle=\color{teal},
  showstringspaces=false,
  columns=fullflexible,
  frame=single,
  framerule=0.2pt
}

\title{Python Programming\\Unit 04 -- Lecture 05 Notes\\DB-API 2.0, CRUD Operations, MongoDB Overview}
\author{Tofik Ali}
\date{\today}

\begin{document}
\maketitle
\tableofcontents

\section{Lecture Overview}
Python connects to relational databases using the DB-API 2.0 pattern:
\begin{itemize}
  \item open a connection,
  \item create a cursor,
  \item execute SQL queries,
  \item fetch results,
  \item commit changes (for INSERT/UPDATE/DELETE),
  \item close connection.
\end{itemize}
This lecture uses SQLite (\texttt{sqlite3}) for hands-on CRUD and discusses MongoDB at a high level.

\section{Core Concepts}

\subsection{DB-API 2.0 Workflow (SQLite Example)}
\begin{lstlisting}[language=Python]
import sqlite3

con = sqlite3.connect("app.db")
cur = con.cursor()

cur.execute("CREATE TABLE IF NOT EXISTS students(id INTEGER PRIMARY KEY, name TEXT)")
con.commit()

cur.execute("SELECT * FROM students")
rows = cur.fetchall()

con.close()
\end{lstlisting}

\subsection{CRUD in SQL}
\begin{itemize}
  \item \textbf{Create:} \texttt{INSERT INTO table (...) VALUES (...)}
  \item \textbf{Read:} \texttt{SELECT ... FROM table WHERE ...}
  \item \textbf{Update:} \texttt{UPDATE table SET ... WHERE ...}
  \item \textbf{Delete:} \texttt{DELETE FROM table WHERE ...}
\end{itemize}

\subsection{Parameterised Queries}
Never build SQL by concatenation with user input:
\begin{lstlisting}[language=Python]
# BAD (do not do this)
cur.execute("SELECT * FROM students WHERE sapid = '" + sapid + "'")
\end{lstlisting}

Use placeholders:
\begin{lstlisting}[language=Python]
cur.execute("SELECT * FROM students WHERE sapid = ?", (sapid,))
\end{lstlisting}
Benefits:
\begin{itemize}
  \item avoids SQL injection,
  \item handles quoting automatically,
  \item reduces bugs.
\end{itemize}

\subsection{Fetching Results}
Common methods:
\begin{itemize}
  \item \texttt{fetchone()} for a single row
  \item \texttt{fetchall()} for all rows
\end{itemize}

\subsection{MongoDB Overview}
MongoDB is a document database:
\begin{itemize}
  \item data is stored as documents (JSON-like),
  \item documents are grouped into collections.
\end{itemize}

Python driver: \texttt{pymongo} (external install). Example (conceptual):
\begin{lstlisting}[language=Python]
# pip install pymongo
from pymongo import MongoClient

client = MongoClient("mongodb://localhost:27017")
db = client["college"]
students = db["students"]
students.insert_one({"name": "Asha", "sapid": "5001", "marks": 88})
\end{lstlisting}

\section{Demo Walkthrough}
\textbf{SQLite CRUD:} \texttt{demo/sqlite\_crud\_demo.py}\\
\textbf{MongoDB (optional):} \texttt{demo/mongodb\_optional\_demo.py}

\section{Interactive Checkpoints (with Solutions)}

\subsection*{Checkpoint 1 Solution}
\textbf{Question:} purpose of \texttt{cursor.execute(...)}?

\textbf{Answer:} It sends an SQL statement (or command) to the database engine to be executed.

\subsection*{Checkpoint 2 Solution}
\textbf{Question:} why parameterised queries are safer?

\textbf{Answer:} They separate SQL code from data values, preventing injection and handling escaping correctly.

\section{Practice Exercises (with Solutions)}

\subsection*{Exercise 1: Update Marks}
\textbf{Task:} Update marks for a student by SAP ID.

\textbf{Solution:}
\begin{lstlisting}[language=Python]
cur.execute("UPDATE students SET marks = ? WHERE sapid = ?", (marks, sapid))
con.commit()
\end{lstlisting}

\subsection*{Exercise 2: Delete a Student}
\textbf{Task:} Delete a row by SAP ID.

\textbf{Solution:}
\begin{lstlisting}[language=Python]
cur.execute("DELETE FROM students WHERE sapid = ?", (sapid,))
con.commit()
\end{lstlisting}

\section{Exit Question (with Solution)}
\textbf{Question:} SQL query to update marks by SAP ID?

\textbf{Answer (pattern):}
\begin{lstlisting}
UPDATE students SET marks = ? WHERE sapid = ?;
\end{lstlisting}

\end{document}

