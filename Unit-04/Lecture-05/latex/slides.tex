\documentclass{beamer}

\usetheme{Berlin}
\usecolortheme{Orchid}
\useoutertheme{miniframes}
\setbeamertemplate{navigation symbols}{}

\usepackage{amsmath}
\usepackage{booktabs}
\usepackage{graphicx}
\usepackage{xcolor}
\usepackage{listings}
\usepackage{hyperref}

\lstset{
  basicstyle=\ttfamily\small,
  keywordstyle=\color{blue},
  commentstyle=\color{gray},
  stringstyle=\color{teal},
  showstringspaces=false
}

\title[Python Programming]{Python Programming}
\subtitle{Unit 04 -- Lecture 05: DB-API 2.0, CRUD Operations, MongoDB Overview}
\author{Tofik Ali}
\institute{School of Computer Science, UPES Dehradun}
\date{\today}

\begin{document}

\begin{frame}[fragile]
  \titlepage
  \vspace{0.5em}
  \begin{center}
  \footnotesize Repository: \texttt{https://github.com/tali7c/Python-Programming}
  \end{center}
\end{frame}

\begin{frame}[fragile]{Quick Links}
  \centering
  \hyperlink{sec:core}{\beamerbutton{Core Concepts}}\hspace{1em}
  \hyperlink{sec:demo}{\beamerbutton{Demo}}\hspace{1em}
  \hyperlink{sec:interactive}{\beamerbutton{Interactive}}\hspace{1em}
  \hyperlink{sec:summary}{\beamerbutton{Summary}}
\end{frame}

\begin{frame}[fragile]{Agenda}
  \tableofcontents
\end{frame}

\section{Core Concepts}
\label{sec:core}

\begin{frame}[fragile]{Learning Outcomes}
  \begin{itemize}[<+->]
    \item Understand the DB-API 2.0 workflow (connection, cursor, execute)
    \item Perform CRUD operations using parameterised queries
    \item Fetch results using \texttt{fetchone} / \texttt{fetchall}
    \item Describe how MongoDB differs from relational databases
  \end{itemize}
\end{frame}

\begin{frame}[fragile]{DB-API 2.0: The Common Pattern}
  \begin{lstlisting}[language=Python]
import sqlite3
con = sqlite3.connect("app.db")
cur = con.cursor()
cur.execute("SELECT * FROM students")
rows = cur.fetchall()
con.close()
  \end{lstlisting}
\end{frame}

\begin{frame}[fragile]{CRUD}
  \begin{itemize}[<+->]
    \item \textbf{C}reate: \texttt{INSERT}
    \item \textbf{R}ead: \texttt{SELECT}
    \item \textbf{U}pdate: \texttt{UPDATE}
    \item \textbf{D}elete: \texttt{DELETE}
  \end{itemize}
\end{frame}

\begin{frame}[fragile]{Parameterised Queries (Important)}
  \begin{itemize}[<+->]
    \item Avoid string concatenation in SQL
    \item Use placeholders to prevent SQL injection and quoting bugs
  \end{itemize}
  \vspace{0.4em}
  \begin{lstlisting}[language=Python]
cur.execute(
    "INSERT INTO students(name, sapid) VALUES (?, ?)",
    (name, sapid)
)
  \end{lstlisting}
\end{frame}

\begin{frame}[fragile]{MongoDB (High-Level View)}
  \begin{itemize}[<+->]
    \item MongoDB is a document database (NoSQL)
    \item Stores JSON-like documents in collections
    \item Useful when schema is flexible or data is nested
    \item Python driver: \texttt{pymongo} (requires installation)
  \end{itemize}
\end{frame}

\section{Demo}
\label{sec:demo}

\begin{frame}[fragile]{Demo: SQLite CRUD + Optional MongoDB Stub}
  \begin{itemize}[<+->]
    \item SQLite CRUD: \texttt{demo/sqlite\_crud\_demo.py}
    \item MongoDB (optional): \texttt{demo/mongodb\_optional\_demo.py}
  \end{itemize}
\end{frame}

\section{Interactive}
\label{sec:interactive}

\begin{frame}[fragile]{Checkpoint 1}
  \textbf{Question:} What is the purpose of \texttt{cursor.execute(...)}?
\end{frame}

\begin{frame}[fragile]{Checkpoint 2}
  \textbf{Question:} Why are parameterised queries safer than string concatenation?
\end{frame}

\begin{frame}[fragile]{Think-Pair-Share}
  Discuss:
  \begin{itemize}
    \item Which operations should be allowed in a student database app:
      Insert, Update, Delete? Why?
  \end{itemize}
\end{frame}

\section{Summary}
\label{sec:summary}

\begin{frame}[fragile]{Key Takeaways}
  \begin{itemize}[<+->]
    \item DB-API uses connection + cursor + execute + fetch + commit
    \item CRUD maps to SQL commands (INSERT/SELECT/UPDATE/DELETE)
    \item Parameterised queries improve safety and correctness
    \item MongoDB stores documents (NoSQL) and is accessed via drivers like \texttt{pymongo}
  \end{itemize}
\end{frame}

\begin{frame}[fragile]{Exit Question}
  Write one SQL query to update marks of a student with a given SAP ID.
\end{frame}

\end{document}

