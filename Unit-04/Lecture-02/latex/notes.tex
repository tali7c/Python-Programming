\documentclass[11pt]{article}
\usepackage[utf8]{inputenc}
\usepackage[T1]{fontenc}
\usepackage{geometry}
\usepackage{amsmath}
\usepackage{hyperref}
\usepackage{xcolor}
\usepackage{listings}
\geometry{margin=1in}

\lstset{
  basicstyle=\ttfamily\small,
  keywordstyle=\color{blue},
  commentstyle=\color{gray},
  stringstyle=\color{teal},
  showstringspaces=false,
  columns=fullflexible,
  frame=single,
  framerule=0.2pt
}

\title{Python Programming\\Unit 04 -- Lecture 02 Notes\\Layout Managers, Menus, Dialogs, Message Boxes}
\author{Tofik Ali}
\date{\today}

\begin{document}
\maketitle
\tableofcontents

\section{Lecture Overview}
After learning widgets, the next challenge is \textbf{layout}:
placing widgets in a clean and consistent way. This lecture covers:
\begin{itemize}
  \item \texttt{pack} and \texttt{grid} layout managers,
  \item using Frames to structure large GUIs,
  \item building menus, and
  \item dialogs and message boxes.
\end{itemize}

\section{Core Concepts}

\subsection{\texttt{pack}}
\texttt{pack} places widgets relative to each other (top/bottom/left/right).
It is great for simple layouts.

\begin{lstlisting}[language=Python]
tk.Label(root, text="Title").pack(pady=10)
tk.Button(root, text="OK").pack()
\end{lstlisting}

Common options:
\begin{itemize}
  \item \texttt{side}: \texttt{"top"}, \texttt{"bottom"}, \texttt{"left"}, \texttt{"right"}
  \item \texttt{fill}: \texttt{"x"}, \texttt{"y"}, \texttt{"both"}
  \item \texttt{expand}: allow expanding in extra space
  \item \texttt{padx/pady}: spacing
\end{itemize}

\subsection{\texttt{grid}}
\texttt{grid} places widgets in rows and columns. It is best for forms.

\begin{lstlisting}[language=Python]
tk.Label(root, text="Name").grid(row=0, column=0, sticky="w")
tk.Entry(root).grid(row=0, column=1, padx=6)
\end{lstlisting}

Useful options:
\begin{itemize}
  \item \texttt{sticky}: alignment inside the cell (\texttt{"nsew"})
  \item \texttt{padx/pady}: spacing
  \item \texttt{columnspan/rowspan}: span multiple cells
\end{itemize}

\textbf{Rule:} Do not mix \texttt{pack} and \texttt{grid} in the same container widget.
(You can use both if they are used in different Frames.)

\subsection{Frames (Containers)}
Frames are used to group widgets.
A common pattern:
\begin{itemize}
  \item Header Frame (title)
  \item Form Frame (input fields)
  \item Buttons Frame (submit/cancel)
\end{itemize}

\subsection{Menus}
Menus provide standard navigation.

\begin{lstlisting}[language=Python]
menubar = tk.Menu(root)
file_menu = tk.Menu(menubar, tearoff=0)
file_menu.add_command(label="Exit", command=root.destroy)
menubar.add_cascade(label="File", menu=file_menu)
root.config(menu=menubar)
\end{lstlisting}

\subsection{Message Boxes and File Dialogs}
Message boxes provide feedback.
File dialogs allow file selection.

\begin{lstlisting}[language=Python]
from tkinter import messagebox, filedialog

messagebox.showinfo("Saved", "Done")
path = filedialog.askopenfilename(title="Select a file")
\end{lstlisting}

\section{Demo Walkthrough}
\textbf{File:} \texttt{demo/layout\_menus\_dialogs\_demo.py}

This demo illustrates:
\begin{itemize}
  \item grid-based layout for a small form,
  \item a menu bar with File/Help,
  \item a message box on submit,
  \item and selecting a file using a file dialog.
\end{itemize}

\section{Interactive Checkpoints (with Solutions)}

\subsection*{Checkpoint 1 Solution}
\textbf{Question:} Why not mix \texttt{pack} and \texttt{grid} in the same container?

\textbf{Answer:} Tkinter can raise errors and produce unpredictable layout behavior.
Use one layout manager per container (Frame/root), but you can use different managers
in different Frames.

\subsection*{Checkpoint 2 Solution}
\textbf{Question:} Purpose of Frame?

\textbf{Answer:} A Frame groups widgets, improves structure, and allows using separate
layout strategies inside different UI sections.

\section{Practice Exercises (with Solutions)}

\subsection*{Exercise 1: Create a Simple Form}
\textbf{Task:} Create a GUI with two fields (Name, Age) using \texttt{grid} and a Submit button.

\textbf{Solution (outline):}
\begin{lstlisting}[language=Python]
import tkinter as tk

root = tk.Tk()

tk.Label(root, text="Name").grid(row=0, column=0, sticky="w")
tk.Entry(root).grid(row=0, column=1)

tk.Label(root, text="Age").grid(row=1, column=0, sticky="w")
tk.Entry(root).grid(row=1, column=1)

tk.Button(root, text="Submit").grid(row=2, column=0, columnspan=2)
root.mainloop()
\end{lstlisting}

\subsection*{Exercise 2: Add an Exit Menu}
\textbf{Task:} Add a menu item \texttt{File -> Exit}.

\textbf{Solution (idea):}
\begin{lstlisting}[language=Python]
menubar = tk.Menu(root)
file_menu = tk.Menu(menubar, tearoff=0)
file_menu.add_command(label="Exit", command=root.destroy)
menubar.add_cascade(label="File", menu=file_menu)
root.config(menu=menubar)
\end{lstlisting}

\section{Exit Question (with Solution)}
\textbf{Question:} Show info message box titled \texttt{"Saved"} with message \texttt{"Done"}.

\textbf{Solution:}
\begin{lstlisting}[language=Python]
from tkinter import messagebox
messagebox.showinfo("Saved", "Done")
\end{lstlisting}

\end{document}

