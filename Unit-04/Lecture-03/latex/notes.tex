\documentclass[11pt]{article}
\usepackage[utf8]{inputenc}
\usepackage[T1]{fontenc}
\usepackage{geometry}
\usepackage{amsmath}
\usepackage{hyperref}
\usepackage{xcolor}
\usepackage{listings}
\geometry{margin=1in}

\lstset{
  basicstyle=\ttfamily\small,
  keywordstyle=\color{blue},
  commentstyle=\color{gray},
  stringstyle=\color{teal},
  showstringspaces=false,
  columns=fullflexible,
  frame=single,
  framerule=0.2pt
}

\title{Python Programming\\Unit 04 -- Lecture 03 Notes\\Event Handling and Input Validation}
\author{Tofik Ali}
\date{\today}

\begin{document}
\maketitle
\tableofcontents

\section{Lecture Overview}
GUIs are event-driven: the program waits for user actions (clicks, typing) and
responds using callback functions. Good GUI programs also validate inputs so that:
\begin{itemize}
  \item the app does not crash,
  \item stored data remains consistent,
  \item and the user gets clear feedback.
\end{itemize}

\section{Core Concepts}

\subsection{Callbacks with \texttt{command=}}
The simplest event handling in Tkinter is attaching a callback to a Button:
\begin{lstlisting}[language=Python]
def submit():
    print("Submitted")

tk.Button(root, text="Submit", command=submit).pack()
\end{lstlisting}

\subsection{Binding Events with \texttt{bind}}
\texttt{bind} attaches a function to a specific event sequence.

\begin{lstlisting}[language=Python]
def on_key_release(event):
    print("Key:", event.keysym)

entry.bind("<KeyRelease>", on_key_release)
\end{lstlisting}

Common event sequences:
\begin{itemize}
  \item \texttt{<Button-1>} left mouse click
  \item \texttt{<Return>} Enter key
  \item \texttt{<KeyRelease>} after a key is released
  \item \texttt{<Control-s>} Ctrl+S shortcut
\end{itemize}

\subsection{Validation: What and When}
\textbf{What to validate:}
\begin{itemize}
  \item required fields (must not be empty),
  \item numeric fields (age must be a number),
  \item ranges (age 0--120),
  \item patterns (email).
\end{itemize}

\textbf{When to validate:}
\begin{itemize}
  \item on submit (simple and reliable),
  \item while typing (better user experience, but more code).
\end{itemize}

\subsection{Email Validation (Simple Regex)}
For beginner apps, a simple pattern is often enough.
\begin{lstlisting}[language=Python]
import re
EMAIL = r"^[A-Za-z0-9._%+-]+@[A-Za-z0-9.-]+\.[A-Za-z]{2,}$"
\end{lstlisting}
This checks format only. Real-world email validation is more complex.

\subsection{User Feedback}
Good feedback:
\begin{itemize}
  \item explains what is wrong,
  \item tells the user how to fix it,
  \item does not expose technical details.
\end{itemize}

\section{Demo Walkthrough}
\textbf{File:} \texttt{demo/event\_validation\_demo.py}

What to observe:
\begin{itemize}
  \item Button callback used for submission.
  \item Live validation using \texttt{bind} on input fields.
  \item Message boxes for clear feedback.
  \item Data is ``accepted'' only when all fields are valid.
\end{itemize}

\section{Interactive Checkpoints (with Solutions)}

\subsection*{Checkpoint 1 Solution}
\textbf{Question:} When use \texttt{command=} vs \texttt{bind()}?

\textbf{Answer:}
\begin{itemize}
  \item \texttt{command=} is best for button clicks (no event object needed).
  \item \texttt{bind()} is best for key/mouse events and shortcuts (event object needed).
\end{itemize}

\subsection*{Checkpoint 2 Solution}
\textbf{Question:} Why is validation important?

\textbf{Answer:} It prevents wrong data, avoids crashes, and improves user experience.

\section{Practice Exercises (with Solutions)}

\subsection*{Exercise 1: Age Validation Rule}
\textbf{Task:} Accept age only if it is an integer between 0 and 120.

\textbf{Solution:}
\begin{lstlisting}[language=Python]
age_text = age_var.get().strip()
if not age_text.isdigit():
    error = "Age must be an integer."
else:
    age = int(age_text)
    if not (0 <= age <= 120):
        error = "Age must be between 0 and 120."
\end{lstlisting}

\subsection*{Exercise 2: Keyboard Shortcut}
\textbf{Task:} Bind Ctrl+L to clear the form.

\textbf{Solution (idea):}
\begin{lstlisting}[language=Python]
def clear(event=None):
    name_var.set("")
    email_var.set("")
    age_var.set("")

root.bind("<Control-l>", clear)
\end{lstlisting}

\section{Exit Question (with Solution)}
\textbf{Question:} Write one validation rule for Age input.

\textbf{Example answer:} Age must be a non-negative integer (or in a range 0--120).

\end{document}

