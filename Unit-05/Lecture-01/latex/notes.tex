\documentclass[11pt]{article}
\usepackage[utf8]{inputenc}
\usepackage[T1]{fontenc}
\usepackage{geometry}
\usepackage{amsmath}
\usepackage{hyperref}
\usepackage{xcolor}
\usepackage{listings}
\geometry{margin=1in}

\lstset{
  basicstyle=\ttfamily\small,
  keywordstyle=\color{blue},
  commentstyle=\color{gray},
  stringstyle=\color{teal},
  showstringspaces=false,
  columns=fullflexible,
  frame=single,
  framerule=0.2pt
}

\title{Python Programming\\Unit 05 -- Lecture 01 Notes\\OOP Basics, Classes, Objects, Constructors}
\author{Tofik Ali}
\date{\today}

\begin{document}
\maketitle
\tableofcontents

\section{Lecture Overview}
Object-Oriented Programming (OOP) helps you design programs by modeling entities
as \textbf{objects} that combine:
\begin{itemize}
  \item \textbf{data} (attributes),
  \item \textbf{behavior} (methods).
\end{itemize}
This lecture introduces classes, objects, constructors, and class vs instance variables.

\section{Core Concepts}

\subsection{Class and Object}
\textbf{Class:} blueprint (definition).\\
\textbf{Object:} instance created from the class.

\subsection{Defining a Class}
\begin{lstlisting}[language=Python]
class Student:
    def __init__(self, name, sapid):
        self.name = name
        self.sapid = sapid

    def display(self):
        print(self.name, self.sapid)
\end{lstlisting}

\subsection{What is \texttt{self}?}
\texttt{self} refers to the current object.
When you call \texttt{s.display()}, Python internally calls \texttt{Student.display(s)}.

\subsection{Constructor: \texttt{\_\_init\_\_}}
\texttt{\_\_init\_\_} initializes a new object.
\begin{lstlisting}[language=Python]
s1 = Student("Asha", "5001")
\end{lstlisting}

\subsection{Special Methods (Dunder Methods)}
Special methods start and end with double underscores.
Common examples:
\begin{itemize}
  \item \texttt{\_\_init\_\_}: constructor
  \item \texttt{\_\_str\_\_}: string representation (used by \texttt{print})
  \item \texttt{\_\_repr\_\_}: developer representation
\end{itemize}

\begin{lstlisting}[language=Python]
class Student:
    def __str__(self):
        return f"{self.name} ({self.sapid})"
\end{lstlisting}

\subsection{Class Variables vs Instance Variables}
Class variable is shared among all objects:
\begin{lstlisting}[language=Python]
class Student:
    college = "UPES"
\end{lstlisting}
Instance variables belong to each object:
\begin{lstlisting}[language=Python]
def __init__(self, name):
    self.name = name
\end{lstlisting}

\section{Demo Walkthrough}
\textbf{File:} \texttt{demo/student\_class\_demo.py}

Observe:
\begin{itemize}
  \item multiple objects share the class variable \texttt{college},
  \item each object has its own \texttt{name}, \texttt{sapid}, and marks list,
  \item \texttt{\_\_str\_\_} makes printing objects easy.
\end{itemize}

\section{Interactive Checkpoints (with Solutions)}

\subsection*{Checkpoint 1 Solution}
\textbf{Question:} purpose of \texttt{self}?

\textbf{Answer:} It refers to the current object and allows methods to access the object's data.

\subsection*{Checkpoint 2 Solution}
\textbf{Question:} how many copies of a class variable exist?

\textbf{Answer:} One shared copy in the class (objects reference it).

\section{Practice Exercises (with Solutions)}

\subsection*{Exercise 1: Rectangle Class}
\textbf{Task:} Create a \texttt{Rectangle} class with \texttt{length}, \texttt{width} and a method \texttt{area()}.

\textbf{Solution:}
\begin{lstlisting}[language=Python]
class Rectangle:
    def __init__(self, length, width):
        self.length = length
        self.width = width

    def area(self):
        return self.length * self.width

r = Rectangle(4, 3)
print(r.area())
\end{lstlisting}

\subsection*{Exercise 2: Add \texttt{\_\_str\_\_}}
\textbf{Task:} Add \texttt{\_\_str\_\_} to print \texttt{"Rectangle(4,3)"}.

\textbf{Solution:}
\begin{lstlisting}[language=Python]
def __str__(self):
    return f"Rectangle({self.length},{self.width})"
\end{lstlisting}

\section{Exit Question (with Solution)}
\textbf{Question:} special method used as constructor?

\textbf{Answer:} \texttt{\_\_init\_\_}

\end{document}

