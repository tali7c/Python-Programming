\documentclass{beamer}

\usetheme{Berlin}
\usecolortheme{Orchid}
\useoutertheme{miniframes}
\setbeamertemplate{navigation symbols}{}

\usepackage{amsmath}
\usepackage{booktabs}
\usepackage{graphicx}
\usepackage{xcolor}
\usepackage{listings}
\usepackage{hyperref}

\lstset{
  basicstyle=\ttfamily\small,
  keywordstyle=\color{blue},
  commentstyle=\color{gray},
  stringstyle=\color{teal},
  showstringspaces=false
}

\title[Python Programming]{Python Programming}
\subtitle{Unit 05 -- Lecture 01: OOP Basics, Classes, Objects, Constructors}
\author{Tofik Ali}
\institute{School of Computer Science, UPES Dehradun}
\date{\today}

\begin{document}

\begin{frame}[fragile]
  \titlepage
  \vspace{0.5em}
  \begin{center}
  \footnotesize Repository: \texttt{https://github.com/tali7c/Python-Programming}
  \end{center}
\end{frame}

\begin{frame}[fragile]{Quick Links}
  \centering
  \hyperlink{sec:core}{\beamerbutton{Core Concepts}}\hspace{1em}
  \hyperlink{sec:demo}{\beamerbutton{Demo}}\hspace{1em}
  \hyperlink{sec:interactive}{\beamerbutton{Interactive}}\hspace{1em}
  \hyperlink{sec:summary}{\beamerbutton{Summary}}
\end{frame}

\begin{frame}[fragile]{Agenda}
  \tableofcontents
\end{frame}

\section{Core Concepts}
\label{sec:core}

\begin{frame}[fragile]{Learning Outcomes}
  \begin{itemize}[<+->]
    \item Explain OOP concepts (class, object, encapsulation)
    \item Define classes and create objects in Python
    \item Use constructors (\texttt{\_\_init\_\_}) and instance methods
    \item Use special methods like \texttt{\_\_str\_\_}
    \item Distinguish class variables and instance variables
  \end{itemize}
\end{frame}

\begin{frame}[fragile]{Why OOP?}
  \begin{itemize}[<+->]
    \item Models real-world entities (Student, Account, Car)
    \item Groups data + behavior together
    \item Improves reusability and maintainability
    \item Supports inheritance and polymorphism
  \end{itemize}
\end{frame}

\begin{frame}[fragile]{Class and Object}
  \begin{itemize}[<+->]
    \item \textbf{Class}: blueprint (defines attributes and methods)
    \item \textbf{Object}: instance created from a class
  \end{itemize}
\end{frame}

\begin{frame}[fragile]{Defining a Class}
  \begin{lstlisting}[language=Python]
class Student:
    def __init__(self, name, sapid):
        self.name = name
        self.sapid = sapid

    def display(self):
        print(self.name, self.sapid)
  \end{lstlisting}
\end{frame}

\begin{frame}[fragile]{Constructor: \texttt{\_\_init\_\_}}
  \begin{itemize}[<+->]
    \item Called automatically when you create an object
    \item Initializes instance variables
  \end{itemize}
  \vspace{0.4em}
  \begin{lstlisting}[language=Python]
s1 = Student("Asha", "5001")
  \end{lstlisting}
\end{frame}

\begin{frame}[fragile]{Special Methods (\texttt{\_\_str\_\_})}
  \begin{itemize}[<+->]
    \item \texttt{\_\_str\_\_} controls how an object prints
  \end{itemize}
  \vspace{0.4em}
  \begin{lstlisting}[language=Python]
class Student:
    def __str__(self):
        return f"{self.name} ({self.sapid})"
  \end{lstlisting}
\end{frame}

\begin{frame}[fragile]{Class vs Instance Variables}
  \begin{itemize}[<+->]
    \item Class variable: shared by all objects
    \item Instance variable: unique per object (\texttt{self.x})
  \end{itemize}
  \vspace{0.4em}
  \begin{lstlisting}[language=Python]
class Student:
    college = "UPES"  # class variable

    def __init__(self, name):
        self.name = name  # instance variable
  \end{lstlisting}
\end{frame}

\section{Demo}
\label{sec:demo}

\begin{frame}[fragile]{Demo: Student Class}
  \begin{itemize}[<+->]
    \item File: \texttt{demo/student\_class\_demo.py}
    \item Creates multiple objects and shows:
      \begin{itemize}
        \item constructor usage
        \item instance methods
        \item class variable behavior
      \end{itemize}
  \end{itemize}
\end{frame}

\section{Interactive}
\label{sec:interactive}

\begin{frame}[fragile]{Checkpoint 1}
  \textbf{Question:} What is the purpose of \texttt{self} in Python methods?
\end{frame}

\begin{frame}[fragile]{Checkpoint 2}
  \textbf{Question:} If \texttt{college} is a class variable, how many copies exist for 100 students?
\end{frame}

\begin{frame}[fragile]{Think-Pair-Share}
  Discuss:
  \begin{itemize}
    \item What attributes and methods should a \texttt{BankAccount} class have?
  \end{itemize}
\end{frame}

\section{Summary}
\label{sec:summary}

\begin{frame}[fragile]{Key Takeaways}
  \begin{itemize}[<+->]
    \item Classes model data + behavior
    \item \texttt{\_\_init\_\_} initializes objects
    \item Special methods like \texttt{\_\_str\_\_} improve usability
    \item Class variables are shared; instance variables are per object
  \end{itemize}
\end{frame}

\begin{frame}[fragile]{Exit Question}
  Write the name of the special method used as a constructor in Python.
\end{frame}

\end{document}

