\documentclass[11pt]{article}
\usepackage[utf8]{inputenc}
\usepackage[T1]{fontenc}
\usepackage{geometry}
\usepackage{amsmath}
\usepackage{hyperref}
\usepackage{xcolor}
\usepackage{listings}
\geometry{margin=1in}

\lstset{
  basicstyle=\ttfamily\small,
  keywordstyle=\color{blue},
  commentstyle=\color{gray},
  stringstyle=\color{teal},
  showstringspaces=false,
  columns=fullflexible,
  frame=single,
  framerule=0.2pt
}

\title{Python Programming\\Unit 05 -- Lecture 03 Notes\\Inheritance and Types of Inheritance}
\author{Tofik Ali}
\date{\today}

\begin{document}
\maketitle
\tableofcontents

\section{Lecture Overview}
Inheritance is a key OOP feature that allows one class (child) to reuse and extend
another class (parent). It reduces duplication and models ``is-a'' relationships.

\section{Core Concepts}

\subsection{Basic Inheritance}
\begin{lstlisting}[language=Python]
class Person:
    def __init__(self, name):
        self.name = name

class Student(Person):
    def __init__(self, name, sapid):
        super().__init__(name)
        self.sapid = sapid
\end{lstlisting}

\subsection{Types of Inheritance}
\begin{itemize}
  \item \textbf{Single:} one parent, one child.
  \item \textbf{Multilevel:} child becomes parent for another class.
  \item \textbf{Hierarchical:} one parent has multiple children.
  \item \textbf{Multiple:} a class has more than one parent (use carefully).
\end{itemize}

\subsection{The Role of \texttt{super()}}
\texttt{super()} calls parent class methods without writing the parent class name.
It is especially useful for constructors and for multiple inheritance patterns.

\section{Demo Walkthrough}
\textbf{File:} \texttt{demo/inheritance\_demo.py}

The demo shows:
\begin{itemize}
  \item \texttt{Person} base class,
  \item \texttt{Student} subclass,
  \item \texttt{PlacementStudent} as a multilevel subclass.
\end{itemize}

\section{Interactive Checkpoints (with Solutions)}

\subsection*{Checkpoint 1 Solution}
\textbf{Question:} What does \texttt{super()} do in a constructor?

\textbf{Answer:} It calls the parent class constructor so parent attributes are initialized correctly.

\subsection*{Checkpoint 2 Solution}
\textbf{Question:} Give one real-world example.

\textbf{Answer (examples):} Vehicle $\rightarrow$ Car, Person $\rightarrow$ Student, Shape $\rightarrow$ Circle.

\section{Practice Exercises (with Solutions)}

\subsection*{Exercise 1: Base + Child Class}
\textbf{Task:} Create a base class \texttt{Vehicle} (brand) and subclass \texttt{Car} (brand, seats).

\textbf{Solution:}
\begin{lstlisting}[language=Python]
class Vehicle:
    def __init__(self, brand):
        self.brand = brand

class Car(Vehicle):
    def __init__(self, brand, seats):
        super().__init__(brand)
        self.seats = seats
\end{lstlisting}

\subsection*{Exercise 2: Hierarchical Inheritance}
\textbf{Task:} Create \texttt{Employee} base class and subclasses \texttt{Teacher} and \texttt{Engineer}.

\textbf{Solution (idea):}
\begin{lstlisting}[language=Python]
class Employee:
    def __init__(self, name):
        self.name = name

class Teacher(Employee):
    pass

class Engineer(Employee):
    pass
\end{lstlisting}

\section{Exit Question (with Solution)}
\textbf{Question:} A $\rightarrow$ (B, C) is which type?

\textbf{Answer:} Hierarchical inheritance.

\end{document}

