\documentclass{beamer}

\usetheme{Berlin}
\usecolortheme{Orchid}
\useoutertheme{miniframes}
\setbeamertemplate{navigation symbols}{}

\usepackage{amsmath}
\usepackage{booktabs}
\usepackage{graphicx}
\usepackage{xcolor}
\usepackage{listings}
\usepackage{hyperref}

\lstset{
  basicstyle=\ttfamily\small,
  keywordstyle=\color{blue},
  commentstyle=\color{gray},
  stringstyle=\color{teal},
  showstringspaces=false
}

\title[Python Programming]{Python Programming}
\subtitle{Unit 05 -- Lecture 04: Polymorphism (Overriding and Operator Overloading)}
\author{Tofik Ali}
\institute{School of Computer Science, UPES Dehradun}
\date{\today}

\begin{document}

\begin{frame}[fragile]
  \titlepage
  \vspace{0.5em}
  \begin{center}
  \footnotesize Repository: \texttt{https://github.com/tali7c/Python-Programming}
  \end{center}
\end{frame}

\begin{frame}[fragile]{Quick Links}
  \centering
  \hyperlink{sec:core}{\beamerbutton{Core Concepts}}\hspace{1em}
  \hyperlink{sec:demo}{\beamerbutton{Demo}}\hspace{1em}
  \hyperlink{sec:interactive}{\beamerbutton{Interactive}}\hspace{1em}
  \hyperlink{sec:summary}{\beamerbutton{Summary}}
\end{frame}

\begin{frame}[fragile]{Agenda}
  \tableofcontents
\end{frame}

\section{Core Concepts}
\label{sec:core}

\begin{frame}[fragile]{Learning Outcomes}
  \begin{itemize}[<+->]
    \item Explain polymorphism in OOP
    \item Implement method overriding in subclasses
    \item Implement operator overloading using special methods
    \item Use \texttt{\_\_str\_\_} for readable object printing
  \end{itemize}
\end{frame}

\begin{frame}[fragile]{Polymorphism (Idea)}
  \begin{itemize}[<+->]
    \item ``Same interface, different behavior''
    \item Example: \texttt{area()} behaves differently for Circle and Rectangle
  \end{itemize}
\end{frame}

\begin{frame}[fragile]{Method Overriding}
  \begin{itemize}[<+->]
    \item A child class provides its own implementation of a parent method
    \item Python chooses the method based on the object's actual class
  \end{itemize}
\end{frame}

\begin{frame}[fragile]{Operator Overloading}
  \begin{itemize}[<+->]
    \item Define how operators work for your objects
    \item Examples:
      \begin{itemize}
        \item \texttt{+} uses \texttt{\_\_add\_\_}
        \item \texttt{==} uses \texttt{\_\_eq\_\_}
        \item \texttt{len(obj)} uses \texttt{\_\_len\_\_}
      \end{itemize}
  \end{itemize}
\end{frame}

\begin{frame}[fragile]{Example: Point Addition}
  \begin{lstlisting}[language=Python]
class Point:
    def __init__(self, x, y):
        self.x = x
        self.y = y

    def __add__(self, other):
        return Point(self.x + other.x, self.y + other.y)
  \end{lstlisting}
\end{frame}

\section{Demo}
\label{sec:demo}

\begin{frame}[fragile]{Demo: Overriding + Operator Overloading}
  \begin{itemize}[<+->]
    \item File: \texttt{demo/polymorphism\_operator\_overloading\_demo.py}
    \item Shows:
      \begin{itemize}
        \item polymorphic \texttt{area()} calls
        \item \texttt{Point + Point} using \texttt{\_\_add\_\_}
      \end{itemize}
  \end{itemize}
\end{frame}

\section{Interactive}
\label{sec:interactive}

\begin{frame}[fragile]{Checkpoint 1}
  \textbf{Question:} If a parent and child both define \texttt{describe()}, which one runs for a child object?
\end{frame}

\begin{frame}[fragile]{Checkpoint 2}
  \textbf{Question:} Which special method is used for operator \texttt{+}?
\end{frame}

\begin{frame}[fragile]{Think-Pair-Share}
  Discuss:
  \begin{itemize}
    \item Should every class implement operator overloading? When is it helpful vs confusing?
  \end{itemize}
\end{frame}

\section{Summary}
\label{sec:summary}

\begin{frame}[fragile]{Key Takeaways}
  \begin{itemize}[<+->]
    \item Overriding enables polymorphism (same method name, different behavior)
    \item Operator overloading uses special methods like \texttt{\_\_add\_\_}
    \item Use overloading only when it improves readability and matches meaning
  \end{itemize}
\end{frame}

\begin{frame}[fragile]{Exit Question}
  Name the special method used to control printing of an object using \texttt{print(obj)}.
\end{frame}

\end{document}

