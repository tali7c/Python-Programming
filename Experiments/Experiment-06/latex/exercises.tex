\documentclass[11pt]{article}
\usepackage[T1]{fontenc}
\usepackage[utf8]{inputenc}
\usepackage{geometry}
\usepackage{enumitem}
\usepackage{hyperref}

\geometry{margin=1in}
\setlength{\parindent}{0pt}
\setlength{\parskip}{6pt}

\title{Python Programming (B.Tech CSE - Sem 2)\\Experiment 06 -- Exercise Sheet\\Functions, Recursion, and Lambda}
\author{Tofik Ali}
\date{\today}

\begin{document}
\maketitle

\textbf{Repository:} \texttt{https://github.com/tali7c/Python-Programming}

\textbf{Note:} This document contains only problem statements (no solutions).

\section*{Instructions}
\begin{itemize}[leftmargin=*]
  \item Write a separate Python program for each exercise.
  \item Use functions to structure your logic.
  \item Add a base case for every recursive function.
  \item Use lambda only where it improves clarity (short, one-line functions).
\end{itemize}

\section*{Exercises}

\subsection*{Exercise 01: Max and Min Without Built-ins}
Write a function that finds the maximum and minimum of a list \textbf{without}
using built-in \texttt{max()} or \texttt{min()}.

\textbf{Input:} a list of integers \\
\textbf{Output:} maximum and minimum (handle empty list appropriately)

\subsection*{Exercise 02: Sum of Cubes}
Write a function \texttt{sum\_cubes(n)} that returns the sum of cubes of positive
integers smaller than $n$:
$$1^3 + 2^3 + \cdots + (n-1)^3$$

\textbf{Input:} integer $n$ \\
\textbf{Output:} sum of cubes

\subsection*{Exercise 03: Recursive Print (1 to n)}
Write a recursive function that prints integers from 1 to $n$ in increasing order.

\textbf{Input:} integer $n$ \\
\textbf{Output:} numbers 1 to $n$, one per line

\subsection*{Exercise 04: Recursive Fibonacci Series}
Write a recursive function \texttt{fib(n)} and use it to generate the first $n$
terms of the Fibonacci series.

\textbf{Input:} integer $n$ (number of terms) \\
\textbf{Output:} the Fibonacci series (first $n$ terms)

\subsection*{Exercise 05: Lambda for Volume of a Cone}
Create a lambda function to compute the volume of a cone:
$$V = \frac{1}{3}\pi r^2 h$$

\textbf{Input:} radius $r$ and height $h$ \\
\textbf{Output:} cone volume

\subsection*{Exercise 06: Lambda for (max, min)}
Write a lambda function that takes a list and returns a tuple \texttt{(max, min)}
using built-in functions.

\textbf{Input:} a list of integers \\
\textbf{Output:} a tuple \texttt{(max, min)}

\subsection*{Exercise 07: Function Arguments Demonstration}
Write functions demonstrating:
\begin{itemize}[leftmargin=*]
  \item default argument (\texttt{msg="Hello"} in a greeting function),
  \item variable-length positional arguments (\texttt{*args}),
  \item variable-length keyword arguments (\texttt{**kwargs}).
\end{itemize}

\textbf{Input:} function calls of your choice \\
\textbf{Output:} printed results showing that all cases work

\subsection*{Exercise 08: Check All Dictionary Values Are Same}
Given a dictionary, check whether all values are the same. Use a lambda function
in your implementation.

\textbf{Input:} a dictionary \\
\textbf{Output:} \texttt{True} or \texttt{False}

\subsection*{Exercise 09: Create Dictionary From Two Lists}
Read two lists (keys and values) and create a dictionary by pairing them.

\textbf{Input:} list of keys and list of values \\
\textbf{Output:} constructed dictionary

\end{document}

