\documentclass[11pt]{article}
\usepackage[T1]{fontenc}
\usepackage[utf8]{inputenc}
\usepackage{geometry}
\usepackage{enumitem}
\usepackage{hyperref}

\geometry{margin=1in}
\setlength{\parindent}{0pt}
\setlength{\parskip}{6pt}

\title{Python Programming (B.Tech CSE - Sem 2)\\Experiment 02 -- Exercise Sheet\\Conditional Statements}
\author{Tofik Ali}
\date{\today}

\begin{document}
\maketitle

\textbf{Repository:} \texttt{https://github.com/tali7c/Python-Programming}

\textbf{Note:} This document contains only problem statements (no solutions).

\section*{Instructions}
\begin{itemize}[leftmargin=*]
  \item Write a separate program for each exercise.
  \item Use \texttt{if/elif/else} clearly and validate input.
  \item Print outputs in a clean, user-friendly format.
\end{itemize}

\section*{Exercises}

\subsection*{Exercise 01: Divisible by 3 and 5}
Check whether the given number is divisible by 3 and 5 both.

\subsection*{Exercise 02: Multiple of 5}
Check whether a given number is multiple of five or not.

\subsection*{Exercise 03: Greatest Among Two Numbers}
Find the greatest among the two numbers. If numbers are equal then print
\texttt{"Numbers are equal"}.

\subsection*{Exercise 04: Greatest Among Three Numbers}
Find the greatest among three numbers assuming no two values are the same.

\subsection*{Exercise 05: Quadratic Equation Roots}
Check whether the quadratic equation has real roots or imaginary roots.
Display the roots.

\subsection*{Exercise 06: Leap Year}
Find whether a given year is a leap year or not.

\subsection*{Exercise 07: Next Date}
Write a program which takes any date as input and displays the next date of the calendar.

\textbf{Example:}\\
Input: day=20, month=9, year=2005\\
Output: day=21, month=9, year=2005

\subsection*{Exercise 08: Grade Sheet}
Print the grade sheet of a student for the given range of CGPA.
Scan marks of five subjects and calculate the percentage.

\textbf{Formula:}\\
CGPA = percentage / 10

\textbf{CGPA ranges:}
\begin{itemize}[leftmargin=*]
  \item 0.0 to 3.4 $\rightarrow$ F
  \item 3.5 to 5.0 $\rightarrow$ C+
  \item 5.1 to 6.0 $\rightarrow$ B
  \item 6.1 to 7.0 $\rightarrow$ B+
  \item 7.1 to 8.0 $\rightarrow$ A
  \item 8.1 to 9.0 $\rightarrow$ A+
  \item 9.1 to 10.0 $\rightarrow$ O (Outstanding)
\end{itemize}

\textbf{Sample (format idea):}
\begin{itemize}[leftmargin=*]
  \item Name, Roll Number, SAP ID, Semester, Course
  \item Subject-wise marks (5 subjects)
  \item Percentage, CGPA, Grade
\end{itemize}

\end{document}

