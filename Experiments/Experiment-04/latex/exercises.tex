\documentclass[11pt]{article}
\usepackage[T1]{fontenc}
\usepackage[utf8]{inputenc}
\usepackage{geometry}
\usepackage{enumitem}
\usepackage{hyperref}

\geometry{margin=1in}
\setlength{\parindent}{0pt}
\setlength{\parskip}{6pt}

\title{Python Programming (B.Tech CSE - Sem 2)\\Experiment 04 -- Exercise Sheet\\Strings and Sets}
\author{Tofik Ali}
\date{\today}

\begin{document}
\maketitle

\textbf{Repository:} \texttt{https://github.com/tali7c/Python-Programming}

\textbf{Note:} This document contains only problem statements (no solutions).

\section*{Instructions}
\begin{itemize}[leftmargin=*]
  \item Write a separate program for each exercise.
  \item For string tasks, handle case carefully and print clear output.
  \item For set tasks, use set operations (\texttt{| \& - \string^}) where appropriate.
\end{itemize}

\section*{Exercises}

\subsection*{Exercise 01: Count Capital Letters}
Write a program to count and display the number of capital letters in a given string.

\subsection*{Exercise 02: Count Vowels}
Count total number of vowels in a given string.

\subsection*{Exercise 03: Words in Separate Lines}
Input a sentence and print words in separate lines.

\subsection*{Exercise 04: Substring Occurrence Count}
Enter a string and a substring. Print the number of times that the substring occurs
in the given string. Traversal will take place from left to right.

\textbf{Sample Input:}
\begin{verbatim}
ABCDCDC
CDC
\end{verbatim}
\textbf{Sample Output:}
\begin{verbatim}
2
\end{verbatim}

\subsection*{Exercise 05: Alphabet Frequency (Case-Insensitive)}
Given a string containing both upper and lower case alphabets, count the number of
occurrences of each alphabet (case insensitive) and display the same.

\textbf{Sample Input:}
\begin{verbatim}
ABaBCbGc
\end{verbatim}
\textbf{Sample Output (one per line):}
\begin{verbatim}
2A
3B
2C
1G
\end{verbatim}

\subsection*{Exercise 06: Unique Words Using Sets}
Count number of unique words in a given sentence using sets.

\subsection*{Exercise 07: Two Fruit Sets}
Create two sets \texttt{s1} and \texttt{s2} of \texttt{n} fruits each by taking input from user and find:
\begin{itemize}[leftmargin=*]
  \item Fruits which are in both sets \texttt{s1} and \texttt{s2}
  \item Fruits only in \texttt{s1} but not in \texttt{s2}
  \item Count of all unique fruits from \texttt{s1} and \texttt{s2}
\end{itemize}

\subsection*{Exercise 08: Set Operations (Sample Sets)}
Take two sets and apply various set operations on them:
\begin{verbatim}
S1 = {Red, yellow, orange, blue}
S2 = {violet, blue, purple}
\end{verbatim}
Apply union, intersection, difference, and symmetric difference.

\end{document}

