\documentclass[11pt]{article}
\usepackage[T1]{fontenc}
\usepackage[utf8]{inputenc}
\usepackage{geometry}
\usepackage{enumitem}
\usepackage{hyperref}

\geometry{margin=1in}
\setlength{\parindent}{0pt}
\setlength{\parskip}{6pt}

\title{Python Programming (B.Tech CSE - Sem 2)\\Experiment 03 -- Exercise Sheet\\Loops and Basic Programs}
\author{Tofik Ali}
\date{\today}

\begin{document}
\maketitle

\textbf{Repository:} \texttt{https://github.com/tali7c/Python-Programming}

\textbf{Note:} This document contains only problem statements (no solutions).

\section*{Instructions}
\begin{itemize}[leftmargin=*]
  \item Write a separate Python program for each exercise.
  \item Use loops and conditionals as needed.
  \item Validate input where appropriate (example: negative numbers).
  \item Display output clearly and consistently.
\end{itemize}

\section*{Exercises}

\subsection*{Exercise 01: Factorial of a Number}
Write a Python program to compute the factorial of a non-negative integer $n$
using a loop.

\textbf{Input:} an integer $n$ \\
\textbf{Output:} $n!$ (and a suitable message if $n < 0$)

\subsection*{Exercise 02: Armstrong Number Check}
Write a Python program to check whether an integer $n$ is an Armstrong number.
An Armstrong number is a number equal to the sum of its digits raised to the
power of the number of digits.

\textbf{Input:} an integer $n$ \\
\textbf{Output:} print whether $n$ is Armstrong or not

\subsection*{Exercise 03: Fibonacci Series}
Write a Python program to print the first $n$ terms of the Fibonacci series
using iteration.

\textbf{Input:} an integer $n$ (number of terms) \\
\textbf{Output:} the first $n$ Fibonacci numbers

\subsection*{Exercise 04: Prime Number Check}
Write a Python program to check whether a given integer $n$ is prime.

\textbf{Input:} an integer $n$ \\
\textbf{Output:} print whether $n$ is prime or not

\subsection*{Exercise 05: Palindrome Check}
Write a Python program to check whether an input string/number is a palindrome
(reads the same forward and backward).

\textbf{Input:} a string (or number as input text) \\
\textbf{Output:} print whether it is a palindrome or not

\subsection*{Exercise 06: Sum of Digits}
Write a Python program to compute the sum of digits of an integer.

\textbf{Input:} an integer $n$ \\
\textbf{Output:} sum of digits of $n$

\subsection*{Exercise 07: Numbers Divisible by 5 or 7 (1 to 100)}
Write a Python program to generate and display all integers between 1 and 100
that are divisible by 5 or by 7. Also display the total count.

\textbf{Input:} none \\
\textbf{Output:} list of numbers and the count

\subsection*{Exercise 08: Lowercase to Uppercase}
Write a Python program to convert an input string to uppercase.

\textbf{Input:} a string \\
\textbf{Output:} uppercase version of the string

\subsection*{Exercise 09: Multiplication Table}
Write a Python program to print the multiplication table of a given integer
from 1 to 10.

\textbf{Input:} an integer $n$ \\
\textbf{Output:} lines of the form \texttt{n * i = result} for $i=1..10$

\subsection*{Exercise 10: Pattern Printing}
Write a Python program to print the following pattern for $n=5$:

\begin{verbatim}
123454321
1234 * 321
123  * *  21
12   * * *   1
1    * * * *
\end{verbatim}

\textbf{Input:} none (or optionally take $n$ as input) \\
\textbf{Output:} print the pattern line by line

\subsection*{Exercise 11: Harmonic Series Sum}
Write a Python program to compute the sum of the harmonic series:
$$1 + \frac{1}{2} + \frac{1}{3} + \cdots + \frac{1}{n}$$

\textbf{Input:} an integer $n$ \\
\textbf{Output:} the sum (print up to 4 decimal places)

\end{document}

