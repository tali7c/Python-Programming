\documentclass[11pt]{article}
\usepackage[T1]{fontenc}
\usepackage[utf8]{inputenc}
\usepackage{geometry}
\usepackage{enumitem}
\usepackage{hyperref}

\geometry{margin=1in}
\setlength{\parindent}{0pt}
\setlength{\parskip}{6pt}

\title{Python Programming (B.Tech CSE - Sem 2)\\Experiment 05 -- Exercise Sheet\\Collections and Mini-Apps}
\author{Tofik Ali}
\date{\today}

\begin{document}
\maketitle

\textbf{Repository:} \texttt{https://github.com/tali7c/Python-Programming}

\textbf{Note:} This document contains only problem statements (no solutions).

\section*{Instructions}
\begin{itemize}[leftmargin=*]
  \item Write a separate Python program for each exercise.
  \item Use appropriate data structures (list/tuple/dict) based on the task.
  \item Validate input and handle edge cases (empty input, duplicates).
  \item For menu-driven tasks, keep the menu clear and loop until Exit.
\end{itemize}

\section*{Exercises}

\subsection*{Exercise 01: Count Occurrences (0 to 3)}
Read $n$ integers (each expected in the range 0 to 3). Count occurrences of 0,
1, 2, and 3 and display the counts.

\textbf{Input:} $n$ and $n$ integers \\
\textbf{Output:} counts for 0, 1, 2, and 3

\subsection*{Exercise 02: Tuple Average}
Read $n$ numbers, store them in a tuple, and compute the average.

\textbf{Input:} $n$ and $n$ numbers \\
\textbf{Output:} the tuple and its average

\subsection*{Exercise 03: Runner-Up Score}
Read $n$ scores and print the runner-up score (the second highest distinct score).
If there are fewer than two distinct scores, print a suitable message.

\textbf{Input:} $n$ and $n$ integers \\
\textbf{Output:} runner-up score (or message if not possible)

\subsection*{Exercise 04: Name-City Dictionary and City Counts}
Read details of $n$ persons (name and city) and store them in a dictionary.
Then:
\begin{itemize}[leftmargin=*]
  \item print all names,
  \item print all cities,
  \item print name and city pairs,
  \item compute and print how many persons belong to each city.
\end{itemize}

\textbf{Input:} $n$ pairs of (name, city) \\
\textbf{Output:} names, cities, pairs, and city-wise counts

\subsection*{Exercise 05: Movie Dictionary Queries}
Read details of $n$ movies and store them in a dictionary. For each movie, store:
name, year, director, production cost, and collection. Then:
\begin{itemize}[leftmargin=*]
  \item print all movie details,
  \item list movies released before 2015,
  \item list movies that made a profit (collection $>$ cost),
  \item ask for a director name and list movies by that director (case-insensitive).
\end{itemize}

\textbf{Input:} movie details and a director name to search \\
\textbf{Output:} results of the queries

\subsection*{Exercise 06: Contact Book (Menu-Driven)}
Build a contact book using a dictionary with the following operations:
\begin{itemize}[leftmargin=*]
  \item Add a contact (name, phone),
  \item Search by name,
  \item Update phone number,
  \item Delete a contact,
  \item View all contacts,
  \item Exit.
\end{itemize}

\textbf{Input:} menu choices and contact details \\
\textbf{Output:} results of operations and current contacts

\subsection*{Exercise 07: Todo List Manager (Menu-Driven)}
Build a todo list manager using a list with the following operations:
\begin{itemize}[leftmargin=*]
  \item Add a task,
  \item View all tasks (with numbering),
  \item Remove a task by its number,
  \item Exit.
\end{itemize}

\textbf{Input:} menu choices and task details \\
\textbf{Output:} updated task list after each operation

\end{document}

