\documentclass[11pt]{article}
\usepackage[utf8]{inputenc}
\usepackage[T1]{fontenc}
\usepackage{geometry}
\usepackage{hyperref}
\geometry{margin=1in}

\title{Experiment 01 - Setup Notes: Python Installation and Modes}
\author{Tofik Ali}
\date{\today}

\begin{document}
\maketitle

\section*{Objective}
Install Python and understand the difference between \textbf{interactive mode}
(REPL) and \textbf{scripting mode} (\texttt{.py} files).

\section*{Installation (Windows - Quick Steps)}
\begin{enumerate}
  \item Download Python 3 from the official website (python.org).
  \item During installation, check \textbf{Add Python to PATH}.
  \item Finish installation.
  \item Verify in Command Prompt / PowerShell:
  \begin{verbatim}
python --version
python -c "print('Hello')"
  \end{verbatim}
\end{enumerate}

\section*{Interactive Mode vs Scripting Mode}
\textbf{Interactive mode (REPL):}
\begin{itemize}
  \item Run one statement at a time and see immediate output.
  \item Useful for quick testing and learning.
\end{itemize}

\textbf{Scripting mode:}
\begin{itemize}
  \item Write code in a \texttt{.py} file and run it.
  \item Useful for assignments, experiments, and reusable programs.
\end{itemize}

\section*{IDLE (Optional)}
Python ships with IDLE, which can run both:
\begin{itemize}
  \item Shell (interactive mode)
  \item Editor + Run Module (scripting mode)
\end{itemize}

\end{document}

