\documentclass[11pt]{article}
\usepackage[T1]{fontenc}
\usepackage[utf8]{inputenc}
\usepackage{geometry}
\usepackage{enumitem}
\usepackage{hyperref}

\geometry{margin=1in}
\setlength{\parindent}{0pt}
\setlength{\parskip}{6pt}

\title{Python Programming (B.Tech CSE - Sem 2)\\Experiment 01 -- Exercise Sheet\\Python Installation and Basic Python Statements}
\author{Tofik Ali}
\date{\today}

\begin{document}
\maketitle

\textbf{Repository:} \texttt{https://github.com/tali7c/Python-Programming}

\textbf{Note:} This document contains only problem statements (no solutions).

\section*{Instructions}
\begin{itemize}[leftmargin=*]
  \item Write a separate Python program for each programming exercise.
  \item Use meaningful variable names and clear output formatting.
  \item Validate input where appropriate (example: non-negative values).
\end{itemize}

\section*{Exercises}

\subsection*{Exercise 00: Python Installation and Modes (Non-coding)}
\begin{itemize}[leftmargin=*]
  \item Install Python 3.
  \item Write the steps of installation.
  \item Explain the difference between interactive mode and scripting mode (IDLE/terminal).
\end{itemize}

\subsection*{Exercise 01: Age Type}
Create a variable to store your age and print its type using \texttt{type()}.

\subsection*{Exercise 02: Hello String}
Declare a string variable called \texttt{x} and assign it the value \texttt{"Hello"}.
Print the value of \texttt{x}.

\subsection*{Exercise 03: Print Different Data Types}
Take different data types (int, float, string, boolean, etc.) and print values using
the \texttt{print} function. Also print their types.

\subsection*{Exercise 04: Integer Arithmetic}
Declare \texttt{x=9} and \texttt{y=7}. Perform addition, multiplication, division and
subtraction on these two variables and print the results.

\subsection*{Exercise 05: Hypotenuse (Pythagoras Theorem)}
Write a program to compute the length of the hypotenuse ($c$) of a right triangle
using Pythagoras theorem.

\subsection*{Exercise 06: Simple Interest}
Write a program to find simple interest.

\subsection*{Exercise 07: Triangle Area (Sides Given)}
Write a program to find area of a triangle when lengths of sides are given.
Validate whether the sides form a valid triangle.

\subsection*{Exercise 08: Seconds Conversion}
Convert given seconds into hours, minutes and remaining seconds.

\subsection*{Exercise 09: Swap Without Extra Variable}
Swap two numbers without taking an additional variable.

\subsection*{Exercise 10: Sum of First \texttt{n} Natural Numbers}
Find sum of first \texttt{n} natural numbers.

\subsection*{Exercise 11: Truth Table for Bitwise Operators}
Print truth table for bitwise operators \texttt{\&}, \texttt{|} and \texttt{\string^}
for inputs $a,b \in \{0,1\}$.

\subsection*{Exercise 12: Left Shift and Right Shift}
Find left shift and right shift values of a given number.

\subsection*{Exercise 13: Membership Operator}
Using membership operator, find whether a given number is in sequence
\texttt{(10, 20, 56, 78, 89)}.

\end{document}

