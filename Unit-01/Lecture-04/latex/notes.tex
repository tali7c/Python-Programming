\documentclass[11pt]{article}
\usepackage[utf8]{inputenc}
\usepackage[T1]{fontenc}
\usepackage{geometry}
\usepackage{amsmath}
\usepackage{booktabs}
\usepackage{hyperref}
\usepackage{xcolor}
\usepackage{listings}
\geometry{margin=1in}

\lstset{
  basicstyle=\ttfamily\small,
  keywordstyle=\color{blue},
  commentstyle=\color{gray},
  stringstyle=\color{teal},
  showstringspaces=false,
  columns=fullflexible,
  frame=single,
  framerule=0.2pt
}

\title{Python Programming\\Unit 01 -- Lecture 04 Notes\\Input/Output, Escape Sequences, Operators, Precedence}
\author{Tofik Ali}
\date{\today}

\begin{document}
\maketitle
\tableofcontents

\section{Lecture Overview}
This lecture explains how to:
\begin{itemize}
  \item take input correctly (\texttt{input()} returns a string),
  \item print clean output (escape sequences and \texttt{print} options),
  \item use operators (arithmetic, logical, bitwise, membership, identity),
  \item and avoid mistakes using precedence and parentheses.
\end{itemize}

\section{Core Concepts}

\subsection{Input and Output}
\texttt{input(prompt)} reads a line of text from the user and returns a string.
So conversion is necessary when you expect numbers:
\begin{lstlisting}[language=Python]
a = int(input("Enter a: "))
b = float(input("Enter b: "))
print("a + b =", a + b)
\end{lstlisting}

\textbf{Useful \texttt{print} options:}
\begin{itemize}
  \item \texttt{sep}: separator between printed values
  \item \texttt{end}: what to print at the end (default is newline)
\end{itemize}
\begin{lstlisting}[language=Python]
print("A", "B", "C", sep=" | ")  # A | B | C
print("Hello", end=" ")
print("World")                   # Hello World
\end{lstlisting}

\subsection{Escape Sequences}
Escape sequences start with a backslash \texttt{\textbackslash}:
\begin{itemize}
  \item \texttt{\textbackslash n}: new line
  \item \texttt{\textbackslash t}: tab
  \item \texttt{\textbackslash\textbackslash}: backslash character
  \item \texttt{\textbackslash"}: double quote inside a double-quoted string
\end{itemize}
\begin{lstlisting}[language=Python]
print("Line1\nLine2")
print("A\tB\tC")
print("C:\\\\Users\\\\tofik.ali")
\end{lstlisting}

\subsection{Operator Categories}
\begin{center}
\begin{tabular}{ll}
  \toprule
  Category & Examples \\
  \midrule
  Arithmetic & \texttt{+ - * / \% // **} \\
  Relational & \texttt{< <= > >= == !=} \\
  Logical & \texttt{and or not} \\
  Bitwise & \texttt{\& | \string^ \string~ << >>} \\
  Assignment & \texttt{= += -= *= /=} \\
  Membership & \texttt{in}, \texttt{not in} \\
  Identity & \texttt{is}, \texttt{is not} \\
  \bottomrule
\end{tabular}
\end{center}

\subsection{Precedence and Parentheses}
Precedence decides which operator is evaluated first.
\begin{lstlisting}[language=Python]
print(2 + 3 * 4)     # 14
print((2 + 3) * 4)   # 20
\end{lstlisting}

\textbf{Best practice:} even if you know the precedence, use parentheses when
it improves clarity.

\subsection{Membership vs Identity}
\textbf{Membership} checks if a value exists in a collection:
\begin{lstlisting}[language=Python]
nums = [10, 20, 30]
print(20 in nums)      # True
print(50 not in nums)  # True
\end{lstlisting}

\textbf{Identity} checks whether two variables refer to the same object:
\begin{lstlisting}[language=Python]
a = [1, 2]
b = a
c = [1, 2]
print(a is b)  # True  (same object)
print(a is c)  # False (different object)
\end{lstlisting}

\textbf{Important:} \texttt{==} compares values; \texttt{is} compares identity.

\section{Demo Walkthrough}
\textbf{File:} \texttt{demo/operator\_precedence\_playground.py}

\subsection*{What this demo shows}
\begin{itemize}
  \item precedence examples (with and without parentheses),
  \item a bitwise truth table for \texttt{\&}, \texttt{|}, and \texttt{\string^},
  \item membership and identity checks.
\end{itemize}

\section{Interactive Checkpoints (with Solutions)}

\subsection*{Checkpoint 1 Solution}
\textbf{Question:} Evaluate:
\texttt{2 + 3 * 4} and \texttt{(2 + 3) * 4}.

\textbf{Answer:}
\begin{itemize}
  \item \texttt{2 + 3 * 4 = 14} because \texttt{*} happens before \texttt{+}.
  \item \texttt{(2 + 3) * 4 = 20} because parentheses force addition first.
\end{itemize}

\subsection*{Checkpoint 2 Solution}
\textbf{Question:} difference between \texttt{==} and \texttt{is}?

\textbf{Answer:}
\begin{itemize}
  \item \texttt{==} compares values (equality).
  \item \texttt{is} compares identities (same object or not).
\end{itemize}

\section{Practice Exercises (with Solutions)}

\subsection*{Exercise 1: Safe Division}
\textbf{Task:} Read two numbers and divide them. Print a message if divisor is 0.

\textbf{Solution:}
\begin{lstlisting}[language=Python]
a = float(input("a: "))
b = float(input("b: "))
if b != 0:
    print("a / b =", a / b)
else:
    print("Division by zero is not allowed.")
\end{lstlisting}

\subsection*{Exercise 2: Membership Check}
\textbf{Task:} Check whether a number is in the sequence (10,20,56,78,89).

\textbf{Solution:}
\begin{lstlisting}[language=Python]
seq = (10, 20, 56, 78, 89)
n = int(input("Enter n: "))
print(n in seq)
\end{lstlisting}

\subsection*{Exercise 3: Escape Sequences}
\textbf{Task:} Print the following in two lines:
\texttt{Hello} on first line and \texttt{World} on second line.

\textbf{Solution:}
\begin{lstlisting}[language=Python]
print("Hello\nWorld")
\end{lstlisting}

\section{Exit Question (with Solution)}
\textbf{Question:} give one example each: membership operator, identity operator.

\textbf{Answer (example):}
\begin{itemize}
  \item membership: \texttt{5 in [1,2,3,4,5]}
  \item identity: \texttt{a is b}
\end{itemize}

\end{document}

