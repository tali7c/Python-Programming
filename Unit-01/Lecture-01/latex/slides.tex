\documentclass{beamer}

\usetheme{Berlin}
\usecolortheme{Orchid}
\useoutertheme{miniframes}
\setbeamertemplate{navigation symbols}{}

\usepackage{amsmath}
\usepackage{booktabs}
\usepackage{graphicx}
\usepackage{xcolor}
\usepackage{listings}
\usepackage{hyperref}

\lstset{
  basicstyle=\ttfamily\small,
  keywordstyle=\color{blue},
  commentstyle=\color{gray},
  stringstyle=\color{teal},
  showstringspaces=false
}

\title[Python Programming]{Python Programming}
\subtitle{Unit 01 -- Lecture 01: Introduction and Working with Python}
\author{Tofik Ali}
\institute{School of Computer Science, UPES Dehradun}
\date{\today}

\begin{document}

\begin{frame}[fragile]
  \titlepage
  \vspace{0.5em}
  \begin{center}
  \footnotesize Repository: \texttt{https://github.com/tali7c/Python-Programming}
  \end{center}
\end{frame}

\begin{frame}[fragile]{Quick Links}
  \centering
  \hyperlink{sec:core}{\beamerbutton{Core Concepts}}\hspace{1em}
  \hyperlink{sec:demo}{\beamerbutton{Demo}}\hspace{1em}
  \hyperlink{sec:interactive}{\beamerbutton{Interactive}}\hspace{1em}
  \hyperlink{sec:summary}{\beamerbutton{Summary}}
\end{frame}

\begin{frame}[fragile]{Agenda}
  \tableofcontents
\end{frame}

\section{Overview}

\section{Core Concepts}
\label{sec:core}

\begin{frame}[fragile]{Learning Outcomes}
  \begin{itemize}[<+->]
    \item Explain \textbf{interactive (REPL)} vs \textbf{scripting} mode
    \item Run Python from terminal/IDLE/IDE
    \item Write a first program using \texttt{print()} and \texttt{input()}
    \item Follow the basic Input $\rightarrow$ Process $\rightarrow$ Output pattern
  \end{itemize}
\end{frame}

\begin{frame}[fragile]{Why Python?}
  \begin{itemize}[<+->]
    \item Simple, readable syntax (good for beginners)
    \item Huge ecosystem (web, data, AI, automation)
    \item Cross-platform (Windows, Linux, macOS)
    \item Great for rapid prototyping and scripting
  \end{itemize}
\end{frame}

\begin{frame}[fragile]{Two Ways to Run Python}
  \begin{itemize}[<+->]
    \item \textbf{Interactive mode (REPL)}: run one line at a time
    \item \textbf{Scripting mode}: save code in a \texttt{.py} file and run it
  \end{itemize}
  \vspace{0.4em}
  \begin{lstlisting}[language=Python]
>>> 2 + 3
5
  \end{lstlisting}
\end{frame}

\begin{frame}[fragile]{Interactive Mode (REPL)}
  Best for:
  \begin{itemize}[<+->]
    \item quick calculations and testing small ideas
    \item learning syntax (instant feedback)
    \item checking library functions
  \end{itemize}
  \vspace{0.4em}
  \begin{lstlisting}[language=Python]
>>> name = "UPES"
>>> name.lower()
'upes'
  \end{lstlisting}
\end{frame}

\begin{frame}[fragile]{Scripting Mode (\texttt{.py} files)}
  Best for:
  \begin{itemize}[<+->]
    \item programs you want to \textbf{save, reuse, and share}
    \item assignments, labs, mini-projects
  \end{itemize}
  \vspace{0.4em}
  Typical run command:
  \begin{lstlisting}
python my_program.py
  \end{lstlisting}
\end{frame}

\begin{frame}[fragile]{First Program: Input $\rightarrow$ Process $\rightarrow$ Output}
  \begin{lstlisting}[language=Python]
name = input("Enter your name: ")
print("Hello,", name)
  \end{lstlisting}
  \begin{itemize}[<+->]
    \item \texttt{input()} always returns a \texttt{str}
    \item Use type conversion when you need numbers
  \end{itemize}
\end{frame}

\begin{frame}[fragile]{Type Conversion (Quick Example)}
  \begin{lstlisting}[language=Python]
a = input("Enter a number: ")
b = input("Enter another number: ")
total = int(a) + int(b)
print("Sum =", total)
  \end{lstlisting}
  \begin{itemize}[<+->]
    \item Without \texttt{int()}, \texttt{"2" + "3"} becomes \texttt{"23"}
  \end{itemize}
\end{frame}

\section{Demo}
\label{sec:demo}

\begin{frame}[fragile]{Demo: Mini Calculator}
  \begin{itemize}[<+->]
    \item File: \texttt{demo/mini\_calculator.py}
    \item Reads two numbers and prints +, -, *, /
    \item Shows input conversion and division-by-zero handling
  \end{itemize}
\end{frame}

\begin{frame}[fragile]{Demo Snippet}
  \begin{lstlisting}[language=Python]
a = float(input("Enter a: "))
b = float(input("Enter b: "))
print("a + b =", a + b)
print("a - b =", a - b)
print("a * b =", a * b)
if b != 0:
    print("a / b =", a / b)
else:
    print("Division by zero is not allowed.")
  \end{lstlisting}
\end{frame}

\section{Interactive}
\label{sec:interactive}

\begin{frame}[fragile]{Checkpoint 1}
  \textbf{Question:} When is interactive mode (REPL) better than a script?
  \vspace{0.6em}

  \textbf{Your answer:} write 2 use-cases.
\end{frame}

\begin{frame}[fragile]{Checkpoint 2}
  \textbf{Question:} What is the type of \texttt{input()}?

  \vspace{0.6em}
  \begin{lstlisting}[language=Python]
x = input("Enter something: ")
print(type(x))
  \end{lstlisting}
\end{frame}

\begin{frame}[fragile]{Think-Pair-Share}
  In pairs, discuss:
  \begin{itemize}
    \item Where can Python help you in daily life or engineering?
    \item Give 3 examples (automation / data / web / etc.).
  \end{itemize}
\end{frame}

\section{Summary}
\label{sec:summary}

\begin{frame}[fragile]{Key Takeaways}
  \begin{itemize}[<+->]
    \item REPL is for quick exploration; scripts are for reusable programs
    \item \texttt{input()} returns a string; convert to \texttt{int/float} when needed
    \item Most programs follow Input $\rightarrow$ Process $\rightarrow$ Output
  \end{itemize}
\end{frame}

\begin{frame}[fragile]{Exit Question}
  Write the command to run a Python script named \texttt{task.py} from terminal.
\end{frame}

\end{document}

