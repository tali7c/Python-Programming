\documentclass[11pt]{article}
\usepackage[utf8]{inputenc}
\usepackage[T1]{fontenc}
\usepackage{geometry}
\usepackage{amsmath}
\usepackage{hyperref}
\usepackage{xcolor}
\usepackage{listings}
\geometry{margin=1in}

\lstset{
  basicstyle=\ttfamily\small,
  keywordstyle=\color{blue},
  commentstyle=\color{gray},
  stringstyle=\color{teal},
  showstringspaces=false,
  columns=fullflexible,
  frame=single,
  framerule=0.2pt
}

\title{Python Programming\\Unit 01 -- Lecture 01 Notes\\Introduction and Working with Python}
\author{Tofik Ali}
\date{\today}

\begin{document}
\maketitle
\tableofcontents

\section{Lecture Overview}
This lecture helps you \textbf{start} with Python the right way:
\begin{itemize}
  \item what Python is and why it is popular,
  \item how to run Python in \textbf{interactive mode (REPL)} and \textbf{scripting mode},
  \item and how to write a first program that follows the common pattern:
  \textbf{Input} $\rightarrow$ \textbf{Process} $\rightarrow$ \textbf{Output}.
\end{itemize}

\textbf{Repository (course materials):}
\texttt{https://github.com/tali7c/Python-Programming}

\section{Core Concepts}

\subsection{What is Python? (Quick Picture)}
Python is a general-purpose programming language known for:
\begin{itemize}
  \item \textbf{readable syntax} (you can focus on logic, not symbols),
  \item a \textbf{large ecosystem} of libraries (web, data analysis, AI, automation),
  \item and \textbf{cross-platform} support.
\end{itemize}
In this course, Python is used to build programming fundamentals that later
support subjects like data structures, databases, and data analysis.

\subsection{Two Ways to Run Python}
There are two everyday ways to run Python code:
\begin{enumerate}
  \item \textbf{Interactive mode (REPL):} You type a statement and get the result
  immediately. Good for exploration.
  \item \textbf{Scripting mode:} You save a program in a \texttt{.py} file and
  execute it. Good for assignments, labs, and reusable programs.
\end{enumerate}

\subsubsection{Interactive Mode (REPL)}
Example (Python prompt):
\begin{lstlisting}[language=Python]
>>> 2 + 3
5
>>> "UPES".lower()
'upes'
\end{lstlisting}
Use REPL when you want quick feedback, or when you are learning a new concept.

\subsubsection{Scripting Mode (\texttt{.py} file)}
A script is a plain text file that contains Python statements.
Typical run command in a terminal:
\begin{lstlisting}
python my_program.py
\end{lstlisting}
If your file is \texttt{task.py}, the command is \texttt{python task.py}.

\subsection{The IPO Pattern: Input $\rightarrow$ Process $\rightarrow$ Output}
Most beginner programs follow this pattern:
\begin{itemize}
  \item \textbf{Input:} read values from the user (keyboard).
  \item \textbf{Process:} compute something using variables and operators.
  \item \textbf{Output:} show the result using \texttt{print()}.
\end{itemize}

\begin{lstlisting}[language=Python]
name = input("Enter your name: ")
print("Hello,", name)
\end{lstlisting}

\textbf{Important:} \texttt{input()} always returns a \texttt{str}.
So if you read numbers, you must convert:
\begin{lstlisting}[language=Python]
a = input("Enter a number: ")
b = input("Enter another number: ")
total = int(a) + int(b)
print("Sum =", total)
\end{lstlisting}
If you forget \texttt{int(...)} here, then \texttt{"2" + "3"} becomes \texttt{"23"},
because string concatenation is used.

\section{Demo Walkthrough: Mini Calculator}
\textbf{File:} \texttt{demo/mini\_calculator.py}

\subsection*{Goal}
Read two numbers and print their sum, difference, product, and quotient.
Also handle division by zero safely.

\subsection*{How to run}
From the lecture folder:
\begin{lstlisting}
python demo/mini_calculator.py
\end{lstlisting}

\subsection*{Key idea}
Use \texttt{float(...)} to allow decimal inputs. Use an \texttt{if} condition to
avoid dividing by zero.

\section{Interactive Checkpoints (with Solutions)}

\subsection*{Checkpoint 1 Solution}
\textbf{Question:} When is interactive mode (REPL) better than a script?

\textbf{Answer (examples):}
\begin{itemize}
  \item When you want to quickly test a small expression (e.g., check operator precedence).
  \item When you want to explore a library function without writing a full program.
\end{itemize}

\subsection*{Checkpoint 2 Solution}
\textbf{Question:} What is the type of \texttt{input()}?

\textbf{Answer:} \texttt{input()} returns a \texttt{str}. You can verify using:
\begin{lstlisting}[language=Python]
x = input("Enter something: ")
print(type(x))  # <class 'str'>
\end{lstlisting}

\section{Practice Exercises (with Solutions)}
Try these after the lecture. Solutions are provided so you can self-check.

\subsection*{Exercise 1: Sum of Two Integers}
\textbf{Task:} Read two integers and print their sum.

\textbf{Solution:}
\begin{lstlisting}[language=Python]
a = int(input("Enter a: "))
b = int(input("Enter b: "))
print("Sum =", a + b)
\end{lstlisting}

\subsection*{Exercise 2: Area of a Circle}
\textbf{Task:} Read radius $r$ and print area using $\pi r^2$ (use $\pi=3.14159$).

\textbf{Solution:}
\begin{lstlisting}[language=Python]
pi = 3.14159
r = float(input("Enter radius: "))
area = pi * r * r
print("Area =", area)
\end{lstlisting}

\subsection*{Exercise 3: Print a Clean Output Line}
\textbf{Task:} Print \texttt{Name: <name>, Age: <age>} in one line.

\textbf{Solution:}
\begin{lstlisting}[language=Python]
name = input("Name: ")
age = int(input("Age: "))
print("Name:", name + ",", "Age:", age)
\end{lstlisting}

\section{Exit Question (with Solution)}
\textbf{Question:} Write the command to run \texttt{task.py} from terminal.

\textbf{Solution:} \texttt{python task.py}

\end{document}

