\documentclass{beamer}

\usetheme{Berlin}
\usecolortheme{Orchid}
\useoutertheme{miniframes}
\setbeamertemplate{navigation symbols}{}

\usepackage{amsmath}
\usepackage{booktabs}
\usepackage{graphicx}
\usepackage{xcolor}
\usepackage{listings}
\usepackage{hyperref}

\lstset{
  basicstyle=\ttfamily\small,
  keywordstyle=\color{blue},
  commentstyle=\color{gray},
  stringstyle=\color{teal},
  showstringspaces=false
}

\title[Python Programming]{Python Programming}
\subtitle{Unit 01 -- Lecture 02: Basic Syntax, Comments, Dynamic Typing, Mutability}
\author{Tofik Ali}
\institute{School of Computer Science, UPES Dehradun}
\date{\today}

\begin{document}

\begin{frame}[fragile]
  \titlepage
  \vspace{0.5em}
  \begin{center}
  \footnotesize Repository: \texttt{https://github.com/tali7c/Python-Programming}
  \end{center}
\end{frame}

\begin{frame}[fragile]{Quick Links}
  \centering
  \hyperlink{sec:core}{\beamerbutton{Core Concepts}}\hspace{1em}
  \hyperlink{sec:demo}{\beamerbutton{Demo}}\hspace{1em}
  \hyperlink{sec:interactive}{\beamerbutton{Interactive}}\hspace{1em}
  \hyperlink{sec:summary}{\beamerbutton{Summary}}
\end{frame}

\begin{frame}[fragile]{Agenda}
  \tableofcontents
\end{frame}

\section{Core Concepts}
\label{sec:core}

\begin{frame}[fragile]{Learning Outcomes}
  \begin{itemize}[<+->]
    \item Write valid Python blocks using indentation
    \item Use comments to make code readable
    \item Explain \textbf{dynamic typing} and inspect values with \texttt{type()}
    \item Distinguish \textbf{mutable} and \textbf{immutable} data types
  \end{itemize}
\end{frame}

\begin{frame}[fragile]{Syntax Rule \#1: Indentation}
  \begin{itemize}[<+->]
    \item Python uses indentation to define blocks (no \{ \})
    \item A colon \texttt{:} starts a block (if/for/while/def)
    \item Consistent indentation matters (use 4 spaces)
  \end{itemize}
  \vspace{0.4em}
  \begin{lstlisting}[language=Python]
if 10 > 5:
    print("Yes")
    print("Still inside if")
print("Outside if")
  \end{lstlisting}
\end{frame}

\begin{frame}[fragile]{Comments}
  \begin{itemize}[<+->]
    \item Single-line comment: \texttt{\# ...}
    \item Use comments to explain \textbf{why}, not obvious \textbf{what}
  \end{itemize}
  \vspace{0.4em}
  \begin{lstlisting}[language=Python]
rate = 0.08  # annual interest rate (8%)
  \end{lstlisting}
\end{frame}

\begin{frame}[fragile]{Variables and Assignment}
  \begin{itemize}[<+->]
    \item Variables are created when you assign a value
    \item You can reassign anytime (dynamic typing)
  \end{itemize}
  \vspace{0.4em}
  \begin{lstlisting}[language=Python]
x = 10
x = "ten"      # allowed
print(type(x)) # <class 'str'>
  \end{lstlisting}
\end{frame}

\begin{frame}[fragile]{Dynamic Typing (What it Means)}
  \begin{itemize}[<+->]
    \item The \textbf{value} has a type, not the variable name
    \item \texttt{type(x)} checks the current type of the value stored in \texttt{x}
  \end{itemize}
  \vspace{0.4em}
  \begin{lstlisting}[language=Python]
x = 5
print(type(x))  # int
x = 5.0
print(type(x))  # float
  \end{lstlisting}
\end{frame}

\begin{frame}[fragile]{Mutable vs Immutable}
  \small
  \begin{tabular}{ll}
    \toprule
    Type & Mutability \\
    \midrule
    \texttt{int}, \texttt{float}, \texttt{bool} & immutable \\
    \texttt{str}, \texttt{tuple} & immutable \\
    \texttt{list}, \texttt{dict}, \texttt{set} & mutable \\
    \bottomrule
  \end{tabular}
  \normalsize
  \vspace{0.6em}
  \begin{itemize}[<+->]
    \item Immutable: cannot change the object in place
    \item Mutable: can change contents without creating a new object
  \end{itemize}
\end{frame}

\begin{frame}[fragile]{Example: Immutable String}
  \begin{lstlisting}[language=Python]
s = "python"
s2 = s.upper()
print(s)   # python (original unchanged)
print(s2)  # PYTHON
  \end{lstlisting}
\end{frame}

\begin{frame}[fragile]{Example: Mutable List}
  \begin{lstlisting}[language=Python]
a = [1, 2, 3]
a.append(99)
print(a)  # [1, 2, 3, 99]
  \end{lstlisting}
\end{frame}

\begin{frame}[fragile]{Pitfall: Aliasing (Same List Object)}
  \begin{lstlisting}[language=Python]
a = [1, 2, 3]
b = a          # b points to the same list
b.append(10)
print(a)       # [1, 2, 3, 10]
  \end{lstlisting}
  \begin{itemize}[<+->]
    \item Fix: copy the list using \texttt{a.copy()} or \texttt{a[:]}
  \end{itemize}
\end{frame}

\section{Demo}
\label{sec:demo}

\begin{frame}[fragile]{Demo: Dynamic Typing and Mutability}
  \begin{itemize}[<+->]
    \item File: \texttt{demo/dynamic\_typing\_and\_mutability.py}
    \item Shows:
      \begin{itemize}
        \item reassignment changes type of a variable
        \item \texttt{id()} intuition for objects
        \item aliasing and safe copies
      \end{itemize}
  \end{itemize}
\end{frame}

\section{Interactive}
\label{sec:interactive}

\begin{frame}[fragile]{Checkpoint 1}
  \textbf{Question:} Identify the error and fix it.
  \vspace{0.6em}
  \begin{lstlisting}[language=Python]
if 5 > 2
    print("OK")
  \end{lstlisting}
\end{frame}

\begin{frame}[fragile]{Checkpoint 2}
  \textbf{Question:} Predict the output.
  \vspace{0.6em}
  \begin{lstlisting}[language=Python]
a = [10, 20]
b = a
b.append(30)
print(a)
  \end{lstlisting}
\end{frame}

\begin{frame}[fragile]{Think-Pair-Share}
  Discuss with a partner:
  \begin{itemize}
    \item Why might immutable objects be safer in programs?
    \item Give one example where mutation can cause a bug.
  \end{itemize}
\end{frame}

\section{Summary}
\label{sec:summary}

\begin{frame}[fragile]{Key Takeaways}
  \begin{itemize}[<+->]
    \item Indentation defines blocks; \texttt{:} starts a block
    \item Python is dynamically typed (types belong to values)
    \item Mutable types can change in place; immutable types cannot
    \item Aliasing can surprise you; use copies when needed
  \end{itemize}
\end{frame}

\begin{frame}[fragile]{Exit Question}
  Name one mutable type and one immutable type in Python.
\end{frame}

\end{document}

