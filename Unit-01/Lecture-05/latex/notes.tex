\documentclass[11pt]{article}
\usepackage[utf8]{inputenc}
\usepackage[T1]{fontenc}
\usepackage{geometry}
\usepackage{amsmath}
\usepackage{hyperref}
\usepackage{xcolor}
\usepackage{listings}
\geometry{margin=1in}

\lstset{
  basicstyle=\ttfamily\small,
  keywordstyle=\color{blue},
  commentstyle=\color{gray},
  stringstyle=\color{teal},
  showstringspaces=false,
  columns=fullflexible,
  frame=single,
  framerule=0.2pt
}

\title{Python Programming\\Unit 01 -- Lecture 05 Notes\\Decision Making and Looping Structures}
\author{Tofik Ali}
\date{\today}

\begin{document}
\maketitle
\tableofcontents

\section{Lecture Overview}
Programs make decisions and repeat work. This lecture covers:
\begin{itemize}
  \item decision statements: \texttt{if/elif/else}, nested \texttt{if},
  \item \texttt{match-case} (Python 3.10+) for clean multi-way branching,
  \item looping structures: \texttt{for} and \texttt{while},
  \item loop control: \texttt{break}, \texttt{continue}, \texttt{pass}, and loop \texttt{else}.
\end{itemize}

\section{Core Concepts}

\subsection{Decision Making with \texttt{if/elif/else}}
Basic pattern:
\begin{lstlisting}[language=Python]
marks = int(input("Marks: "))
if marks >= 90:
    print("A+")
elif marks >= 75:
    print("A")
else:
    print("Needs improvement")
\end{lstlisting}

\textbf{Tips:}
\begin{itemize}
  \item Conditions are evaluated top to bottom.
  \item Use \texttt{elif} to avoid deeply nested code.
  \item Always think about boundary conditions (e.g., what about 89, 90?).
\end{itemize}

\subsection{Nested \texttt{if}}
Nested \texttt{if} is valid, but can reduce readability if used too much.
\begin{lstlisting}[language=Python]
age = int(input("Age: "))
if age >= 18:
    if age >= 60:
        print("Senior")
    else:
        print("Adult")
else:
    print("Minor")
\end{lstlisting}

\subsection{\texttt{match-case} (Python 3.10+)}
\texttt{match-case} is useful for menu options and multi-way choices.
\begin{lstlisting}[language=Python]
option = input("Choose A/B/C: ").strip().upper()
match option:
    case "A":
        print("You selected A")
    case "B":
        print("You selected B")
    case "C":
        print("You selected C")
    case _:
        print("Invalid option")
\end{lstlisting}

\subsection{Loops}

\subsubsection{The \texttt{for} loop (with \texttt{range})}
Use \texttt{for} when you know the count or you are iterating over items.
\begin{lstlisting}[language=Python]
for i in range(1, 6):
    print(i)
\end{lstlisting}

\texttt{range(start, stop, step)} examples:
\begin{lstlisting}[language=Python]
range(5)        # 0..4
range(1, 6)     # 1..5
range(10, 0, -2)  # 10,8,6,4,2
\end{lstlisting}

\subsubsection{The \texttt{while} loop}
Use \texttt{while} when repetition depends on a condition.
\begin{lstlisting}[language=Python]
count = 3
while count > 0:
    print(count)
    count -= 1
\end{lstlisting}

\textbf{Warning:} if the condition never becomes false, you get an infinite loop.

\subsection{Loop Control Statements}
\begin{itemize}
  \item \texttt{break}: exits the loop immediately.
  \item \texttt{continue}: skips the rest of the current iteration.
  \item \texttt{pass}: does nothing (placeholder).
  \item Loop \texttt{else}: runs only if the loop completes without \texttt{break}.
\end{itemize}

\begin{lstlisting}[language=Python]
for i in range(1, 6):
    if i == 3:
        break
else:
    print("Runs only if there was no break")
\end{lstlisting}

\section{Demo Walkthrough: Menu-Driven Program}
\textbf{File:} \texttt{demo/menu\_driven\_number\_analyzer.py}

\subsection*{Idea}
This demo uses a loop to keep showing a menu until the user chooses to exit.
It uses \texttt{match-case} to handle choices cleanly.

\subsection*{How to run}
\begin{lstlisting}
python demo/menu_driven_number_analyzer.py
\end{lstlisting}

\section{Interactive Checkpoints (with Solutions)}

\subsection*{Checkpoint 1 Solution}
\textbf{Question:} When does the \texttt{else} part of a loop execute?

\textbf{Answer:} The loop \texttt{else} executes only when the loop finishes
normally (no \texttt{break}). If a \texttt{break} happens, \texttt{else} is skipped.

\subsection*{Checkpoint 2 Solution}
\textbf{Question:} difference between \texttt{break} and \texttt{continue}?

\textbf{Answer:}
\begin{itemize}
  \item \texttt{break} ends the loop completely.
  \item \texttt{continue} skips the current iteration and moves to the next one.
\end{itemize}

\section{Practice Exercises (with Solutions)}

\subsection*{Exercise 1: Even or Odd}
\textbf{Task:} Input a number and print whether it is even or odd.

\textbf{Solution:}
\begin{lstlisting}[language=Python]
n = int(input("Enter n: "))
if n % 2 == 0:
    print("Even")
else:
    print("Odd")
\end{lstlisting}

\subsection*{Exercise 2: Sum 1..n}
\textbf{Task:} Input \texttt{n} and compute sum of first \texttt{n} natural numbers using a loop.

\textbf{Solution:}
\begin{lstlisting}[language=Python]
n = int(input("Enter n: "))
total = 0
for i in range(1, n + 1):
    total += i
print("Sum =", total)
\end{lstlisting}

\subsection*{Exercise 3: Skip Multiples of 3}
\textbf{Task:} Print numbers 1..10 but skip multiples of 3.

\textbf{Solution:}
\begin{lstlisting}[language=Python]
for i in range(1, 11):
    if i % 3 == 0:
        continue
    print(i)
\end{lstlisting}

\section{Exit Question (with Solution)}
\textbf{Question:} Write a loop that prints 1..10 but skips multiples of 3.

\textbf{Solution:} see Exercise 3 above.

\end{document}

