\documentclass{beamer}

\usetheme{Berlin}
\usecolortheme{Orchid}
\useoutertheme{miniframes}
\setbeamertemplate{navigation symbols}{}

\usepackage{amsmath}
\usepackage{booktabs}
\usepackage{graphicx}
\usepackage{xcolor}
\usepackage{listings}
\usepackage{hyperref}

\lstset{
  basicstyle=\ttfamily\small,
  keywordstyle=\color{blue},
  commentstyle=\color{gray},
  stringstyle=\color{teal},
  showstringspaces=false
}

\title[Python Programming]{Python Programming}
\subtitle{Unit 01 -- Lecture 05: Decision Making and Looping Structures}
\author{Tofik Ali}
\institute{School of Computer Science, UPES Dehradun}
\date{\today}

\begin{document}

\begin{frame}[fragile]
  \titlepage
  \vspace{0.5em}
  \begin{center}
  \footnotesize Repository: \texttt{https://github.com/tali7c/Python-Programming}
  \end{center}
\end{frame}

\begin{frame}[fragile]{Quick Links}
  \centering
  \hyperlink{sec:core}{\beamerbutton{Core Concepts}}\hspace{1em}
  \hyperlink{sec:demo}{\beamerbutton{Demo}}\hspace{1em}
  \hyperlink{sec:interactive}{\beamerbutton{Interactive}}\hspace{1em}
  \hyperlink{sec:summary}{\beamerbutton{Summary}}
\end{frame}

\begin{frame}[fragile]{Agenda}
  \tableofcontents
\end{frame}

\section{Core Concepts}
\label{sec:core}

\begin{frame}[fragile]{Learning Outcomes}
  \begin{itemize}[<+->]
    \item Write decisions using \texttt{if/elif/else} and nested conditions
    \item Use \texttt{match-case} for multi-way branching (Python 3.10+)
    \item Write \texttt{for} and \texttt{while} loops using \texttt{range()}
    \item Control loops using \texttt{break}, \texttt{continue}, \texttt{pass}, and loop \texttt{else}
  \end{itemize}
\end{frame}

\begin{frame}[fragile]{Decision Making: \texttt{if/elif/else}}
  \begin{lstlisting}[language=Python]
marks = int(input("Marks: "))
if marks >= 90:
    print("A+")
elif marks >= 75:
    print("A")
else:
    print("Needs improvement")
  \end{lstlisting}
\end{frame}

\begin{frame}[fragile]{Nested \texttt{if} (Example)}
  \begin{lstlisting}[language=Python]
age = int(input("Age: "))
if age >= 18:
    if age >= 60:
        print("Senior")
    else:
        print("Adult")
else:
    print("Minor")
  \end{lstlisting}
\end{frame}

\begin{frame}[fragile]{\texttt{match-case} (Python 3.10+)}
  \begin{itemize}[<+->]
    \item Useful for clean multi-way choices
    \item Alternative to long \texttt{elif} ladders
  \end{itemize}
  \vspace{0.4em}
  \begin{lstlisting}[language=Python]
day = int(input("Enter day number (1-7): "))
match day:
    case 1:
        print("Monday")
    case 2:
        print("Tuesday")
    case _:
        print("Other day")
  \end{lstlisting}
\end{frame}

\begin{frame}[fragile]{Loops: \texttt{for} with \texttt{range}}
  \begin{itemize}[<+->]
    \item \texttt{range(n)} generates 0..n-1
    \item \texttt{range(start, stop, step)} is flexible
  \end{itemize}
  \vspace{0.4em}
  \begin{lstlisting}[language=Python]
for i in range(1, 6):
    print(i)
  \end{lstlisting}
\end{frame}

\begin{frame}[fragile]{Loops: \texttt{while}}
  \begin{itemize}[<+->]
    \item Use when you do not know the exact number of iterations
    \item Always ensure the loop condition eventually becomes false
  \end{itemize}
  \vspace{0.4em}
  \begin{lstlisting}[language=Python]
n = 5
while n > 0:
    print(n)
    n -= 1
  \end{lstlisting}
\end{frame}

\begin{frame}[fragile]{Loop Control}
  \begin{itemize}[<+->]
    \item \texttt{break}: stop the loop
    \item \texttt{continue}: skip to next iteration
    \item \texttt{pass}: do nothing (placeholder)
    \item \texttt{else} in loops: runs only if loop ends normally (no \texttt{break})
  \end{itemize}
\end{frame}

\begin{frame}[fragile]{\texttt{else} with a \texttt{for} Loop}
  \begin{lstlisting}[language=Python]
for i in range(1, 6):
    if i == 3:
        break
else:
    print("Loop finished without break")
  \end{lstlisting}
  \begin{itemize}[<+->]
    \item Here \texttt{else} does not run because we used \texttt{break}.
  \end{itemize}
\end{frame}

\section{Demo}
\label{sec:demo}

\begin{frame}[fragile]{Demo: Menu-Driven Program}
  \begin{itemize}[<+->]
    \item File: \texttt{demo/menu\_driven\_number\_analyzer.py}
    \item Uses a \texttt{while True} loop + \texttt{match-case}
    \item Demonstrates clean branching + repetition until user exits
  \end{itemize}
\end{frame}

\section{Interactive}
\label{sec:interactive}

\begin{frame}[fragile]{Checkpoint 1}
  \textbf{Question:} When does the \texttt{else} part of a loop execute?
\end{frame}

\begin{frame}[fragile]{Checkpoint 2}
  \textbf{Question:} What is the difference between \texttt{break} and \texttt{continue}?
\end{frame}

\begin{frame}[fragile]{Think-Pair-Share}
  Choose the better loop and justify:
  \begin{itemize}
    \item Printing 1..100
    \item Repeating until the user enters ``exit''
  \end{itemize}
\end{frame}

\section{Summary}
\label{sec:summary}

\begin{frame}[fragile]{Key Takeaways}
  \begin{itemize}[<+->]
    \item Use \texttt{if/elif/else} for decisions and \texttt{match-case} for clean multi-way choices
    \item Use \texttt{for} when you have a count/range; use \texttt{while} for condition-based repetition
    \item \texttt{break} stops a loop; \texttt{continue} skips to next iteration
    \item Loop \texttt{else} runs only when no \texttt{break} happens
  \end{itemize}
\end{frame}

\begin{frame}[fragile]{Exit Question}
  Write a loop that prints numbers 1..10 but skips multiples of 3.
\end{frame}

\end{document}

