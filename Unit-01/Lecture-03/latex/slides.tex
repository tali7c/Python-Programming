\documentclass{beamer}

\usetheme{Berlin}
\usecolortheme{Orchid}
\useoutertheme{miniframes}
\setbeamertemplate{navigation symbols}{}

\usepackage{amsmath}
\usepackage{booktabs}
\usepackage{graphicx}
\usepackage{xcolor}
\usepackage{listings}
\usepackage{hyperref}

\lstset{
  basicstyle=\ttfamily\small,
  keywordstyle=\color{blue},
  commentstyle=\color{gray},
  stringstyle=\color{teal},
  showstringspaces=false
}

\title[Python Programming]{Python Programming}
\subtitle{Unit 01 -- Lecture 03: Tokens, Naming, Strings, and Numeric Types}
\author{Tofik Ali}
\institute{School of Computer Science, UPES Dehradun}
\date{\today}

\begin{document}

\begin{frame}[fragile]
  \titlepage
  \vspace{0.5em}
  \begin{center}
  \footnotesize Repository: \texttt{https://github.com/tali7c/Python-Programming}
  \end{center}
\end{frame}

\begin{frame}[fragile]{Quick Links}
  \centering
  \hyperlink{sec:core}{\beamerbutton{Core Concepts}}\hspace{1em}
  \hyperlink{sec:demo}{\beamerbutton{Demo}}\hspace{1em}
  \hyperlink{sec:interactive}{\beamerbutton{Interactive}}\hspace{1em}
  \hyperlink{sec:summary}{\beamerbutton{Summary}}
\end{frame}

\begin{frame}[fragile]{Agenda}
  \tableofcontents
\end{frame}

\section{Core Concepts}
\label{sec:core}

\begin{frame}[fragile]{Learning Outcomes}
  \begin{itemize}[<+->]
    \item Identify Python tokens (keywords, identifiers, literals, operators)
    \item Follow good naming conventions (snake\_case)
    \item Use common string methods and operators
    \item Format output using \texttt{format()} / f-strings
    \item Work with numeric types (\texttt{int}, \texttt{float}, \texttt{complex})
  \end{itemize}
\end{frame}

\begin{frame}[fragile]{Python Tokens (Big Picture)}
  \begin{itemize}[<+->]
    \item \textbf{Keywords}: reserved words (e.g., \texttt{if}, \texttt{for}, \texttt{def})
    \item \textbf{Identifiers}: names you create (variables, functions)
    \item \textbf{Literals}: fixed values (\texttt{10}, \texttt{3.14}, \texttt{"hi"})
    \item \textbf{Operators}: \texttt{+ - * / == <} etc.
    \item \textbf{Delimiters}: \texttt{( ) [ ] \{ \} , :}
  \end{itemize}
\end{frame}

\begin{frame}[fragile]{Identifiers and Naming Conventions}
  \begin{itemize}[<+->]
    \item Use \textbf{snake\_case}: \texttt{total\_marks}, \texttt{student\_name}
    \item Names should be meaningful and consistent
    \item Do not use keywords as names (e.g., \texttt{for = 5} is invalid)
  \end{itemize}
\end{frame}

\begin{frame}[fragile]{String Basics}
  \begin{itemize}[<+->]
    \item Strings are sequences of characters (immutable)
    \item Indexing and slicing works like lists
  \end{itemize}
  \vspace{0.4em}
  \begin{lstlisting}[language=Python]
s = "python"
print(s[0])    # p
print(s[-1])   # n
print(s[1:4])  # yth
  \end{lstlisting}
\end{frame}

\begin{frame}[fragile]{Common String Methods}
  \begin{itemize}[<+->]
    \item \texttt{lower()}, \texttt{upper()}, \texttt{title()}
    \item \texttt{strip()} (remove leading/trailing spaces)
    \item \texttt{split()} (string $\rightarrow$ list of words)
    \item \texttt{replace(old, new)}
    \item \texttt{find(sub)} / \texttt{count(sub)}
  \end{itemize}
\end{frame}

\begin{frame}[fragile]{String Operators}
  \begin{itemize}[<+->]
    \item Concatenation: \texttt{"Py" + "thon"} $\rightarrow$ \texttt{"Python"}
    \item Repetition: \texttt{"py" * 3} $\rightarrow$ \texttt{"pypypy"}
    \item Membership: \texttt{"th" in "python"} $\rightarrow$ \texttt{True}
  \end{itemize}
\end{frame}

\begin{frame}[fragile]{Formatting Output}
  \begin{itemize}[<+->]
    \item \texttt{format()} method:
  \end{itemize}
  \begin{lstlisting}[language=Python]
name = "Rohit"
cgpa = 7.0
print("Name: {}, CGPA: {:.1f}".format(name, cgpa))
  \end{lstlisting}
  \begin{itemize}[<+->]
    \item f-strings (recommended for readability):
  \end{itemize}
  \begin{lstlisting}[language=Python]
print(f"Name: {name}, CGPA: {cgpa:.1f}")
  \end{lstlisting}
\end{frame}

\begin{frame}[fragile]{Numeric Data Types}
  \begin{itemize}[<+->]
    \item \texttt{int}: integers (e.g., \texttt{5}, \texttt{-12})
    \item \texttt{float}: decimals (e.g., \texttt{3.14})
    \item \texttt{complex}: complex numbers (e.g., \texttt{2+3j})
    \item Convert types when needed: \texttt{int("10")}, \texttt{float("2.5")}
  \end{itemize}
\end{frame}

\section{Demo}
\label{sec:demo}

\begin{frame}[fragile]{Demo: Cleaning and Formatting Strings}
  \begin{itemize}[<+->]
    \item File: \texttt{demo/string\_formatting\_demo.py}
    \item Reads a name and marks, cleans extra spaces
    \item Prints a formatted student summary line
  \end{itemize}
\end{frame}

\section{Interactive}
\label{sec:interactive}

\begin{frame}[fragile]{Checkpoint 1}
  \textbf{Question:} What is the output?
  \vspace{0.6em}
  \begin{lstlisting}[language=Python]
print("py" * 3)
  \end{lstlisting}
\end{frame}

\begin{frame}[fragile]{Checkpoint 2}
  \textbf{Question:} What is the difference between \texttt{split()} and \texttt{join()}?
  \vspace{0.6em}
  \begin{lstlisting}[language=Python]
sentence = "Python is fun"
parts = sentence.split()
again = "-".join(parts)
  \end{lstlisting}
\end{frame}

\begin{frame}[fragile]{Think-Pair-Share}
  You have input text with extra spaces:
  \begin{itemize}
    \item \texttt{"  UPES   Dehradun  "}
  \end{itemize}
  Discuss: which methods would you use to clean it?
\end{frame}

\section{Summary}
\label{sec:summary}

\begin{frame}[fragile]{Key Takeaways}
  \begin{itemize}[<+->]
    \item Tokens build programs: keywords, identifiers, literals, operators
    \item Use clean naming: snake\_case, meaningful names
    \item Strings are immutable; methods return new strings
    \item Use \texttt{split/join/strip} for text processing
    \item f-strings and \texttt{format()} produce clean output
  \end{itemize}
\end{frame}

\begin{frame}[fragile]{Exit Question}
  Format \texttt{3.14159} to \textbf{2 decimal places}.
\end{frame}

\end{document}
