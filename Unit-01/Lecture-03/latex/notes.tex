\documentclass[11pt]{article}
\usepackage[utf8]{inputenc}
\usepackage[T1]{fontenc}
\usepackage{geometry}
\usepackage{amsmath}
\usepackage{hyperref}
\usepackage{xcolor}
\usepackage{listings}
\geometry{margin=1in}

\lstset{
  basicstyle=\ttfamily\small,
  keywordstyle=\color{blue},
  commentstyle=\color{gray},
  stringstyle=\color{teal},
  showstringspaces=false,
  columns=fullflexible,
  frame=single,
  framerule=0.2pt
}

\title{Python Programming\\Unit 01 -- Lecture 03 Notes\\Tokens, Naming, Strings, and Numeric Types}
\author{Tofik Ali}
\date{\today}

\begin{document}
\maketitle
\tableofcontents

\section{Lecture Overview}
This lecture focuses on the \textbf{building blocks} of Python code:
\begin{itemize}
  \item tokens (keywords, identifiers, literals, operators),
  \item naming conventions for readable programs,
  \item strings (methods, operators, formatting),
  \item and basic numeric data types (\texttt{int}, \texttt{float}, \texttt{complex}).
\end{itemize}

\section{Core Concepts}

\subsection{Python Tokens}
A \textbf{token} is a small unit of a program. Common token categories:
\begin{itemize}
  \item \textbf{Keywords:} reserved words such as \texttt{if}, \texttt{for}, \texttt{while}, \texttt{def}.
  \item \textbf{Identifiers:} names you create (variables, functions, classes).
  \item \textbf{Literals:} fixed values like \texttt{10}, \texttt{3.14}, \texttt{"hello"}.
  \item \textbf{Operators:} symbols/words that perform operations: \texttt{+}, \texttt{-}, \texttt{*}, \texttt{and}, \texttt{==}.
  \item \textbf{Delimiters:} punctuation used in syntax: \texttt{( ) [ ] \{ \} , :}.
\end{itemize}

\subsection{Identifiers and Naming Conventions}
\textbf{Rules (practical):}
\begin{itemize}
  \item Start with a letter or underscore; remaining characters can include digits.
  \item Identifiers are case-sensitive: \texttt{Name} and \texttt{name} are different.
  \item Do not use keywords as identifiers.
\end{itemize}

\textbf{Conventions (recommended):}
\begin{itemize}
  \item Use \textbf{snake\_case} for variables and functions: \texttt{total\_marks}, \texttt{calculate\_cgpa}.
  \item Choose meaningful names. Prefer \texttt{marks} over \texttt{m}.
\end{itemize}

\subsection{Strings: Values, Indexing, Slicing}
A \textbf{string} is a sequence of characters, written with quotes.

\begin{lstlisting}[language=Python]
s1 = "Python"
s2 = 'UPES'
\end{lstlisting}

Strings support indexing and slicing:
\begin{lstlisting}[language=Python]
s = "python"
s[0]     # 'p'
s[-1]    # 'n'
s[1:4]   # 'yth'
\end{lstlisting}

\textbf{Note:} strings are \textbf{immutable}. Methods usually return a new string.

\subsection{Common String Methods}
Useful beginner methods:
\begin{itemize}
  \item \texttt{lower()}, \texttt{upper()}, \texttt{title()}
  \item \texttt{strip()} removes leading/trailing whitespace
  \item \texttt{split()} splits into a list (by spaces by default)
  \item \texttt{replace(old, new)} replaces occurrences
  \item \texttt{find(sub)}, \texttt{count(sub)}
\end{itemize}

\begin{lstlisting}[language=Python]
text = "  UPES Dehradun  "
clean = text.strip()
print(clean)              # "UPES Dehradun"
print(clean.lower())      # "upes dehradun"
print(clean.split())      # ["UPES", "Dehradun"]
\end{lstlisting}

\subsection{String Operators}
Strings support some operators:
\begin{itemize}
  \item \textbf{Concatenation} using \texttt{+}
  \item \textbf{Repetition} using \texttt{*}
  \item \textbf{Membership} using \texttt{in}
\end{itemize}

\begin{lstlisting}[language=Python]
print("py" * 3)            # pypypy
print("th" in "python")    # True
\end{lstlisting}

\subsection{Formatting Output: \texttt{format()} and f-strings}
Formatting is important for readable output (grade sheets, reports, logs).

\subsubsection*{\texttt{format()} method}
\begin{lstlisting}[language=Python]
name = "Rohit"
cgpa = 7.0
print("Name: {}, CGPA: {:.1f}".format(name, cgpa))
\end{lstlisting}
\texttt{:.1f} means ``one digit after the decimal point''.

\subsubsection*{f-strings}
f-strings are short and readable:
\begin{lstlisting}[language=Python]
print(f"Name: {name}, CGPA: {cgpa:.1f}")
\end{lstlisting}

\subsection{Numeric Data Types}
Main numeric types in this unit:
\begin{itemize}
  \item \texttt{int}: integers (no decimals)
  \item \texttt{float}: real numbers with decimals
  \item \texttt{complex}: numbers like \texttt{2+3j}
\end{itemize}

Type conversions (very common):
\begin{lstlisting}[language=Python]
x = int("10")      # 10
y = float("2.5")   # 2.5
\end{lstlisting}

\section{Demo Walkthrough}
\textbf{File:} \texttt{demo/string\_formatting\_demo.py}

\subsection*{Goal}
Read user text, clean extra spaces, compute a small summary, and print
a nicely formatted line.

\subsection*{Run}
\begin{lstlisting}
python demo/string_formatting_demo.py
\end{lstlisting}

\section{Interactive Checkpoints (with Solutions)}

\subsection*{Checkpoint 1 Solution}
\textbf{Question:} output of \texttt{print("py" * 3)}?

\textbf{Answer:} \texttt{pypypy}

\subsection*{Checkpoint 2 Solution}
\textbf{Question:} difference between \texttt{split()} and \texttt{join()}?

\textbf{Answer:}
\begin{itemize}
  \item \texttt{split()} breaks a string into a list.
  \item \texttt{join()} joins a list of strings into a single string.
\end{itemize}

\begin{lstlisting}[language=Python]
sentence = "Python is fun"
parts = sentence.split()          # ["Python", "is", "fun"]
again = "-".join(parts)           # "Python-is-fun"
\end{lstlisting}

\section{Practice Exercises (with Solutions)}

\subsection*{Exercise 1: Clean a String}
\textbf{Task:} Read a string and print it without leading/trailing spaces.

\textbf{Solution:}
\begin{lstlisting}[language=Python]
s = input("Enter text: ")
print(s.strip())
\end{lstlisting}

\subsection*{Exercise 2: Count Vowels}
\textbf{Task:} Count vowels in an input string (a, e, i, o, u; case-insensitive).

\textbf{Solution:}
\begin{lstlisting}[language=Python]
s = input("Enter text: ").lower()
vowels = "aeiou"
count = 0
for ch in s:
    if ch in vowels:
        count += 1
print("Vowel count =", count)
\end{lstlisting}

\subsection*{Exercise 3: Format to Two Decimals}
\textbf{Task:} Print a floating value with 2 decimal places.

\textbf{Solution (f-string):}
\begin{lstlisting}[language=Python]
x = 3.14159
print(f"{x:.2f}")
\end{lstlisting}

\section{Exit Question (with Solution)}
\textbf{Question:} Format \texttt{3.14159} to 2 decimal places.

\textbf{Answer:} \texttt{3.14}. One way:
\begin{lstlisting}[language=Python]
print(f"{3.14159:.2f}")
\end{lstlisting}

\end{document}

