\documentclass{beamer}

\usetheme{Berlin}
\usecolortheme{Orchid}
\useoutertheme{miniframes}
\setbeamertemplate{navigation symbols}{}

\usepackage{amsmath}
\usepackage{booktabs}
\usepackage{graphicx}
\usepackage{xcolor}
\usepackage{listings}
\usepackage{hyperref}

\lstset{
  basicstyle=\ttfamily\small,
  keywordstyle=\color{blue},
  commentstyle=\color{gray},
  stringstyle=\color{teal},
  showstringspaces=false
}

\title[Python Programming]{Python Programming}
\subtitle{Unit 03 -- Lecture 02: Errors vs Exceptions and Exception Handling}
\author{Tofik Ali}
\institute{School of Computer Science, UPES Dehradun}
\date{\today}

\begin{document}

\begin{frame}[fragile]
  \titlepage
  \vspace{0.5em}
  \begin{center}
  \footnotesize Repository: \texttt{https://github.com/tali7c/Python-Programming}
  \end{center}
\end{frame}

\begin{frame}[fragile]{Quick Links}
  \centering
  \hyperlink{sec:core}{\beamerbutton{Core Concepts}}\hspace{1em}
  \hyperlink{sec:demo}{\beamerbutton{Demo}}\hspace{1em}
  \hyperlink{sec:interactive}{\beamerbutton{Interactive}}\hspace{1em}
  \hyperlink{sec:summary}{\beamerbutton{Summary}}
\end{frame}

\begin{frame}[fragile]{Agenda}
  \tableofcontents
\end{frame}

\section{Core Concepts}
\label{sec:core}

\begin{frame}[fragile]{Learning Outcomes}
  \begin{itemize}[<+->]
    \item Explain the difference between \textbf{errors} and \textbf{exceptions}
    \item Use \texttt{try/except/else/finally} for safe programs
    \item Handle multiple exceptions correctly and specifically
    \item Use \texttt{raise} and \texttt{assert} appropriately
  \end{itemize}
\end{frame}

\begin{frame}[fragile]{Errors vs Exceptions}
  \begin{itemize}[<+->]
    \item \textbf{Syntax errors}: program cannot start (invalid code)
    \item \textbf{Exceptions}: runtime problems (division by zero, bad input, missing file)
    \item A robust program anticipates exceptions and handles them cleanly
  \end{itemize}
\end{frame}

\begin{frame}[fragile]{The Exception Model (Big Picture)}
  \begin{itemize}[<+->]
    \item An exception is \textbf{raised} when something goes wrong
    \item If not handled, it \textbf{propagates} up and stops the program
    \item \texttt{try/except} lets you catch and recover
  \end{itemize}
\end{frame}

\begin{frame}[fragile]{Basic \texttt{try/except}}
  \begin{lstlisting}[language=Python]
try:
    n = int(input("Enter n: "))
    print(10 / n)
except ValueError:
    print("Please enter an integer.")
except ZeroDivisionError:
    print("Division by zero is not allowed.")
  \end{lstlisting}
\end{frame}

\begin{frame}[fragile]{\texttt{else} and \texttt{finally}}
  \begin{itemize}[<+->]
    \item \texttt{else}: runs only if \textbf{no exception} occurred
    \item \texttt{finally}: runs \textbf{always} (cleanup)
  \end{itemize}
  \vspace{0.4em}
  \begin{lstlisting}[language=Python]
try:
    f = open("data.txt", "r")
    data = f.read()
except FileNotFoundError:
    print("File not found")
else:
    print("Length:", len(data))
finally:
    # runs even if exception happens
    try:
        f.close()
    except Exception:
        pass
  \end{lstlisting}
\end{frame}

\begin{frame}[fragile]{Multiple Exceptions}
  \begin{itemize}[<+->]
    \item Catch \textbf{specific} exceptions first
    \item Avoid catching everything unless you re-raise or log
  \end{itemize}
  \vspace{0.4em}
  \begin{lstlisting}[language=Python]
try:
    x = int(input("x: "))
    y = int(input("y: "))
    print(x // y)
except (ValueError, ZeroDivisionError) as e:
    print("Error:", e)
  \end{lstlisting}
\end{frame}

\begin{frame}[fragile]{\texttt{raise} and \texttt{assert}}
  \begin{itemize}[<+->]
    \item \texttt{raise} creates your own exception with a message
    \item \texttt{assert} is mainly for debugging assumptions
  \end{itemize}
  \vspace{0.4em}
  \begin{lstlisting}[language=Python]
age = int(input("Age: "))
if age < 0:
    raise ValueError("Age must be non-negative")

assert age >= 0
  \end{lstlisting}
\end{frame}

\section{Demo}
\label{sec:demo}

\begin{frame}[fragile]{Demo: Safe Input + Safe File Read}
  \begin{itemize}[<+->]
    \item File: \texttt{demo/exception\_handling\_demo.py}
    \item Shows:
      \begin{itemize}
        \item repeated input until valid integer
        \item safe division
        \item safe file reading (missing file handling)
      \end{itemize}
  \end{itemize}
\end{frame}

\section{Interactive}
\label{sec:interactive}

\begin{frame}[fragile]{Checkpoint 1}
  \textbf{Question:} When does the \texttt{else} block of a \texttt{try} statement run?
\end{frame}

\begin{frame}[fragile]{Checkpoint 2}
  \textbf{Question:} Why is it better to catch \texttt{ValueError} than to catch
  \texttt{Exception} for user input conversion?
\end{frame}

\begin{frame}[fragile]{Think-Pair-Share}
  Discuss:
  \begin{itemize}
    \item Where should you place \texttt{try/except} blocks?
    \item Around the entire program OR only around the risky lines?
  \end{itemize}
\end{frame}

\section{Summary}
\label{sec:summary}

\begin{frame}[fragile]{Key Takeaways}
  \begin{itemize}[<+->]
    \item Exceptions are runtime issues; syntax errors stop the program immediately
    \item Use \texttt{try/except} to handle expected failures
    \item \texttt{else} runs only on success; \texttt{finally} runs always
    \item Catch specific exceptions and keep error messages user-friendly
  \end{itemize}
\end{frame}

\begin{frame}[fragile]{Exit Question}
  Write a snippet that reads an integer and prints it.
  If the input is invalid, print \texttt{"Invalid integer"} without crashing.
\end{frame}

\end{document}

