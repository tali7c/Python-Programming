\documentclass[11pt]{article}
\usepackage[utf8]{inputenc}
\usepackage[T1]{fontenc}
\usepackage{geometry}
\usepackage{amsmath}
\usepackage{hyperref}
\usepackage{xcolor}
\usepackage{listings}
\geometry{margin=1in}

\lstset{
  basicstyle=\ttfamily\small,
  keywordstyle=\color{blue},
  commentstyle=\color{gray},
  stringstyle=\color{teal},
  showstringspaces=false,
  columns=fullflexible,
  frame=single,
  framerule=0.2pt
}

\title{Python Programming\\Unit 03 -- Lecture 02 Notes\\Errors vs Exceptions and Exception Handling}
\author{Tofik Ali}
\date{\today}

\begin{document}
\maketitle
\tableofcontents

\section{Lecture Overview}
When a program meets unexpected input or a missing resource (file, database,
network), it should not crash with a confusing traceback for the user. Instead,
it should:
\begin{itemize}
  \item detect the problem,
  \item display a clear message, and
  \item recover or exit gracefully.
\end{itemize}
This lecture introduces exceptions and shows how to handle them using
\texttt{try/except/else/finally}.

\section{Core Concepts}

\subsection{Syntax Errors vs Exceptions}
\textbf{Syntax errors} happen when Python cannot understand your code.
Example: missing colon after \texttt{if}.

\textbf{Exceptions} are runtime problems.
Example: dividing by zero, converting \texttt{"abc"} to \texttt{int}, missing file, etc.

\subsection{What is an Exception?}
An exception is an object that represents an error condition.
When raised, it interrupts normal flow and searches for a matching \texttt{except}.
If none is found, the program terminates and prints a traceback.

\subsection{Basic \texttt{try/except}}
\begin{lstlisting}[language=Python]
try:
    n = int(input("Enter n: "))
    print(10 / n)
except ValueError:
    print("Please enter an integer.")
except ZeroDivisionError:
    print("Division by zero is not allowed.")
\end{lstlisting}

\textbf{Why catch specific exceptions?}
Because different problems need different messages and different recovery steps.

\subsection{\texttt{else} and \texttt{finally}}
\texttt{else} runs only if the \texttt{try} block succeeded (no exception).
\texttt{finally} runs always (cleanup).

\begin{lstlisting}[language=Python]
f = None
try:
    f = open("data.txt", "r", encoding="utf-8")
    data = f.read()
except FileNotFoundError:
    print("File not found.")
else:
    print("Length:", len(data))
finally:
    if f is not None:
        f.close()
\end{lstlisting}

\textbf{Note:} In real code, prefer \texttt{with open(...)} so you avoid manual close.

\subsection{Handling Multiple Exceptions}
Sometimes multiple error types can happen in the same risky code.
Python allows catching a tuple of exceptions:
\begin{lstlisting}[language=Python]
try:
    x = int(input("x: "))
    y = int(input("y: "))
    print(x // y)
except (ValueError, ZeroDivisionError) as e:
    print("Error:", e)
\end{lstlisting}

\subsection{\texttt{raise}: Creating Your Own Errors}
You can raise an exception to signal invalid state:
\begin{lstlisting}[language=Python]
age = int(input("Age: "))
if age < 0:
    raise ValueError("Age must be non-negative")
\end{lstlisting}
This is useful when you want to enforce a rule and stop execution for invalid inputs.

\subsection{\texttt{assert}: Debugging Assumptions}
\texttt{assert condition} is mainly for debugging and internal checks:
\begin{lstlisting}[language=Python]
assert 2 + 2 == 4
\end{lstlisting}
\textbf{Warning:} assertions can be disabled when Python runs with optimization flags
(\texttt{-O}). Do not use \texttt{assert} as a replacement for user input validation.

\section{Demo Walkthrough}
\textbf{File:} \texttt{demo/exception\_handling\_demo.py}

\subsection*{What the demo teaches}
\begin{itemize}
  \item Create a safe input function that keeps asking until the user enters a valid integer.
  \item Use \texttt{try/except} for division and file reading.
  \item Print user-friendly error messages.
\end{itemize}

\section{Interactive Checkpoints (with Solutions)}

\subsection*{Checkpoint 1 Solution}
\textbf{Question:} When does \texttt{else} run in a \texttt{try} statement?

\textbf{Answer:} \texttt{else} runs only if \textbf{no exception} occurs in the \texttt{try} block.

\subsection*{Checkpoint 2 Solution}
\textbf{Question:} Why prefer catching \texttt{ValueError} instead of catching \texttt{Exception}?

\textbf{Answer:}
\begin{itemize}
  \item Catching \texttt{Exception} can hide bugs you did not expect (programming mistakes).
  \item Catching \texttt{ValueError} is more precise and documents what you are handling.
\end{itemize}

\section{Practice Exercises (with Solutions)}

\subsection*{Exercise 1: Safe Integer Input Function}
\textbf{Task:} Write \texttt{read\_int(prompt)} that keeps asking until user enters a valid integer.

\textbf{Solution:}
\begin{lstlisting}[language=Python]
def read_int(prompt: str) -> int:
    while True:
        try:
            return int(input(prompt).strip())
        except ValueError:
            print("Invalid integer. Try again.")
\end{lstlisting}

\subsection*{Exercise 2: Safe Division}
\textbf{Task:} Read two integers and print \texttt{a/b}. Handle division by zero.

\textbf{Solution:}
\begin{lstlisting}[language=Python]
try:
    a = int(input("a: "))
    b = int(input("b: "))
    print("a / b =", a / b)
except ZeroDivisionError:
    print("Division by zero is not allowed.")
except ValueError:
    print("Invalid integer input.")
\end{lstlisting}

\subsection*{Exercise 3: Safe File Read}
\textbf{Task:} Read a filename and print its content. If file does not exist, print a message.

\textbf{Solution:}
\begin{lstlisting}[language=Python]
path = input("File path: ").strip()
try:
    with open(path, "r", encoding="utf-8") as f:
        print(f.read())
except FileNotFoundError:
    print("File not found.")
\end{lstlisting}

\section{Exit Question (with Solution)}
\textbf{Question:} Read an integer and print it. If invalid, print \texttt{"Invalid integer"}.

\textbf{Solution:}
\begin{lstlisting}[language=Python]
try:
    n = int(input("Enter n: ").strip())
    print("n =", n)
except ValueError:
    print("Invalid integer")
\end{lstlisting}

\end{document}

