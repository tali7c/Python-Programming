\documentclass[11pt]{article}
\usepackage[utf8]{inputenc}
\usepackage[T1]{fontenc}
\usepackage{geometry}
\usepackage{amsmath}
\usepackage{booktabs}
\usepackage{hyperref}
\usepackage{xcolor}
\usepackage{listings}
\geometry{margin=1in}

\lstset{
  basicstyle=\ttfamily\small,
  keywordstyle=\color{blue},
  commentstyle=\color{gray},
  stringstyle=\color{teal},
  showstringspaces=false,
  columns=fullflexible,
  frame=single,
  framerule=0.2pt
}

\title{Python Programming\\Unit 03 -- Lecture 01 Notes\\File Handling Fundamentals and Directories}
\author{Tofik Ali}
\date{\today}

\begin{document}
\maketitle
\tableofcontents

\section{Lecture Overview}
Many programs become truly useful only when they can \textbf{store and load}
data. File handling allows you to:
\begin{itemize}
  \item save results (reports, logs, grades, summaries),
  \item reuse data later (datasets, configuration),
  \item and share data between programs.
\end{itemize}
This lecture covers file opening modes, reading/writing patterns, the \texttt{with}
statement, and basic directory handling using \texttt{pathlib}.

\section{Core Concepts}

\subsection{The \texttt{open()} Function}
The function \texttt{open(path, mode, encoding=...)} returns a file object.
You can use it to read or write.

\textbf{Best practice:} use \texttt{with open(...)} so the file is always closed:
\begin{lstlisting}[language=Python]
with open("data.txt", "r", encoding="utf-8") as f:
    content = f.read()
    print(content)
\end{lstlisting}

\subsection{File Modes (Read/Write/Append)}
Common modes (text mode):
\begin{center}
\begin{tabular}{ll}
  \toprule
  Mode & Meaning \\
  \midrule
  \texttt{r} & read (file must exist) \\
  \texttt{w} & write (overwrite if exists, create if not) \\
  \texttt{a} & append (write at end, create if not) \\
  \texttt{x} & create new (error if file exists) \\
  \texttt{t} & text mode (default) \\
  \texttt{b} & binary mode (images, PDFs, etc.) \\
  \texttt{+} & update mode (read + write) \\
  \bottomrule
\end{tabular}
\end{center}

\textbf{Important behavior:}
\begin{itemize}
  \item \texttt{"w"} truncates the file (clears old content). Use carefully.
  \item \texttt{"a"} keeps old content and adds new lines at the end.
  \item \texttt{"r"} fails if the file does not exist (\texttt{FileNotFoundError}).
\end{itemize}

\subsection{Reading Data}
\subsubsection*{Reading the whole file}
\begin{lstlisting}[language=Python]
with open("names.txt", "r", encoding="utf-8") as f:
    text = f.read()
\end{lstlisting}
Good for small files. Not recommended for very large files.

\subsubsection*{Reading line by line (recommended)}
\begin{lstlisting}[language=Python]
with open("names.txt", "r", encoding="utf-8") as f:
    for line in f:
        name = line.strip()
        print(name)
\end{lstlisting}
\texttt{strip()} removes the newline character \texttt{"\\n"} and extra spaces.

\subsection{Writing Data}
To write a line, use \texttt{write()} and include a newline \texttt{"\\n"}:
\begin{lstlisting}[language=Python]
with open("out.txt", "w", encoding="utf-8") as f:
    f.write("Alice\\n")
    f.write("Bob\\n")
\end{lstlisting}

To write many lines at once, use \texttt{writelines()}:
\begin{lstlisting}[language=Python]
lines = ["A\\n", "B\\n", "C\\n"]
with open("out.txt", "w", encoding="utf-8") as f:
    f.writelines(lines)
\end{lstlisting}

\subsection{Why \texttt{with} is Better Than Manual Close}
Without \texttt{with}, you must remember \texttt{f.close()}:
\begin{lstlisting}[language=Python]
f = open("data.txt", "r")
try:
    print(f.read())
finally:
    f.close()
\end{lstlisting}
\texttt{with} automatically does this for you. This matters on Windows because
open files can remain \emph{locked} if not closed properly.

\subsection{File Pointer: \texttt{tell()} and \texttt{seek()}}
Files have a current reading position.
\begin{lstlisting}[language=Python]
with open("data.txt", "r") as f:
    print("pos =", f.tell())
    first = f.readline()
    print("after readline pos =", f.tell())
    f.seek(0)  # move back to start
    again = f.readline()
\end{lstlisting}

\subsection{Working with Directories (Pathlib)}
The \texttt{pathlib} module gives an object-oriented way to handle paths.
\begin{lstlisting}[language=Python]
from pathlib import Path

root = Path(".")
print("Current folder:", root.resolve())

for p in root.iterdir():
    print(p.name, "dir" if p.is_dir() else "file")
\end{lstlisting}

\textbf{Common useful operations:}
\begin{itemize}
  \item Create a directory: \texttt{Path("data").mkdir(exist\_ok=True)}
  \item Join paths: \texttt{root / "data" / "names.txt"}
  \item Recursive search: \texttt{root.rglob("*.py")}
\end{itemize}

\section{Demo Walkthrough}
\subsection{Demo 1: Read/Write + Simple Stats}
\textbf{File:} \texttt{demo/file\_read\_write\_demo.py}

What it does:
\begin{itemize}
  \item Creates a file in \texttt{data/} and writes a few names.
  \item Reads the file back.
  \item Computes:
    \begin{itemize}
      \item number of names,
      \item longest name,
      \item count of names that start with a vowel.
    \end{itemize}
\end{itemize}

\subsection{Demo 2: Directory Walk}
\textbf{File:} \texttt{demo/directory\_walk\_demo.py}

What it does:
\begin{itemize}
  \item Prints a small ``tree'' of files under the lecture folder.
  \item Demonstrates \texttt{Path.rglob(...)} and basic path operations.
\end{itemize}

\section{Interactive Checkpoints (with Solutions)}

\subsection*{Checkpoint 1 Solution}
\textbf{Question:} difference between modes \texttt{"w"} and \texttt{"a"}?

\textbf{Answer:}
\begin{itemize}
  \item \texttt{"w"} overwrites (truncates) the file.
  \item \texttt{"a"} appends to the end of the file without deleting old content.
\end{itemize}

\subsection*{Checkpoint 2 Solution}
\textbf{Question:} what happens if you open a missing file using \texttt{"r"}?

\textbf{Answer:} Python raises \texttt{FileNotFoundError}.

\section{Practice Exercises (with Solutions)}

\subsection*{Exercise 1: Count Lines}
\textbf{Task:} Read a text file and print number of lines.

\textbf{Solution:}
\begin{lstlisting}[language=Python]
path = input("File path: ").strip()
count = 0
with open(path, "r", encoding="utf-8") as f:
    for _ in f:
        count += 1
print("Line count =", count)
\end{lstlisting}

\subsection*{Exercise 2: Copy a File}
\textbf{Task:} Copy text from one file to another.

\textbf{Solution:}
\begin{lstlisting}[language=Python]
src = input("Source file: ").strip()
dst = input("Destination file: ").strip()
with open(src, "r", encoding="utf-8") as f_in:
    with open(dst, "w", encoding="utf-8") as f_out:
        for line in f_in:
            f_out.write(line)
print("Copied.")
\end{lstlisting}

\subsection*{Exercise 3: List Python Files}
\textbf{Task:} Print all \texttt{.py} files inside the current directory and its subfolders.

\textbf{Solution:}
\begin{lstlisting}[language=Python]
from pathlib import Path
root = Path(".")
for p in root.rglob("*.py"):
    print(p)
\end{lstlisting}

\section{Exit Question (with Solution)}
\textbf{Question:} Read all lines from \texttt{"names.txt"} using \texttt{with} and print without newlines.

\textbf{Solution:}
\begin{lstlisting}[language=Python]
with open("names.txt", "r", encoding="utf-8") as f:
    for line in f:
        print(line.strip())
\end{lstlisting}

\end{document}

