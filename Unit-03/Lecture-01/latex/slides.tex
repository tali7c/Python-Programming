\documentclass{beamer}

\usetheme{Berlin}
\usecolortheme{Orchid}
\useoutertheme{miniframes}
\setbeamertemplate{navigation symbols}{}

\usepackage{amsmath}
\usepackage{booktabs}
\usepackage{graphicx}
\usepackage{xcolor}
\usepackage{listings}
\usepackage{hyperref}

\lstset{
  basicstyle=\ttfamily\small,
  keywordstyle=\color{blue},
  commentstyle=\color{gray},
  stringstyle=\color{teal},
  showstringspaces=false
}

\title[Python Programming]{Python Programming}
\subtitle{Unit 03 -- Lecture 01: File Handling Fundamentals and Directories}
\author{Tofik Ali}
\institute{School of Computer Science, UPES Dehradun}
\date{\today}

\begin{document}

\begin{frame}[fragile]
  \titlepage
  \vspace{0.5em}
  \begin{center}
  \footnotesize Repository: \texttt{https://github.com/tali7c/Python-Programming}
  \end{center}
\end{frame}

\begin{frame}[fragile]{Quick Links}
  \centering
  \hyperlink{sec:core}{\beamerbutton{Core Concepts}}\hspace{1em}
  \hyperlink{sec:demo}{\beamerbutton{Demo}}\hspace{1em}
  \hyperlink{sec:interactive}{\beamerbutton{Interactive}}\hspace{1em}
  \hyperlink{sec:summary}{\beamerbutton{Summary}}
\end{frame}

\begin{frame}[fragile]{Agenda}
  \tableofcontents
\end{frame}

\section{Core Concepts}
\label{sec:core}

\begin{frame}[fragile]{Learning Outcomes}
  \begin{itemize}[<+->]
    \item Open files using correct access modes (read/write/append)
    \item Read and write text data using common file methods
    \item Use the \texttt{with} statement to manage file resources safely
    \item Work with directories using \texttt{pathlib} / \texttt{os}
    \item Apply file handling to simple real-world tasks (logs, reports, data)
  \end{itemize}
\end{frame}

\begin{frame}[fragile]{Why File Handling?}
  \begin{itemize}[<+->]
    \item Data should \textbf{persist} after the program ends
    \item Files help in: logs, configuration, reports, datasets
    \item Working with files is a foundation for:
      \begin{itemize}
        \item CSV processing
        \item data analysis pipelines
        \item web and GUI apps that store data
      \end{itemize}
  \end{itemize}
\end{frame}

\begin{frame}[fragile]{The Basic Pattern}
  \begin{lstlisting}[language=Python]
with open("data.txt", "r", encoding="utf-8") as f:
    for line in f:
        print(line.strip())
  \end{lstlisting}
  \begin{itemize}[<+->]
    \item \texttt{open()} returns a file object
    \item Always close files (the \texttt{with} block does it automatically)
  \end{itemize}
\end{frame}

\begin{frame}[fragile]{File Modes (Text)}
  \small
  \begin{tabular}{ll}
    \toprule
    Mode & Meaning \\
    \midrule
    \texttt{r} & read (file must exist) \\
    \texttt{w} & write (overwrite / create new) \\
    \texttt{a} & append (write at end / create new) \\
    \texttt{x} & exclusive create (fail if exists) \\
    \texttt{t} & text mode (default) \\
    \texttt{b} & binary mode (images, PDFs, etc.) \\
    \texttt{+} & update (read + write) \\
    \bottomrule
  \end{tabular}
  \normalsize
\end{frame}

\begin{frame}[fragile]{Reading from a File}
  Common methods:
  \begin{itemize}[<+->]
    \item \texttt{read()} $\rightarrow$ whole file as one string
    \item \texttt{readline()} $\rightarrow$ one line
    \item \texttt{readlines()} $\rightarrow$ list of lines
    \item Best practice: iterate line-by-line (memory friendly)
  \end{itemize}
  \vspace{0.4em}
  \begin{lstlisting}[language=Python]
with open("names.txt", "r") as f:
    for line in f:
        name = line.strip()
        print(name)
  \end{lstlisting}
\end{frame}

\begin{frame}[fragile]{Writing to a File}
  \begin{itemize}[<+->]
    \item \texttt{write()} writes a string (you manage newlines)
    \item \texttt{writelines()} writes a list of strings
  \end{itemize}
  \vspace{0.4em}
  \begin{lstlisting}[language=Python]
lines = ["Alice\n", "Bob\n", "Charlie\n"]
with open("names.txt", "w") as f:
    f.writelines(lines)
  \end{lstlisting}
\end{frame}

\begin{frame}[fragile]{The \texttt{with} Statement}
  \begin{itemize}[<+->]
    \item A \textbf{context manager} that guarantees closing the file
    \item Even if an exception happens, the file is closed properly
    \item Reduces resource leaks and locking issues
  \end{itemize}
\end{frame}

\begin{frame}[fragile]{File Pointer: \texttt{tell()} and \texttt{seek()}}
  \begin{itemize}[<+->]
    \item \texttt{tell()} gives current position
    \item \texttt{seek(pos)} moves to a position
    \item Useful when you need to re-read or skip parts
  \end{itemize}
  \vspace{0.4em}
  \begin{lstlisting}[language=Python]
with open("data.txt", "r") as f:
    print(f.tell())
    first = f.readline()
    f.seek(0)
    again = f.readline()
  \end{lstlisting}
\end{frame}

\begin{frame}[fragile]{Working with Directories}
  \begin{itemize}[<+->]
    \item Use \texttt{pathlib.Path} for readable path handling
    \item Common tasks:
      \begin{itemize}
        \item list files in a folder
        \item create directories
        \item join paths safely
      \end{itemize}
  \end{itemize}
  \vspace{0.4em}
  \begin{lstlisting}[language=Python]
from pathlib import Path
p = Path(".")
for f in p.iterdir():
    print(f)
  \end{lstlisting}
\end{frame}

\section{Demo}
\label{sec:demo}

\begin{frame}[fragile]{Demo: Read/Write + Directory Listing}
  \begin{itemize}[<+->]
    \item \texttt{demo/file\_read\_write\_demo.py}
      \begin{itemize}
        \item writes a small file in \texttt{data/}
        \item reads it back and computes simple stats
      \end{itemize}
    \item \texttt{demo/directory\_walk\_demo.py}
      \begin{itemize}
        \item lists files and folders under the lecture directory
      \end{itemize}
  \end{itemize}
\end{frame}

\section{Interactive}
\label{sec:interactive}

\begin{frame}[fragile]{Checkpoint 1}
  \textbf{Question:} What is the difference between modes \texttt{"w"} and \texttt{"a"}?

  \vspace{0.6em}
  Write one example for each.
\end{frame}

\begin{frame}[fragile]{Checkpoint 2}
  \textbf{Question:} What happens if you open a non-existing file using \texttt{"r"}?
\end{frame}

\begin{frame}[fragile]{Think-Pair-Share}
  Discuss:
  \begin{itemize}
    \item What should be stored in files vs kept only in memory?
    \item Give 2 examples for each category.
  \end{itemize}
\end{frame}

\section{Summary}
\label{sec:summary}

\begin{frame}[fragile]{Key Takeaways}
  \begin{itemize}[<+->]
    \item Choose correct file mode: \texttt{r/w/a/x} (text vs binary)
    \item Prefer \texttt{with open(...)} to auto-close files safely
    \item Read line-by-line for large files
    \item Use \texttt{pathlib} for clean directory and path handling
  \end{itemize}
\end{frame}

\begin{frame}[fragile]{Exit Question}
  Write a code snippet to read all lines from \texttt{"names.txt"} safely
  using \texttt{with open(...)} and print them without newline characters.
\end{frame}

\end{document}

