\documentclass{beamer}

\usetheme{Berlin}
\usecolortheme{Orchid}
\useoutertheme{miniframes}
\setbeamertemplate{navigation symbols}{}

\usepackage{amsmath}
\usepackage{booktabs}
\usepackage{graphicx}
\usepackage{xcolor}
\usepackage{listings}
\usepackage{hyperref}

\lstset{
  basicstyle=\ttfamily\small,
  keywordstyle=\color{blue},
  commentstyle=\color{gray},
  stringstyle=\color{teal},
  showstringspaces=false
}

\title[Python Programming]{Python Programming}
\subtitle{Unit 03 -- Lecture 03: Modules, Packages, and the Standard Library}
\author{Tofik Ali}
\institute{School of Computer Science, UPES Dehradun}
\date{\today}

\begin{document}

\begin{frame}[fragile]
  \titlepage
  \vspace{0.5em}
  \begin{center}
  \footnotesize Repository: \texttt{https://github.com/tali7c/Python-Programming}
  \end{center}
\end{frame}

\begin{frame}[fragile]{Quick Links}
  \centering
  \hyperlink{sec:core}{\beamerbutton{Core Concepts}}\hspace{1em}
  \hyperlink{sec:demo}{\beamerbutton{Demo}}\hspace{1em}
  \hyperlink{sec:interactive}{\beamerbutton{Interactive}}\hspace{1em}
  \hyperlink{sec:summary}{\beamerbutton{Summary}}
\end{frame}

\begin{frame}[fragile]{Agenda}
  \tableofcontents
\end{frame}

\section{Core Concepts}
\label{sec:core}

\begin{frame}[fragile]{Learning Outcomes}
  \begin{itemize}[<+->]
    \item Explain what modules and packages are in Python
    \item Import modules using different import styles
    \item Use \texttt{\_\_name\_\_ == "\_\_main\_\_"} to control script execution
    \item Use common standard modules: \texttt{sys}, \texttt{math}, \texttt{time}, \texttt{os}, \texttt{pathlib}
  \end{itemize}
\end{frame}

\begin{frame}[fragile]{Why Modules?}
  \begin{itemize}[<+->]
    \item Organize code into reusable files
    \item Improve readability and maintainability
    \item Enable teamwork (different files for different features)
    \item Avoid copying-pasting functions between scripts
  \end{itemize}
\end{frame}

\begin{frame}[fragile]{A Module is a \texttt{.py} File}
  Example:
  \begin{itemize}[<+->]
    \item \texttt{math\_helpers.py} contains functions
    \item another script imports and uses them
  \end{itemize}
  \vspace{0.4em}
  \begin{lstlisting}[language=Python]
import math
print(math.sqrt(16))
  \end{lstlisting}
\end{frame}

\begin{frame}[fragile]{Import Styles}
  \begin{itemize}[<+->]
    \item \texttt{import module}
    \item \texttt{import module as alias}
    \item \texttt{from module import name}
    \item \texttt{from module import name as alias}
  \end{itemize}
  \vspace{0.4em}
  \begin{lstlisting}[language=Python]
import math as m
from math import pi, sqrt
  \end{lstlisting}
\end{frame}

\begin{frame}[fragile]{\texttt{\_\_name\_\_} and Script Entry Point}
  \begin{itemize}[<+->]
    \item If a file is executed directly, \texttt{\_\_name\_\_} is \texttt{"\_\_main\_\_"}
    \item If the file is imported, \texttt{\_\_name\_\_} is the module name
  \end{itemize}
  \vspace{0.4em}
  \begin{lstlisting}[language=Python]
def main():
    print("Running as a script")

if __name__ == "__main__":
    main()
  \end{lstlisting}
\end{frame}

\begin{frame}[fragile]{What is a Package?}
  \begin{itemize}[<+->]
    \item A package is a folder that groups modules
    \item Conventionally contains \texttt{\_\_init\_\_.py}
  \end{itemize}
  \vspace{0.4em}
  Example structure:
  \begin{lstlisting}
demo/
  my_utils/
    __init__.py
    math_helpers.py
    text_helpers.py
  use_my_utils.py
  \end{lstlisting}
\end{frame}

\begin{frame}[fragile]{Standard Modules You Should Know}
  \begin{itemize}[<+->]
    \item \texttt{sys}: command-line args, Python path
    \item \texttt{math}: math functions/constants
    \item \texttt{time}: timestamps, delays
    \item \texttt{os} and \texttt{pathlib}: filesystem and paths
  \end{itemize}
\end{frame}

\begin{frame}[fragile]{Example: \texttt{sys.argv}}
  \begin{lstlisting}[language=Python]
import sys
print(sys.argv)  # list of command-line arguments
  \end{lstlisting}
  \begin{itemize}[<+->]
    \item Useful when building scripts that take inputs from terminal
  \end{itemize}
\end{frame}

\section{Demo}
\label{sec:demo}

\begin{frame}[fragile]{Demo: Create a Small Package}
  \begin{itemize}[<+->]
    \item Package: \texttt{demo/my\_utils/}
    \item Script: \texttt{demo/use\_my\_utils.py}
    \item Also demonstrates standard modules (\texttt{sys}, \texttt{math}, \texttt{time}, \texttt{pathlib})
  \end{itemize}
\end{frame}

\section{Interactive}
\label{sec:interactive}

\begin{frame}[fragile]{Checkpoint 1}
  \textbf{Question:} What is the value of \texttt{\_\_name\_\_} when:
  \begin{itemize}
    \item you run a file directly?
    \item you import that file as a module?
  \end{itemize}
\end{frame}

\begin{frame}[fragile]{Checkpoint 2}
  \textbf{Question:} When should you prefer \texttt{import module} over
  \texttt{from module import name}?
\end{frame}

\begin{frame}[fragile]{Think-Pair-Share}
  You have a project with 200 lines of code in one file.
  Discuss how you would split it into modules:
  \begin{itemize}
    \item What goes into \texttt{utils.py}?
    \item What stays in \texttt{main.py}?
  \end{itemize}
\end{frame}

\section{Summary}
\label{sec:summary}

\begin{frame}[fragile]{Key Takeaways}
  \begin{itemize}[<+->]
    \item Modules and packages help organize and reuse code
    \item \texttt{\_\_name\_\_ == "\_\_main\_\_"} controls what runs on import
    \item The standard library provides powerful tools without extra installs
  \end{itemize}
\end{frame}

\begin{frame}[fragile]{Exit Question}
  Create a package named \texttt{tools} with a module \texttt{helpers.py}.
  Which file makes \texttt{tools} a package?
\end{frame}

\end{document}

