\documentclass[11pt]{article}
\usepackage[utf8]{inputenc}
\usepackage[T1]{fontenc}
\usepackage{geometry}
\usepackage{amsmath}
\usepackage{hyperref}
\usepackage{xcolor}
\usepackage{listings}
\geometry{margin=1in}

\lstset{
  basicstyle=\ttfamily\small,
  keywordstyle=\color{blue},
  commentstyle=\color{gray},
  stringstyle=\color{teal},
  showstringspaces=false,
  columns=fullflexible,
  frame=single,
  framerule=0.2pt
}

\title{Python Programming\\Unit 03 -- Lecture 04 Notes\\Regular Expressions (Regex) in Python}
\author{Tofik Ali}
\date{\today}

\begin{document}
\maketitle
\tableofcontents

\section{Lecture Overview}
Regular expressions (regex) are a powerful way to describe text patterns.
They are widely used in:
\begin{itemize}
  \item input validation (phone numbers, emails, dates),
  \item information extraction (finding all numbers, IDs, links),
  \item and cleaning data (replacing unwanted characters).
\end{itemize}
This lecture focuses on practical regex features that you can use in beginner programs.

\section{Core Concepts}

\subsection{Use Raw Strings for Patterns}
Regex patterns contain backslashes like \texttt{\textbackslash d}. Python strings also use backslashes for escapes.
So it is recommended to use raw strings:
\begin{lstlisting}[language=Python]
pattern = r"\d{4}-\d{2}-\d{2}"
\end{lstlisting}

\subsection{Common Meta Characters (Quick Reference)}
\begin{itemize}
  \item \texttt{.} any character (except newline)
  \item \texttt{[abc]} one of a,b,c
  \item \texttt{[a-z]} range
  \item \texttt{\textbackslash d} digit, \texttt{\textbackslash w} word char, \texttt{\textbackslash s} whitespace
  \item Anchors: \texttt{\string^} start of string, \texttt{\$} end of string
  \item Grouping: \texttt{(...)} creates a group
  \item Alternation: \texttt{|} means OR
\end{itemize}

\subsection{Quantifiers}
\begin{itemize}
  \item \texttt{*} 0 or more
  \item \texttt{+} 1 or more
  \item \texttt{?} 0 or 1
  \item \texttt{\{m\}} exactly m times
  \item \texttt{\{m,n\}} between m and n times
\end{itemize}

Examples:
\begin{lstlisting}[language=Python]
r"\d+"      # one or more digits
r"\d{10}"   # exactly 10 digits
r"[A-Z]{2}\d{4}"  # like AB1234
\end{lstlisting}

\subsection{Core \texttt{re} Functions}
\begin{itemize}
  \item \texttt{re.search(p, text)}: match anywhere
  \item \texttt{re.match(p, text)}: match only from the start
  \item \texttt{re.fullmatch(p, text)}: entire string must match (great for validation)
  \item \texttt{re.findall(p, text)}: all matches
  \item \texttt{re.sub(p, repl, text)}: replace all matches
\end{itemize}

\begin{lstlisting}[language=Python]
import re
text = "Marks: 75, 88, 92"
nums = re.findall(r"\d+", text)  # ["75", "88", "92"]
\end{lstlisting}

\subsection{Validation vs Extraction}
\textbf{Validation:} use \texttt{fullmatch}.
\begin{lstlisting}[language=Python]
import re
pin = input("PIN: ").strip()
if re.fullmatch(r"\d{6}", pin):
    print("Valid PIN")
\end{lstlisting}

\textbf{Extraction:} use \texttt{findall} or \texttt{finditer}.
\begin{lstlisting}[language=Python]
import re
text = "Call 9876543210 or 9123456789"
phones = re.findall(r"\b\d{10}\b", text)
\end{lstlisting}

\section{Demo Walkthrough}
\textbf{File:} \texttt{demo/regex\_extractor\_demo.py}

\subsection*{What it does}
\begin{itemize}
  \item Finds all email addresses in a text block.
  \item Finds all 10-digit phone numbers.
  \item Cleans multiple spaces into a single space using \texttt{re.sub}.
  \item Validates a user-provided phone number using \texttt{fullmatch}.
\end{itemize}

\section{Interactive Checkpoints (with Solutions)}

\subsection*{Checkpoint 1 Solution}
\textbf{Question:} difference between \texttt{search} and \texttt{fullmatch}?

\textbf{Answer:}
\begin{itemize}
  \item \texttt{search} finds a match anywhere inside the text.
  \item \texttt{fullmatch} requires the entire string to match the pattern (best for validation).
\end{itemize}

\subsection*{Checkpoint 2 Solution}
\textbf{Question:} regex for 4-digit PIN? Should it allow leading zeros?

\textbf{Answer:} A 4-digit PIN pattern is \texttt{r"\textbackslash d\{4\}"}.
If leading zeros are allowed, this is fine. If you want to disallow leading zeros,
use \texttt{r"[1-9]\textbackslash d\{3\}"}.

\section{Practice Exercises (with Solutions)}

\subsection*{Exercise 1: Extract All Integers}
\textbf{Task:} Extract all integer numbers from a string.

\textbf{Solution:}
\begin{lstlisting}[language=Python]
import re
text = input("Text: ")
nums = re.findall(r"-?\d+", text)
print(nums)
\end{lstlisting}

\subsection*{Exercise 2: Replace Multiple Spaces}
\textbf{Task:} Replace multiple spaces with a single space.

\textbf{Solution:}
\begin{lstlisting}[language=Python]
import re
s = input("Enter text: ")
clean = re.sub(r"\s+", " ", s).strip()
print(clean)
\end{lstlisting}

\subsection*{Exercise 3: Validate Date (YYYY-MM-DD)}
\textbf{Task:} Validate format \texttt{YYYY-MM-DD}. (Format only, not calendar correctness.)

\textbf{Solution:}
\begin{lstlisting}[language=Python]
import re
date = input("Date: ").strip()
if re.fullmatch(r"\d{4}-\d{2}-\d{2}", date):
    print("Valid format")
else:
    print("Invalid format")
\end{lstlisting}

\section{Exit Question (with Solution)}
\textbf{Question:} regex pattern for \texttt{YYYY-MM-DD}?

\textbf{Answer:} \texttt{r"\textbackslash d\{4\}-\textbackslash d\{2\}-\textbackslash d\{2\}"}

\end{document}

