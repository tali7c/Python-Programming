\documentclass[11pt]{article}
\usepackage[utf8]{inputenc}
\usepackage[T1]{fontenc}
\usepackage{geometry}
\usepackage{amsmath}
\usepackage{booktabs}
\usepackage{hyperref}
\usepackage{xcolor}
\usepackage{listings}
\geometry{margin=1in}

\lstset{
  basicstyle=\ttfamily\small,
  keywordstyle=\color{blue},
  commentstyle=\color{gray},
  stringstyle=\color{teal},
  showstringspaces=false,
  columns=fullflexible,
  frame=single,
  framerule=0.2pt
}

\title{Python Programming\\Unit 02 -- Lecture 03 Notes\\Sets and Dictionaries}
\author{Tofik Ali}
\date{\today}

\begin{document}
\maketitle
\tableofcontents

\section{Lecture Overview}
This lecture introduces two extremely practical collection types:
\begin{itemize}
  \item \textbf{sets} for uniqueness and fast membership checking, and
  \item \textbf{dictionaries} for key-value modeling and frequency counting.
\end{itemize}
Students should leave with a clear mental model: \emph{use a set when you care
about ``is it present?'' and uniqueness; use a dictionary when you care about
``for this key, what value should I store?''}

\section{Sets}

\subsection{What Is a Set?}
A set is an \textbf{unordered} collection of \textbf{unique} elements.
When you add duplicates, they are automatically removed.

\subsection{Creating Sets and Testing Membership}
\begin{lstlisting}[language=Python]
colors = {"red", "blue", "red"}
print(colors)          # duplicates removed
print("red" in colors) # membership check

s = set([1, 2, 2, 3])
empty_set = set()      # NOT {}
\end{lstlisting}
\textbf{Important:} \texttt{\{\}} creates an empty dictionary, not an empty set.

\subsection{Set Operations (Set Algebra)}
Given two sets \texttt{A} and \texttt{B}:
\begin{itemize}
  \item Union: \texttt{A | B} (elements in either)
  \item Intersection: \texttt{A \& B} (elements common to both)
  \item Difference: \texttt{A - B} (in A but not in B)
  \item Symmetric difference: \texttt{A \string^ B} (in exactly one of them)
\end{itemize}
Example:
\begin{lstlisting}[language=Python]
A = {1, 2, 3}
B = {2, 3, 4}
A | B   # {1, 2, 3, 4}
A & B   # {2, 3}
A - B   # {1}
A ^ B   # {1, 4}
\end{lstlisting}

\subsection{When Sets Are the Right Tool}
Sets shine in problems like:
\begin{itemize}
  \item removing duplicates (de-duplication),
  \item checking if something has been seen before, and
  \item computing overlap between two groups.
\end{itemize}
Membership tests in sets are typically fast (average-case constant time), which
is why sets are preferred for repeated \texttt{in} checks.

\subsection{Pitfalls with Sets}
\begin{itemize}
  \item \textbf{No indexing:} sets are unordered, so \texttt{s[0]} is invalid.
  \item \textbf{Elements must be hashable:} you cannot put a list inside a set.
\end{itemize}
If order is needed, convert to a sorted list:
\begin{lstlisting}[language=Python]
s = {3, 1, 2}
ordered = sorted(s)   # [1, 2, 3]
\end{lstlisting}

\section{Dictionaries}

\subsection{What Is a Dictionary?}
A dictionary maps \textbf{keys} to \textbf{values}. Keys are unique and must be
immutable (hashable). Values can be any type.

\subsection{Creation and Basic Operations}
\begin{lstlisting}[language=Python]
student = {"name": "Asha", "marks": 88}
student["name"]          # access
student["marks"] = 90     # update

for k, v in student.items():
    print(k, v)
\end{lstlisting}

\subsection{Access: \texttt{d[key]} vs \texttt{d.get(key)}}
\texttt{d[key]} raises a \texttt{KeyError} if the key does not exist.
\texttt{d.get(key)} returns \texttt{None} by default (or a custom default).
\begin{lstlisting}[language=Python]
d = {"a": 1}
# d["b"]           # KeyError

d.get("b")          # None
d.get("b", 0)       # 0
\end{lstlisting}
\textbf{Rule of thumb:}
\begin{itemize}
  \item Use \texttt{d[key]} when missing keys are a real error.
  \item Use \texttt{get} when missing keys are expected.
\end{itemize}

\subsection{Worked Pattern: Frequency Counting}
Counting is one of the most common dictionary use cases.
\begin{lstlisting}[language=Python]
text = "to be or not to be"
freq = {}
for w in text.split():
    freq[w] = freq.get(w, 0) + 1
\end{lstlisting}
The key idea is \texttt{get(w, 0)} which uses 0 for unseen words.

\subsection{Sorting Dictionary Items}
Dictionaries themselves are mappings; sorting creates a \emph{list of pairs}:
\begin{lstlisting}[language=Python]
counts = {"a": 3, "b": 1, "c": 2}
by_value = sorted(counts.items(), key=lambda kv: kv[1])
by_value_desc = sorted(counts.items(), key=lambda kv: kv[1], reverse=True)
\end{lstlisting}
The result is a list like \texttt{[("b", 1), ("c", 2), ("a", 3)]}.

\section{Demo Walkthrough: Word Frequency and Unique Words}
\textbf{Script:} \texttt{demo/word\_frequency.py}

\subsection{What the Demo Is Teaching}
\begin{itemize}
  \item Use a set to get the unique vocabulary.
  \item Use a dictionary to count word occurrences.
  \item (Extension) Sort counts to find top frequent words.
\end{itemize}

\subsection{Suggested Run Steps}
\begin{lstlisting}
python demo/word_frequency.py
\end{lstlisting}
Suggested student activity: change the sentence and predict which word becomes
most frequent.

\section{Quiz and Solutions}

\subsection{Checkpoint 1 (Set Removes Duplicates)}
\textbf{Question.} What is the result of \texttt{set([1,1,2,2])}?

\textbf{Solution.} \texttt{\{1, 2\}} (order does not matter).

\subsection{Checkpoint 2 (\texttt{d[key]} vs \texttt{get})}
\textbf{Question.} What is the difference between \texttt{d[key]} and
\texttt{d.get(key)}?

\textbf{Solution.} \texttt{d[key]} raises \texttt{KeyError} if the key is
missing. \texttt{d.get(key)} returns \texttt{None} (or a provided default) and
avoids the exception.

\subsection{Think-Pair-Share (Set vs List for Membership)}
\textbf{Prompt.} Choose set vs list for membership checks and justify.

\textbf{Sample response.} Use a set when you repeatedly check \texttt{x in data}
for many \texttt{x}. A set is designed for fast membership; a list must scan
values one by one, which becomes slow as the list grows.

\subsection{Exit Question (Sort by Value Descending)}
\textbf{Question.} Write one line to sort dictionary items by value in
descending order.

\textbf{Solution.}
\begin{lstlisting}[language=Python]
sorted(d.items(), key=lambda kv: kv[1], reverse=True)
\end{lstlisting}

\section{Practice Exercises (With Solutions)}

\subsection{Exercise 1: Unique Words (Case-Insensitive)}
\textbf{Task.} Given a sentence, print the number of unique words ignoring
case.

\textbf{Solution.}
\begin{lstlisting}[language=Python]
text = "Python python is FUN fun"
words = [w.lower() for w in text.split()]
unique = set(words)
print(len(unique))
\end{lstlisting}

\subsection{Exercise 2: Top-3 Frequent Words}
\textbf{Task.} From a text, print the top-3 words by frequency.

\textbf{Solution.}
\begin{lstlisting}[language=Python]
text = "to be or not to be"
freq = {}
for w in text.split():
    freq[w] = freq.get(w, 0) + 1

top3 = sorted(freq.items(), key=lambda kv: kv[1], reverse=True)[:3]
print(top3)
\end{lstlisting}

\subsection{Exercise 3: Jaccard Similarity}
\textbf{Task.} Given two sets \texttt{A} and \texttt{B}, compute
$J(A,B)=\frac{|A\cap B|}{|A\cup B|}$.

\textbf{Solution.}
\begin{lstlisting}[language=Python]
def jaccard(A, B):
    inter = A & B
    uni = A | B
    return len(inter) / len(uni) if uni else 0.0
\end{lstlisting}

\section{Further Reading}
\begin{itemize}
  \item \url{https://docs.python.org/3/tutorial/datastructures.html#sets}
  \item \url{https://docs.python.org/3/tutorial/datastructures.html#dictionaries}
\end{itemize}

\end{document}
