\documentclass{beamer}

\usetheme{Berlin}
\usecolortheme{Orchid}
\useoutertheme{miniframes}
\setbeamertemplate{navigation symbols}{}

\usepackage{amsmath}
\usepackage{booktabs}
\usepackage{graphicx}
\usepackage{xcolor}
\usepackage{listings}
\usepackage{hyperref}

\lstset{
  basicstyle=\ttfamily\small,
  keywordstyle=\color{blue},
  commentstyle=\color{gray},
  stringstyle=\color{teal},
  showstringspaces=false
}

\title[Python Programming]{Python Programming}
\subtitle{Unit 02 -- Lecture 02: Tuples and Immutability}
\author{Tofik Ali}
\institute{School of Computer Science, UPES Dehradun}
\date{\today}

\begin{document}

\begin{frame}[fragile]
  \titlepage
  \vspace{0.5em}
  \begin{center}
  \footnotesize Repository: \texttt{https://github.com/tali7c/Python-Programming}
  \end{center}
\end{frame}

\begin{frame}[fragile]{Quick Links}
  \centering
  \hyperlink{sec:core}{\beamerbutton{Core Concepts}}\hspace{1em}
  \hyperlink{sec:demo}{\beamerbutton{Demo}}\hspace{1em}
  \hyperlink{sec:interactive}{\beamerbutton{Interactive}}\hspace{1em}
  \hyperlink{sec:summary}{\beamerbutton{Summary}}
\end{frame}

\begin{frame}[fragile]{Agenda}
  \tableofcontents
\end{frame}

\section{Overview}

\section{Core Concepts}
\label{sec:core}

\begin{frame}[fragile]{Learning Outcomes}
  \begin{itemize}[<+->]
    \item Create tuples and use indexing and slicing
    \item Apply tuple packing and unpacking
    \item Compare tuples with lists and choose appropriately
    \item Use tuples in real-world scenarios
  \end{itemize}
\end{frame}

\begin{frame}[fragile]{Why Tuples?}
  \begin{itemize}[<+->]
    \item Immutable sequences (cannot be changed)
    \item Useful for fixed records and safe data
    \item Can be used as dictionary keys (if elements are hashable)
    \item Communicates intent: “this should not change”
  \end{itemize}
\end{frame}

\begin{frame}[fragile]{Creating Tuples}
  \begin{itemize}[<+->]
    \item Literal: \texttt{point = (3, 4)}
    \item Packing: \texttt{t = 1, 2, 3}
    \item Singleton requires a comma: \texttt{single = (5,)}
  \end{itemize}
\end{frame}

\begin{frame}[fragile]{Packing and Unpacking}
  \begin{lstlisting}[language=Python]
point = (10, 20)
x, y = point

values = (1, 2, 3, 4)
a, b, *rest = values
  \end{lstlisting}
  \begin{itemize}[<+->]
    \item Unpacking maps tuple positions to variables
    \item \texttt{*rest} collects remaining values
  \end{itemize}
\end{frame}

\begin{frame}[fragile]{Worked Example: Swap and Multi-Return}
  \begin{lstlisting}[language=Python]
a, b = 5, 9
a, b = b, a     # swap

def min_max(nums):
    return (min(nums), max(nums))

mn, mx = min_max([3, 1, 7, 2])
  \end{lstlisting}
  \begin{itemize}[<+->]
    \item Tuples are a common way to return multiple values
  \end{itemize}
\end{frame}

\begin{frame}[fragile]{Tuple Methods and Operations}
  \begin{itemize}[<+->]
    \item Methods: \texttt{count}, \texttt{index}
    \item Supports indexing, slicing, concatenation, repetition
    \item Example: \texttt{(1,2) + (3,4)}
  \end{itemize}
\end{frame}

\begin{frame}[fragile]{Tuples vs Lists}
  \small
  \begin{tabular}{lll}
    \toprule
    Feature & List & Tuple \\
    \midrule
    Mutability & Mutable & Immutable \\
    Syntax & [] & () \\
    Use case & Changeable data & Fixed records \\
    Hashable & No & Yes (if items hashable) \\
    \bottomrule
  \end{tabular}
  \normalsize
\end{frame}

\begin{frame}[fragile]{Nested Tuples and Conversion}
  \begin{itemize}[<+->]
    \item Tuples can contain other tuples
    \item Convert between list and tuple as needed
  \end{itemize}
  \vspace{0.4em}
  \begin{lstlisting}[language=Python]
coords = ((0, 0), (1, 1), (2, 2))
coords_list = list(coords)
coords_tuple = tuple(coords_list)
  \end{lstlisting}
\end{frame}

\begin{frame}[fragile]{Pitfall: Singleton Tuples and “Immutable”}
  \begin{itemize}[<+->]
    \item \texttt{(5)} is an integer in parentheses; use \texttt{(5,)} for a tuple
    \item Tuples are immutable, but they can contain mutable objects
  \end{itemize}
  \vspace{0.4em}
  \begin{lstlisting}[language=Python]
t = ([1, 2],)     # tuple containing a list
t[0].append(3)    # allowed: the list mutates
  \end{lstlisting}
\end{frame}

\begin{frame}[fragile]{Common Use Cases}
  \begin{itemize}[<+->]
    \item Coordinates and geometry data
    \item Dictionary keys (composite keys)
    \item Returning multiple values from functions
  \end{itemize}
\end{frame}

\section{Demo}
\label{sec:demo}

\begin{frame}[fragile]{Demo: Distance Between Points}
  \begin{itemize}[<+->]
    \item Store coordinates as tuples
    \item Unpack and compute distance
    \item Keep data immutable
    \item Extension: compute distances for multiple point pairs
  \end{itemize}
  \vspace{0.4em}
  \textbf{Script:} \texttt{demo/tuple\_distance.py}
\end{frame}

\section{Interactive}
\label{sec:interactive}

\begin{frame}[fragile]{Checkpoint 1}
  Why does \texttt{(5)} not create a tuple?
  \begin{enumerate}[<+->]
    \item Parentheses alone do not create tuples
    \item A trailing comma is required
    \item \texttt{(5)} is just an integer in parentheses
  \end{enumerate}
\end{frame}

\begin{frame}[fragile]{Checkpoint 2}
  Predict the result:
  \begin{lstlisting}[language=Python]
values = (10, 20, 30, 40, 50)
a, b, *rest = values
  \end{lstlisting}
  What are \texttt{a}, \texttt{b}, and \texttt{rest}?
\end{frame}

\begin{frame}[fragile]{Think-Pair-Share}
  Where in programs should immutability be preferred?
  Discuss one example from daily software use.
\end{frame}

\section{Summary}
\label{sec:summary}

\begin{frame}[fragile]{Summary}
  \begin{itemize}[<+->]
    \item Tuples are immutable sequences
    \item Packing/unpacking simplifies assignment and returns
    \item Use tuples for fixed records and keys
    \item Convert between lists and tuples when needed
  \end{itemize}
\end{frame}

\begin{frame}[fragile]{Exit Question}
  Write one line to convert a list \texttt{L} to a tuple and back.
\end{frame}

\end{document}
