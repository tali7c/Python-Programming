\documentclass[11pt]{article}
\usepackage[utf8]{inputenc}
\usepackage[T1]{fontenc}
\usepackage{geometry}
\usepackage{amsmath}
\usepackage{hyperref}
\usepackage{xcolor}
\usepackage{listings}
\geometry{margin=1in}

\lstset{
  basicstyle=\ttfamily\small,
  keywordstyle=\color{blue},
  commentstyle=\color{gray},
  stringstyle=\color{teal},
  showstringspaces=false,
  columns=fullflexible,
  frame=single,
  framerule=0.2pt
}

\title{Python Programming\\Unit 06 -- Lecture 06 Notes\\Plot Types and Subplots}
\author{Tofik Ali}
\date{\today}

\begin{document}
\maketitle
\tableofcontents

\section{Lecture Overview}
Different plot types are used for different questions:
\begin{itemize}
  \item trends $\rightarrow$ line plots,
  \item comparisons $\rightarrow$ bar charts,
  \item distributions $\rightarrow$ histograms,
  \item relationships $\rightarrow$ scatter plots.
\end{itemize}
Subplots help you combine multiple plots in one figure.

\section{Core Concepts}

\subsection{Line Plot}
Good for trends (time series).
\begin{lstlisting}[language=Python]
plt.plot(x, y)
\end{lstlisting}

\subsection{Bar Chart}
Good for category comparisons.
\begin{lstlisting}[language=Python]
plt.bar(categories, values)
\end{lstlisting}

\subsection{Histogram}
Good for distributions.
\begin{lstlisting}[language=Python]
plt.hist(values, bins=10)
\end{lstlisting}

\subsection{Scatter Plot}
Good for relationships between two numeric variables.
\begin{lstlisting}[language=Python]
plt.scatter(x, y)
\end{lstlisting}

\subsection{Pie Chart (Use Carefully)}
Pie charts show proportions, but can be hard to compare precisely.
\begin{lstlisting}[language=Python]
plt.pie(values, labels=labels, autopct="%1.1f%%")
\end{lstlisting}

\subsection{Subplots}
Create multiple axes in one figure:
\begin{lstlisting}[language=Python]
fig, ax = plt.subplots(2, 2, figsize=(8, 6))
ax[0, 0].plot(x, y)
\end{lstlisting}

\section{Demo Walkthrough}
\textbf{File:} \texttt{demo/matplotlib\_plot\_types\_demo.py}

The demo creates a 2x2 subplot figure and saves it into \texttt{images/}.

\section{Interactive Checkpoints (with Solutions)}

\subsection*{Checkpoint 1 Solution}
\textbf{Question:} distribution of marks?

\textbf{Answer:} Histogram.

\subsection*{Checkpoint 2 Solution}
\textbf{Question:} compare marks of 5 subjects?

\textbf{Answer:} Bar chart.

\section{Practice Exercises (with Solutions)}

\subsection*{Exercise 1: Histogram of Random Numbers}
\textbf{Solution:}
\begin{lstlisting}[language=Python]
import numpy as np
import matplotlib.pyplot as plt

x = np.random.randn(1000)
plt.hist(x, bins=20)
plt.title("Histogram")
plt.show()
\end{lstlisting}

\section{Exit Question (with Solution)}
\textbf{Question:} function name used to create subplots?

\textbf{Answer:} \texttt{subplots}

\end{document}

