\documentclass{beamer}

\usetheme{Berlin}
\usecolortheme{Orchid}
\useoutertheme{miniframes}
\setbeamertemplate{navigation symbols}{}

\usepackage{amsmath}
\usepackage{booktabs}
\usepackage{graphicx}
\usepackage{xcolor}
\usepackage{listings}
\usepackage{hyperref}

\lstset{
  basicstyle=\ttfamily\small,
  keywordstyle=\color{blue},
  commentstyle=\color{gray},
  stringstyle=\color{teal},
  showstringspaces=false
}

\title[Python Programming]{Python Programming}
\subtitle{Unit 06 -- Lecture 06: Plot Types and Subplots}
\author{Tofik Ali}
\institute{School of Computer Science, UPES Dehradun}
\date{\today}

\begin{document}

\begin{frame}[fragile]
  \titlepage
  \vspace{0.5em}
  \begin{center}
  \footnotesize Repository: \texttt{https://github.com/tali7c/Python-Programming}
  \end{center}
\end{frame}

\begin{frame}[fragile]{Quick Links}
  \centering
  \hyperlink{sec:core}{\beamerbutton{Core Concepts}}\hspace{1em}
  \hyperlink{sec:demo}{\beamerbutton{Demo}}\hspace{1em}
  \hyperlink{sec:interactive}{\beamerbutton{Interactive}}\hspace{1em}
  \hyperlink{sec:summary}{\beamerbutton{Summary}}
\end{frame}

\begin{frame}[fragile]{Agenda}
  \tableofcontents
\end{frame}

\section{Core Concepts}
\label{sec:core}

\begin{frame}[fragile]{Learning Outcomes}
  \begin{itemize}[<+->]
    \item Create different plot types: line, bar, histogram, scatter, pie
    \item Build multi-plot figures using subplots
    \item Choose the right plot type for the data story
  \end{itemize}
\end{frame}

\begin{frame}[fragile]{Common Plot Types}
  \begin{itemize}[<+->]
    \item Line plot: trend over time
    \item Bar chart: compare categories
    \item Histogram: distribution of values
    \item Scatter plot: relationship between two variables
    \item Pie chart: parts of a whole (use carefully)
  \end{itemize}
\end{frame}

\begin{frame}[fragile]{Subplots}
  \begin{itemize}[<+->]
    \item Subplots allow multiple plots in one figure
    \item Use \texttt{plt.subplots(rows, cols)}
  \end{itemize}
  \vspace{0.4em}
  \begin{lstlisting}[language=Python]
fig, ax = plt.subplots(2, 2)
ax[0, 0].plot(x, y)
  \end{lstlisting}
\end{frame}

\section{Demo}
\label{sec:demo}

\begin{frame}[fragile]{Demo: Multiple Plot Types in One Figure}
  \begin{itemize}[<+->]
    \item File: \texttt{demo/matplotlib\_plot\_types\_demo.py}
    \item Generates a figure with subplots and saves it to \texttt{images/}
  \end{itemize}
\end{frame}

\section{Interactive}
\label{sec:interactive}

\begin{frame}[fragile]{Checkpoint 1}
  \textbf{Question:} Which plot type would you use to show distribution of marks of 100 students?
\end{frame}

\begin{frame}[fragile]{Checkpoint 2}
  \textbf{Question:} Which plot type would you use to compare marks of 5 subjects?
\end{frame}

\begin{frame}[fragile]{Think-Pair-Share}
  Discuss:
  \begin{itemize}
    \item Why can pie charts be misleading?
  \end{itemize}
\end{frame}

\section{Summary}
\label{sec:summary}

\begin{frame}[fragile]{Key Takeaways}
  \begin{itemize}[<+->]
    \item Different plot types answer different questions
    \item Subplots help compare multiple views of data in one figure
    \item Always label and choose scales carefully for honest visuals
  \end{itemize}
\end{frame}

\begin{frame}[fragile]{Exit Question}
  Write the function name used to create subplots in Matplotlib.
\end{frame}

\end{document}

