\documentclass[11pt]{article}
\usepackage[utf8]{inputenc}
\usepackage[T1]{fontenc}
\usepackage{geometry}
\usepackage{amsmath}
\usepackage{hyperref}
\usepackage{xcolor}
\usepackage{listings}
\geometry{margin=1in}

\lstset{
  basicstyle=\ttfamily\small,
  keywordstyle=\color{blue},
  commentstyle=\color{gray},
  stringstyle=\color{teal},
  showstringspaces=false,
  columns=fullflexible,
  frame=single,
  framerule=0.2pt
}

\title{Python Programming\\Unit 06 -- Lecture 01 Notes\\NumPy Basics (ndarray, Operations, Broadcasting)}
\author{Tofik Ali}
\date{\today}

\begin{document}
\maketitle
\tableofcontents

\section{Lecture Overview}
NumPy is the core library for numerical computing in Python. It provides
\textbf{ndarray} objects (n-dimensional arrays) and fast vectorized operations.
This lecture focuses on:
\begin{itemize}
  \item array creation,
  \item array attributes,
  \item vectorized operations,
  \item and broadcasting.
\end{itemize}

\section{Setup (If Needed)}
If NumPy is not installed:
\begin{lstlisting}
pip install numpy
\end{lstlisting}

\section{Core Concepts}

\subsection{List vs NumPy Array}
Python lists can store mixed types and are not optimized for numerical math.
NumPy arrays store values in a compact block of memory and use fixed \texttt{dtype}.

\begin{lstlisting}[language=Python]
import numpy as np

lst = [1, 2, 3]
arr = np.array([1, 2, 3])

print(lst * 2)   # [1, 2, 3, 1, 2, 3]
print(arr * 2)   # [2 4 6]
\end{lstlisting}

\subsection{Array Creation}
Common creation functions:
\begin{itemize}
  \item \texttt{np.array([...])}
  \item \texttt{np.zeros((r,c))}
  \item \texttt{np.ones((r,c))}
  \item \texttt{np.arange(start, stop, step)}
  \item \texttt{np.linspace(start, stop, num)}
\end{itemize}

\begin{lstlisting}[language=Python]
a = np.zeros((2, 3))
b = np.ones((2, 3))
c = np.arange(0, 10, 2)
d = np.linspace(0, 1, 5)
\end{lstlisting}

\subsection{Array Attributes}
Useful attributes:
\begin{itemize}
  \item \texttt{shape}: tuple of dimensions
  \item \texttt{ndim}: number of dimensions
  \item \texttt{size}: total number of elements
  \item \texttt{dtype}: data type of elements
\end{itemize}

\begin{lstlisting}[language=Python]
m = np.array([[1, 2], [3, 4]])
print(m.shape)  # (2, 2)
print(m.ndim)   # 2
print(m.size)   # 4
print(m.dtype)  # int64 (or similar)
\end{lstlisting}

\subsection{Vectorized Operations}
NumPy applies operations element-wise:
\begin{lstlisting}[language=Python]
x = np.array([1, 2, 3])
print(x + 10)   # [11 12 13]
print(x * x)    # [1 4 9]
\end{lstlisting}

\subsection{Broadcasting (Beginner View)}
Broadcasting allows operations between arrays of different shapes when compatible.
The simplest forms:
\begin{itemize}
  \item array and scalar (scalar applied to all elements)
  \item matrix and vector (vector applied across rows/columns depending on shape)
\end{itemize}

\begin{lstlisting}[language=Python]
a = np.array([[1, 2, 3],
              [4, 5, 6]])
v = np.array([10, 20, 30])
print(a + v)
\end{lstlisting}

\section{Demo Walkthrough}
\textbf{File:} \texttt{demo/numpy\_basics\_demo.py}

Observe:
\begin{itemize}
  \item creating arrays using multiple methods,
  \item applying vectorized operations,
  \item broadcasting with a row vector.
\end{itemize}

\section{Interactive Checkpoints (with Solutions)}

\subsection*{Checkpoint 1 Solution}
\textbf{Question:} why is \texttt{[1,2,3]*2} different?

\textbf{Answer:} list multiplication repeats the list, while NumPy array multiplication is numeric and element-wise.

\subsection*{Checkpoint 2 Solution}
\textbf{Question:} what is broadcasting?

\textbf{Answer:} a rule system that expands smaller shapes to match larger shapes for element-wise operations.

\section{Practice Exercises (with Solutions)}

\subsection*{Exercise 1: Create a 3x3 Array of Ones}
\textbf{Solution:}
\begin{lstlisting}[language=Python]
import numpy as np
a = np.ones((3, 3))
print(a)
\end{lstlisting}

\subsection*{Exercise 2: Square All Elements}
\textbf{Solution:}
\begin{lstlisting}[language=Python]
import numpy as np
x = np.array([1, 2, 3, 4])
print(x * x)
\end{lstlisting}

\section{Exit Question (with Solution)}
\textbf{Question:} name two array creation functions.

\textbf{Answer:} \texttt{np.array}, \texttt{np.zeros} (many answers).

\end{document}

