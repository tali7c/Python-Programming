\documentclass{beamer}

\usetheme{Berlin}
\usecolortheme{Orchid}
\useoutertheme{miniframes}
\setbeamertemplate{navigation symbols}{}

\usepackage{amsmath}
\usepackage{booktabs}
\usepackage{graphicx}
\usepackage{xcolor}
\usepackage{listings}
\usepackage{hyperref}

\lstset{
  basicstyle=\ttfamily\small,
  keywordstyle=\color{blue},
  commentstyle=\color{gray},
  stringstyle=\color{teal},
  showstringspaces=false
}

\title[Python Programming]{Python Programming}
\subtitle{Unit 06 -- Lecture 01: NumPy Basics (ndarray, Operations, Broadcasting)}
\author{Tofik Ali}
\institute{School of Computer Science, UPES Dehradun}
\date{\today}

\begin{document}

\begin{frame}[fragile]
  \titlepage
  \vspace{0.5em}
  \begin{center}
  \footnotesize Repository: \texttt{https://github.com/tali7c/Python-Programming}
  \end{center}
\end{frame}

\begin{frame}[fragile]{Quick Links}
  \centering
  \hyperlink{sec:core}{\beamerbutton{Core Concepts}}\hspace{1em}
  \hyperlink{sec:demo}{\beamerbutton{Demo}}\hspace{1em}
  \hyperlink{sec:interactive}{\beamerbutton{Interactive}}\hspace{1em}
  \hyperlink{sec:summary}{\beamerbutton{Summary}}
\end{frame}

\begin{frame}[fragile]{Agenda}
  \tableofcontents
\end{frame}

\section{Core Concepts}
\label{sec:core}

\begin{frame}[fragile]{Learning Outcomes}
  \begin{itemize}[<+->]
    \item Explain why NumPy is used for numerical computing
    \item Create arrays and understand \texttt{dtype} and shape
    \item Apply vectorized operations on arrays
    \item Explain broadcasting with examples
  \end{itemize}
\end{frame}

\begin{frame}[fragile]{Why NumPy?}
  \begin{itemize}[<+->]
    \item Fast numerical computing (implemented in C)
    \item Compact arrays with fixed data types
    \item Vectorized operations (no Python loops for many tasks)
    \item Foundation for Pandas, ML, and scientific computing
  \end{itemize}
\end{frame}

\begin{frame}[fragile]{List vs NumPy Array}
  \begin{itemize}[<+->]
    \item List: can store mixed types, slower for numeric math
    \item NumPy array: fixed dtype, faster operations
  \end{itemize}
  \vspace{0.4em}
  \begin{lstlisting}[language=Python]
import numpy as np
x = np.array([1, 2, 3])
print(x * 2)  # [2 4 6]
  \end{lstlisting}
\end{frame}

\begin{frame}[fragile]{Array Creation}
  \begin{itemize}[<+->]
    \item \texttt{np.array([...])}
    \item \texttt{np.zeros((r,c))}, \texttt{np.ones((r,c))}
    \item \texttt{np.arange(start, stop, step)}
    \item \texttt{np.linspace(start, stop, num)}
  \end{itemize}
\end{frame}

\begin{frame}[fragile]{Array Attributes}
  \begin{itemize}[<+->]
    \item \texttt{shape}, \texttt{ndim}, \texttt{size}, \texttt{dtype}
  \end{itemize}
  \vspace{0.4em}
  \begin{lstlisting}[language=Python]
a = np.array([[1, 2], [3, 4]])
print(a.shape, a.ndim, a.size, a.dtype)
  \end{lstlisting}
\end{frame}

\begin{frame}[fragile]{Broadcasting (Idea)}
  \begin{itemize}[<+->]
    \item NumPy can apply operations between arrays of different shapes
    \item Example: add a scalar to every element
  \end{itemize}
  \vspace{0.4em}
  \begin{lstlisting}[language=Python]
a = np.array([1, 2, 3])
print(a + 10)  # [11 12 13]
  \end{lstlisting}
\end{frame}

\section{Demo}
\label{sec:demo}

\begin{frame}[fragile]{Demo: Array Creation + Broadcasting}
  \begin{itemize}[<+->]
    \item File: \texttt{demo/numpy\_basics\_demo.py}
    \item Shows:
      \begin{itemize}
        \item creating arrays
        \item vectorized math
        \item broadcasting examples
      \end{itemize}
  \end{itemize}
\end{frame}

\section{Interactive}
\label{sec:interactive}

\begin{frame}[fragile]{Checkpoint 1}
  \textbf{Question:} Why is \texttt{[1,2,3] * 2} different from \texttt{np.array([1,2,3]) * 2}?
\end{frame}

\begin{frame}[fragile]{Checkpoint 2}
  \textbf{Question:} What does broadcasting mean in NumPy?
\end{frame}

\begin{frame}[fragile]{Think-Pair-Share}
  Discuss:
  \begin{itemize}
    \item When would you still use a Python list instead of a NumPy array?
  \end{itemize}
\end{frame}

\section{Summary}
\label{sec:summary}

\begin{frame}[fragile]{Key Takeaways}
  \begin{itemize}[<+->]
    \item NumPy arrays are fast and have fixed \texttt{dtype}
    \item Vectorized operations avoid explicit loops
    \item Broadcasting applies operations across compatible shapes
  \end{itemize}
\end{frame}

\begin{frame}[fragile]{Exit Question}
  Name any two NumPy functions used to create arrays.
\end{frame}

\end{document}

