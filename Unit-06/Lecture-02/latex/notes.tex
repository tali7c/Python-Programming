\documentclass[11pt]{article}
\usepackage[utf8]{inputenc}
\usepackage[T1]{fontenc}
\usepackage{geometry}
\usepackage{amsmath}
\usepackage{hyperref}
\usepackage{xcolor}
\usepackage{listings}
\geometry{margin=1in}

\lstset{
  basicstyle=\ttfamily\small,
  keywordstyle=\color{blue},
  commentstyle=\color{gray},
  stringstyle=\color{teal},
  showstringspaces=false,
  columns=fullflexible,
  frame=single,
  framerule=0.2pt
}

\title{Python Programming\\Unit 06 -- Lecture 02 Notes\\NumPy Functions (Stats, Sorting, Searching)}
\author{Tofik Ali}
\date{\today}

\begin{document}
\maketitle
\tableofcontents

\section{Lecture Overview}
This lecture introduces frequently used NumPy functions for:
\begin{itemize}
  \item array generation,
  \item reshaping,
  \item statistics,
  \item sorting and searching,
  \item and basic linear algebra.
\end{itemize}

\section{Core Concepts}

\subsection{Generating Arrays}
Useful functions:
\begin{itemize}
  \item \texttt{np.zeros}, \texttt{np.ones}
  \item \texttt{np.empty} (uninitialized values)
  \item \texttt{np.arange}, \texttt{np.linspace}
  \item random arrays: \texttt{np.random.randint}, \texttt{np.random.random}
\end{itemize}

\begin{lstlisting}[language=Python]
import numpy as np

a = np.zeros((2, 2))
b = np.ones((2, 2))
c = np.arange(0, 10, 2)
d = np.linspace(0, 1, 5)
r = np.random.randint(1, 10, size=(3, 3))
\end{lstlisting}

\subsection{Reshape}
\texttt{reshape} changes the array shape without changing data:
\begin{lstlisting}[language=Python]
x = np.arange(1, 10)          # size = 9
m = x.reshape((3, 3))         # 3x3 matrix
\end{lstlisting}

\subsection{Statistics and Axis}
Common statistics:
\begin{itemize}
  \item \texttt{sum}, \texttt{mean}, \texttt{std}, \texttt{min}, \texttt{max}
\end{itemize}

Axis meaning for 2D array:
\begin{itemize}
  \item \texttt{axis=0}: operate down the rows (column-wise result)
  \item \texttt{axis=1}: operate across columns (row-wise result)
\end{itemize}

\begin{lstlisting}[language=Python]
print(m.sum(axis=0))  # column sums
print(m.sum(axis=1))  # row sums
\end{lstlisting}

\subsection{Sorting and Searching}
Sorting:
\begin{lstlisting}[language=Python]
arr = np.array([3, 1, 2])
print(np.sort(arr))  # returns new sorted array
arr.sort()           # sorts in-place
\end{lstlisting}

Searching:
\begin{lstlisting}[language=Python]
arr = np.array([10, 50, 20, 80])
idx = np.argmax(arr)
print(idx, arr[idx])
\end{lstlisting}

\subsection{Dot Product and Matrix Multiplication}
\begin{lstlisting}[language=Python]
a = np.array([1, 2, 3])
b = np.array([4, 5, 6])
print(a.dot(b))  # 1*4 + 2*5 + 3*6

M = np.arange(1, 10).reshape((3, 3))
print(M @ M)
\end{lstlisting}

\section{Demo Walkthrough}
\textbf{File:} \texttt{demo/numpy\_functions\_demo.py}

Observe:
\begin{itemize}
  \item generating random arrays,
  \item reshaping and computing stats along axes,
  \item sorting and searching for max/2nd max,
  \item dot product and matrix multiplication.
\end{itemize}

\section{Interactive Checkpoints (with Solutions)}

\subsection*{Checkpoint 1 Solution}
\textbf{Question:} meaning of \texttt{axis=0}?

\textbf{Answer:} compute column-wise results (down the rows).

\subsection*{Checkpoint 2 Solution}
\textbf{Question:} difference between \texttt{np.sort(a)} and \texttt{a.sort()}?

\textbf{Answer:}
\begin{itemize}
  \item \texttt{np.sort(a)} returns a new sorted array.
  \item \texttt{a.sort()} sorts the array in-place (mutates \texttt{a}).
\end{itemize}

\section{Practice Exercises (with Solutions)}

\subsection*{Exercise 1: Row and Column Sums}
\textbf{Task:} Create a 3x3 array and compute row sums and column sums.

\textbf{Solution:}
\begin{lstlisting}[language=Python]
import numpy as np
m = np.arange(1, 10).reshape((3, 3))
print("Row sums:", m.sum(axis=1))
print("Col sums:", m.sum(axis=0))
\end{lstlisting}

\subsection*{Exercise 2: Find Second Maximum}
\textbf{Task:} Find second maximum element of an array.

\textbf{Solution (simple):}
\begin{lstlisting}[language=Python]
import numpy as np
arr = np.array([10, 6, 8, 90, 12, 56])
u = np.unique(arr)
u.sort()
print("Second max =", u[-2])
\end{lstlisting}

\section{Exit Question (with Solution)}
\textbf{Question:} one function to generate sequence?

\textbf{Answer:} \texttt{np.arange} (or \texttt{np.linspace})

\end{document}

