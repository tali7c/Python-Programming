\documentclass[11pt]{article}
\usepackage[utf8]{inputenc}
\usepackage[T1]{fontenc}
\usepackage{geometry}
\usepackage{amsmath}
\usepackage{hyperref}
\usepackage{xcolor}
\usepackage{listings}
\geometry{margin=1in}

\lstset{
  basicstyle=\ttfamily\small,
  keywordstyle=\color{blue},
  commentstyle=\color{gray},
  stringstyle=\color{teal},
  showstringspaces=false,
  columns=fullflexible,
  frame=single,
  framerule=0.2pt
}

\title{Python Programming\\Unit 06 -- Lecture 03 Notes\\Pandas Basics (Series and DataFrame)}
\author{Tofik Ali}
\date{\today}

\begin{document}
\maketitle
\tableofcontents

\section{Lecture Overview}
Pandas is a powerful library for working with tabular data (CSV, Excel-like tables).
It provides:
\begin{itemize}
  \item \textbf{Series} (1D labeled data),
  \item \textbf{DataFrame} (2D table).
\end{itemize}
This lecture focuses on creating and manipulating these structures.

\section{Setup (If Needed)}
If Pandas is not installed:
\begin{lstlisting}
pip install pandas
\end{lstlisting}

\section{Core Concepts}

\subsection{Series}
A Series is like a column with an index.
\begin{lstlisting}[language=Python]
import pandas as pd

s = pd.Series([10, 20, 30], index=["a", "b", "c"])
print(s)
\end{lstlisting}

\subsection{DataFrame}
A DataFrame is a table of columns (each column can be considered a Series).
\begin{lstlisting}[language=Python]
import pandas as pd

df = pd.DataFrame({
    "name": ["Asha", "Bilal", "Charu"],
    "marks": [88, 76, 91]
})
\end{lstlisting}

\subsection{Inspecting Data}
\begin{itemize}
  \item \texttt{df.head()} shows first rows
  \item \texttt{df.tail()} shows last rows
  \item \texttt{df.info()} shows dtypes and missing data summary
  \item \texttt{df.describe()} gives statistics for numeric columns
\end{itemize}

\subsection{Column Selection}
\begin{lstlisting}[language=Python]
marks = df["marks"]
\end{lstlisting}

\subsection{Add/Delete Columns}
\begin{lstlisting}[language=Python]
df["cgpa"] = df["marks"] / 10
del df["cgpa"]
\end{lstlisting}

\subsection{Boolean Filtering}
\begin{lstlisting}[language=Python]
passed = df[df["marks"] >= 50]
\end{lstlisting}

\subsection{Iteration (Use Carefully)}
Pandas is designed for vectorized operations. Iteration is slower but sometimes necessary.
Common ways:
\begin{itemize}
  \item \texttt{df.iterrows()}
  \item \texttt{df.itertuples()}
\end{itemize}

\section{Demo Walkthrough}
\textbf{File:} \texttt{demo/pandas\_basics\_demo.py}

Observe:
\begin{itemize}
  \item creating a DataFrame,
  \item adding new derived columns,
  \item filtering using boolean expressions.
\end{itemize}

\section{Interactive Checkpoints (with Solutions)}

\subsection*{Checkpoint 1 Solution}
\textbf{Question:} difference between Series and DataFrame?

\textbf{Answer:} Series is 1D labeled data; DataFrame is 2D tabular data with columns.

\subsection*{Checkpoint 2 Solution}
\textbf{Question:} how to select \texttt{"marks"} column?

\textbf{Answer:} \texttt{df["marks"]}

\section{Practice Exercises (with Solutions)}

\subsection*{Exercise 1: Create a DataFrame}
\textbf{Task:} Create a DataFrame with columns \texttt{name} and \texttt{age}.

\textbf{Solution:}
\begin{lstlisting}[language=Python]
import pandas as pd
df = pd.DataFrame({"name": ["A", "B"], "age": [18, 19]})
print(df)
\end{lstlisting}

\subsection*{Exercise 2: Create \texttt{cgpa} Column}
\textbf{Task:} Create \texttt{cgpa} from marks.

\textbf{Solution:}
\begin{lstlisting}[language=Python]
df["cgpa"] = df["marks"] / 10
\end{lstlisting}

\section{Exit Question (with Solution)}
\textbf{Question:} one line to create \texttt{cgpa} from marks?

\textbf{Answer:} \texttt{df["cgpa"] = df["marks"] / 10}

\end{document}

