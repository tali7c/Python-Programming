\documentclass[11pt]{article}
\usepackage[utf8]{inputenc}
\usepackage[T1]{fontenc}
\usepackage{geometry}
\usepackage{amsmath}
\usepackage{hyperref}
\usepackage{xcolor}
\usepackage{listings}
\geometry{margin=1in}

\lstset{
  basicstyle=\ttfamily\small,
  keywordstyle=\color{blue},
  commentstyle=\color{gray},
  stringstyle=\color{teal},
  showstringspaces=false,
  columns=fullflexible,
  frame=single,
  framerule=0.2pt
}

\title{Python Programming\\Unit 06 -- Lecture 05 Notes\\Matplotlib Fundamentals (pyplot, labels, grid)}
\author{Tofik Ali}
\date{\today}

\begin{document}
\maketitle
\tableofcontents

\section{Lecture Overview}
Matplotlib is the most commonly used plotting library in Python.
This lecture focuses on basic plotting with \texttt{matplotlib.pyplot}:
\begin{itemize}
  \item line plots,
  \item labels and title,
  \item grid and legend,
  \item saving plots to files.
\end{itemize}

\section{Setup (If Needed)}
If Matplotlib is not installed:
\begin{lstlisting}
pip install matplotlib
\end{lstlisting}

\section{Core Concepts}

\subsection{Basic Plot Workflow}
Typical steps:
\begin{enumerate}
  \item prepare x and y data,
  \item call \texttt{plt.plot(...)} or another plotting function,
  \item add labels/title/legend,
  \item show or save the plot.
\end{enumerate}

\begin{lstlisting}[language=Python]
import matplotlib.pyplot as plt

x = [1, 2, 3, 4]
y = [2, 4, 6, 8]

plt.plot(x, y)
plt.show()
\end{lstlisting}

\subsection{Labels, Title, Grid, Legend}
\begin{lstlisting}[language=Python]
plt.plot(x, y, label="y = 2x")
plt.title("Line Plot")
plt.xlabel("x")
plt.ylabel("y")
plt.grid(True)
plt.legend()
\end{lstlisting}

\subsection{Customization}
You can customize style:
\begin{itemize}
  \item color (\texttt{color="red"})
  \item line style (\texttt{linestyle="--"})
  \item marker (\texttt{marker="o"})
\end{itemize}

\subsection{Saving Plots}
\begin{lstlisting}[language=Python]
plt.savefig("images/line_plot.png", dpi=150)
\end{lstlisting}
\textbf{Tip:} create the output folder before saving.

\section{Demo Walkthrough}
\textbf{File:} \texttt{demo/matplotlib\_line\_plot\_demo.py}

This demo creates a customized plot and saves it into \texttt{images/}.

\section{Interactive Checkpoints (with Solutions)}

\subsection*{Checkpoint 1 Solution}
\textbf{Question:} What does \texttt{plt.grid(True)} do?

\textbf{Answer:} It draws a grid on the plot background to improve readability.

\subsection*{Checkpoint 2 Solution}
\textbf{Question:} Why labels and titles?

\textbf{Answer:} They explain what the plot represents; without labels, a plot can be confusing or misleading.

\section{Practice Exercises (with Solutions)}

\subsection*{Exercise 1: Plot Squares}
\textbf{Task:} Plot $y = x^2$ for $x = 1..10$.

\textbf{Solution:}
\begin{lstlisting}[language=Python]
import matplotlib.pyplot as plt

x = list(range(1, 11))
y = [i*i for i in x]

plt.plot(x, y, marker="o")
plt.title("y = x^2")
plt.xlabel("x")
plt.ylabel("y")
plt.grid(True)
plt.show()
\end{lstlisting}

\section{Exit Question (with Solution)}
\textbf{Question:} function used to save a plot?

\textbf{Answer:} \texttt{savefig}

\end{document}

